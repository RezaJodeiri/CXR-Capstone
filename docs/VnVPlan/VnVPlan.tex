\documentclass[12pt, titlepage]{article}
\usepackage{lscape}
\usepackage{booktabs}
\usepackage{tabularx}
\usepackage{hyperref}
\hypersetup{
    colorlinks,
    citecolor=blue,
    filecolor=black,
    linkcolor=red,
    urlcolor=blue
}
\usepackage[round]{natbib}

%% Comments

\usepackage{color}

\newif\ifcomments\commentstrue %displays comments
%\newif\ifcomments\commentsfalse %so that comments do not display

\ifcomments
\newcommand{\authornote}[3]{\textcolor{#1}{[#3 ---#2]}}
\newcommand{\todo}[1]{\textcolor{red}{[TODO: #1]}}
\else
\newcommand{\authornote}[3]{}
\newcommand{\todo}[1]{}
\fi

\newcommand{\wss}[1]{\authornote{blue}{SS}{#1}} 
\newcommand{\plt}[1]{\authornote{magenta}{TPLT}{#1}} %For explanation of the template
\newcommand{\an}[1]{\authornote{cyan}{Author}{#1}}

%% Comments

\usepackage{color}

\newif\ifcomments\commentstrue %displays comments
%\newif\ifcomments\commentsfalse %so that comments do not display

\ifcomments
\newcommand{\authornote}[3]{\textcolor{#1}{[#3 ---#2]}}
\newcommand{\todo}[1]{\textcolor{red}{[TODO: #1]}}
\else
\newcommand{\authornote}[3]{}
\newcommand{\todo}[1]{}
\fi

\newcommand{\wss}[1]{\authornote{blue}{SS}{#1}} 
\newcommand{\plt}[1]{\authornote{magenta}{TPLT}{#1}} %For explanation of the template
\newcommand{\an}[1]{\authornote{cyan}{Author}{#1}}

The purpose of reflection questions is to give you a chance to assess your own
learning and that of your group as a whole, and to find ways to improve in the
future. Reflection is an important part of the learning process.  Reflection is
also an essential component of a successful software development process.  

Reflections are most interesting and useful when they're honest, even if the
stories they tell are imperfect. You will be marked based on your depth of
thought and analysis, and not based on the content of the reflections
themselves. Thus, for full marks we encourage you to answer openly and honestly
and to avoid simply writing ``what you think the evaluator wants to hear.''

Please answer the following questions.  Some questions can be answered on the
team level, but where appropriate, each team member should write their own
response:


\begin{document}

\title{System Verification and Validation Plan for \progname{}} 
\author{\authname}
\date{\today}
	
\maketitle

\pagenumbering{roman}

\section*{Revision History}

\begin{tabularx}{\textwidth}{p{3cm}p{2cm}X}
\toprule {\bf Date} & {\bf Version} & {\bf Notes}\\
\midrule
Date 1 & 1.0 & Notes\\
Date 2 & 1.1 & Notes\\
\bottomrule
\end{tabularx}

~\\
\wss{The intention of the VnV plan is to increase confidence in the software.
However, this does not mean listing every verification and validation technique
that has ever been devised.  The VnV plan should also be a \textbf{feasible}
plan. Execution of the plan should be possible with the time and team available.
If the full plan cannot be completed during the time available, it can either be
modified to ``fake it'', or a better solution is to add a section describing
what work has been completed and what work is still planned for the future.}

\wss{The VnV plan is typically started after the requirements stage, but before
the design stage.  This means that the sections related to unit testing cannot
initially be completed.  The sections will be filled in after the design stage
is complete.  the final version of the VnV plan should have all sections filled
in.}

\newpage

\tableofcontents

\listoftables
\wss{Remove this section if it isn't needed}

\listoffigures
\wss{Remove this section if it isn't needed}

\newpage

\section{Symbols, Abbreviations, and Acronyms}

\renewcommand{\arraystretch}{1.2}
\begin{tabular}{l l} 
  \toprule		
  \textbf{symbol} & \textbf{description}\\
  \midrule 
  T & Test\\
  \bottomrule
\end{tabular}\\

\wss{symbols, abbreviations, or acronyms --- you can simply reference the SRS
  \citep{SRS} tables, if appropriate}

\wss{Remove this section if it isn't needed}

\newpage

\pagenumbering{arabic}

This document ... \wss{provide an introductory blurb and roadmap of the
  Verification and Validation plan}

\section{General Information}

\subsection{Summary}
There are multiple software components that are required for us to complete our end goal, as previously mentioned in SRS and Hazard Analysis, we need to test each of these components to ensure accuracy and reliability in our product. More of the specifics can be found below.  
\subsubsection{User Interface (UI)}
\textbf{Function}: Ensure that the interface is easy to navigate for users like radiologists, doctors, or technicians. This includes testing for intuitive layouts, clear labels, and smooth workflows.
\begin{itemize}
    \item \textbf{Functionality}: Check if every UI element this includes: buttons, drop downs, the menu and ensure the software front ended system responds correctly to user input.
    \item \textbf{Performance}: Verify the UI’s responsiveness and loading times, particularly when displaying large medical images like X-rays.
\end{itemize}

\subsubsection{Image Preprocessing and Enhancement}
\textbf{Function}: Before analysis, the X-ray images undergo preprocessing steps to enhance image quality and ensure consistent inputs for the AI model.
\begin{itemize}
    \item \textbf{Accuracy of Noise Reduction}: Removes artifacts and noise from X-rays, improving image clarity. This is usually done by making the image grey scale or removing color from the image to ensure further accuracy when reading the image. 
    \item \textbf{Normalization}: Standardizes the pixel values across images to ensure uniformity, allowing the AI model to better detect subtle differences.
\end{itemize}

\subsubsection{Processing The Image}
\textbf{Function}: The AI model reads the image and looks at the features or patterns from the X-ray images that are indicative of various lung conditions.
\begin{itemize}
    \item \textbf{Accuarcy of Texture and Shape Analysis}: System analyzes shape and density of lung tissues by looking at tensor values and identifying anomalies.
\end{itemize}
\subsubsection{Disease Classification}
\textbf{Function}: Once features are extracted, using AI classification system categorizes pattern  into different diseases; returning the probabilities.
    \begin{itemize}
        \item \textbf{Accuracy and speed of Disease Models}: Model is trained on large datasets of labeled X-ray images specifically CheXRPT, aiding it to recognize the patterns associated with diseases, this includes: Pneumonia, Lung Cancer, Tuberculosis and more.
        \item \textbf{Accuracy  of Probability Scores}: It provides a probability or confidence score for each disease, indicating the likelihood of the condition being present.
    \end{itemize}
    \textbf{Function}: Integrate our application hospital systems and hospital imaging software  for seamless data flow.
    \begin{itemize}
            \item \textbf{Functionality of Automated Reporting}: Upon completing the analysis, the software can generate the report showing the probabilities of diseases and save that information onto the patients hospital data or user data of that hospital.
    \end{itemize}
    \subsection{Objectives}
% add a more general paragraph on the top.
\subsubsection{Ensure Software Accuracy or Correctness (In scope)}
\begin{itemize}
    \item The primary objective of this project is to build confidence in the correctness of the AI-powered diagnostic tool for lung diseases. 
    \item The software should be able to consistently and accurately predict the with a score of a minimum of 85$\%$ accuracy. 
    \item This probability will be derived from the model estimating the odds of  various lung conditions for example,  pneumonia or lung cancer from X-ray images. 
    \item Verifying the correctness of the AI's predictions is essential to ensure reliable clinical decision-making.
\end{itemize}
\subsubsection{Demonstrate Adequate Usability (In scope)}
\begin{itemize}
    \item Given that medical professionals, such as radiologists, students, and doctors, will use the tool, the UI needs to be intuitive and efficient. 
    \item The focus will be on ensuring the tool’s interface is easy to navigate, with clear access to essential features such as uploading X-rays, viewing probabilities, and generating reports.
    \item This probability will be derived from the model estimating the odds of  various lung conditions for example,  pneumonia or lung cancer from X-ray images. 
    \item  The goal of the application is to ensure correct results and not create an inner challenge to use the software itself, so users can quickly perform key tasks.
\end{itemize}
\subsubsection{Security of Patient Data (In scope)}
\begin{itemize}
    \item Since the software will handle sensitive patient information, it must comply with privacy regulations such as PIPEDA for Canada or HIPAA if used in the US. 
    \item This objective will focus on verifying that patient data is securely stored and transferred, and that access control mechanisms are in place to protect against unauthorized use.
\end{itemize}

\subsubsection{Verification of External Libraries and APIs (Out of scope)}
\begin{itemize}
    \item  The system relies on pre-existing libraries and APIs for tasks such as image processing, patient data handling, and possibly pre-trained AI models.
    \item The quality of these external dependencies will not be verified in this project.
    \item Our team believes that the External libraries and APIs will be assumed to have been tested by their implementation teams.
\end{itemize}
\subsubsection{Extensive Usability Testing (Out of scope)}
\begin{itemize}
    \item Ensuring we do adequate usability testing is a priority, however going further than that is a  problem for the longer future as we have limited resources, only basic usability testing will be done to ensure the software meets general usability standards. 
\end{itemize}
\subsubsection{Non Canadian/American Regulatory Compliance (Out of scope)}
\begin{itemize}
    \item The project will not focus on ensuring compliance with other  health data privacy regulation, for example, the General Data Protection Regulation in the European Union. As we are currently only looking to launch this application for North America more specifically Canada and the United States. 
\end{itemize}

\subsection{Challenge Level and Extras}

\subsubsection{Challenge Level}

Our team believes the challenge level for this project to be advanced. This classification is based on the complexity of integrating AI for diagnostic purposes, the necessity for robust data handling and privacy compliance, and the need to develop a user-friendly interface for medical professionals. The project requires a solid understanding of machine learning principles, software development practices, and compliance with health data regulations. Our knowledge of Tensors, Linear Algebra, machine learning, and image processing will be essential for this application to function effectively. \\
In terms of feasibility, our approach includes specific strategies to handle data privacy and large-scale data management. We plan to adhere to compliance regulations (such as HIPAA guidelines) and use anonymized datasets where necessary. Additionally, our team has access to sufficient computational resources and relevant medical datasets, ensuring we can manage the data and AI components of the project effectively. Links to some resources are below:\\
\url{https://youtu.be/lvXc3O2n56w?si=CdYoJH9WRrPKQU3i}
\url{https://stanfordmlgroup.github.io/competitions/chexpert/} \\ 
We have also identified several "extras" to enhance the application, such as incorporating a user feedback loop and additional image processing features that could benefit medical professionals and future development phases. Our System Verification and VAlidation plan will include cross-validation and real-world testing scenarios to ensure accuracy and reliability in a clinical context. \\
\subsubsection{Extras}

\textbf{Extra Testing:}
As mentioned in SRS and in the previous sections: Basic usability testing will be conducted to ensure that the user interface is intuitive and easy for healthcare professionals to navigate. This will include gathering feedback from potential users to identify any usability issues and make necessary adjustments. \\\\  
\textbf{Feature one:}
One additional feature we plan to add, is a quick notification system to the people who are responsible patient. For instance, if the patient was to have a serious condition, for example, benign cancer in that region of his lungs, his physician would be immediately updated via SMS or other communication platforms. \\\\  
\textbf{Feature Two:}
Another feature we can look to add to our application is a checklist of the patients symptoms this would be used in a verification role. For example, if the patient facilitated his symptoms to be shortness of breath, the application would use this information and incorporate in the AI trained model, returning diseases which could have this potential symptom.\\  

\subsection{Relevant Documentation}

The three main relevant documentation that helped guide us in System Verification and Validation Plan (VnV) was Software Requirements Specification (SRS), Hazard Analysis (HA), and development plan. SRS was crucial for the successful development of the AI-powered diagnostic tool for lung diseases, as it clearly outlines the software’s intended functionality, performance requirements, and design constraints, providing a road map for us developers and aiding us understand our stakeholders.  The problem statement identifies the specific challenges to be addressed, while the development plan documented helped us by outlining what approach and resources were needed to achieve our project goals. Lastly, Hazard Analysis helped us identify security and safety requirements, with that in mind we were able to test the system more thoroughly. As HA helped us derive requirements for our implementation plan by creating by additional mitigation of health and safety risks. For example, if our server went down, we would not potentially fulfill the Service Level Agreement due to the downtime of the system. However, through HA we were able to create counter measures to ensure when things like that happen, those risks are mitigated.These documents are very relevant as they helped ensure that the SRS aligns with the project’s objectives, guiding the VnV process by defining acceptance criteria, enabling the creation of test cases linked to specific requirements, and establishing a traceability matrix to ensure all functionalities are tested. Additionally, the SRS aids in managing changes, identifying risks, and ensuring compliance with regulations like PIPEDA or HIPAA, ultimately enhancing quality assurance practices and helping to deliver a reliable and effective software product.



\citet{SRS}

\wss{Don't just list the other documents.  You should explain why they are relevant and 
how they relate to your VnV efforts.}

\section{Plan}

\wss{Introduce this section.  You can provide a roadmap of the sections to
  come.}

\subsection{Verification and Validation Team}

\wss{Your teammates.  Maybe your supervisor.
  You should do more than list names.  You should say what each person's role is
  for the project's verification.  A table is a good way to summarize this information.}

\subsubsection{Understanding Medical Conditions}
\textbf{Assigned Team Members: All} \\\\
For the verification process to ensure that the AI accurately detects lung diseases in chest X-rays, all team members must gain a deep understanding of the relevant medical conditions. This is essential for assessing whether the AI outputs align with real-world clinical expectations. To achieve this, the team needs to thoroughly review annotated chest X-ray datasets and medical literature. Without this foundational knowledge, the team cannot effectively validate the model’s diagnostic accuracy. During verification, each member must be capable of recognizing potential errors in the model’s predictions and correlating them with known medical indicators, ensuring the AI delivers clinically relevant results.
\subsubsection{Learning PyTorch Fundamentals}
\textbf{Assigned Team Members: Patrick, Reza, and Kelly}\\\\
To achieve high-quality verification of the AI model, the team must master PyTorch’s core functionalities, particularly related to the data flow and training process. The team needs to understand how to verify that the model is not just functioning, but also learning and improving in a meaningful way. This includes using PyTorch’s tensor operations and autograd features to test whether gradients are being calculated and applied correctly during backpropagation. Additionally, the team must verify that the preprocessing steps—such as resizing, normalizing, and augmenting X-ray images—are executed properly, as incorrect data handling could introduce significant errors during model training. High-quality verification requires confirming that all data transformations support the model’s learning objectives and that the AI is robust across different image variations.
\subsubsection{Data Visualization Analysis}
\textbf{Assigned Team Members: Ayman and Nathan}\\\\ 
For effective verification, the team must ensure that data visualizations clearly reflect the AI model’s performance, making it easier to spot anomalies or trends that might indicate issues. Ayman needs to focus on developing visuals that accurately represent both the training progress and the diagnostic results from the AI model. The team must verify that these visualizations allow for easy comparison between predicted outcomes and the actual clinical diagnoses. Ensuring that the plots are interpretable and correctly display model metrics (e.g., accuracy, loss over time, false positives/negatives) is key to validating the AI’s reliability. Without this verification, it would be difficult to pinpoint areas where the model may require refinement.


\subsubsection{Web Creation}
\textbf{Assigned Team Members: Ayman and Nathan} \\\\
To meet the high standards for user experience and functionality, the Nathan and Ayman must rigorously verify that the frontend interface is not only visually appealing but also fully functional and responsive. Nathan needs to verify that the UI design facilitates a seamless interaction with the AI model, ensuring that healthcare professionals can input data and interpret results with ease. It is critical to validate the integration between the frontend and backend, ensuring that data is accurately retrieved, processed, and displayed without errors. Usability testing must be a priority during verification, as any gaps in the user interface could hinder the adoption of the tool by medical professionals. Ensuring accessibility and performance across various devices is also essential to meet the project’s quality standards.

\subsubsection{Users of the Application}
\textbf{Assigned Team Members: All} \\\\
Understanding user expectations is critical for verifying that the final product meets the needs of its primary users—healthcare professionals. The team must ensure that the AI model's outputs are not only accurate but also presented in a format that healthcare users find intuitive and actionable. Verification must include gathering feedback from potential users to assess whether the AI-generated diagnostics align with clinical workflows. Additionally, the team needs to verify that the interface provides relevant and precise information without overwhelming the user with unnecessary details. Ensuring that the tool is user-friendly and caters to both specialists and general practitioners is key to validating its practicality and effectiveness in real-world scenarios.


\subsection{SRS Verification Plan}
\subsubsection{Peer Review}
\begin{itemize}
    \item \textbf{Ad Hoc Feedback:} We will conduct peer reviews where classmates and our primary reviewer provide feedback. 
    \item \textbf{Creating Issues:} Each team member will be responsible for reading the SRS, evaluating its clarity, and checking for missing requirements and for anything missing that team member will required to make a github issue entailing that we need to look to add or fix a section of SRS. 
    \item \textbf{Peer reviews:} Supervisor, stakeholders, and other peers will read our the document and give use feedback by doing this we can ensure its understandable to all stakeholders, especially those unfamiliar with the project.
    \item \textbf{Pre-Meeting Preparation:} Before meeting with the supervisor, the team will ensure that the SRS is updated and ready for review. We would this by going through the rubric and creating a checklist. This checklist would include: Questions like are all the functional and non-functional requirements included? Are the constraints in scope or outside scope constraints are they truly constraints. Making a checklist with questions similar to these will help verify that the project is coherent and comprehensive.
    \item \textbf{Post-Meeting Notes:} we will take notes on what the supervisor or TA told us and ensure that we look to add or fix whatever was mentioned during the meeting. This will be another checklist of things we must fix not like the previous checklist that is primarily focused on asking questions. 
    \item \textbf{Questions for the Supervisor $\&$ TA's:} We will ask key questions that would focus on verifying the scope, clarity, and feasibility of our SRS document. An example, could be asking if the NFRs or non-functional requirements align with the project’s quality objectives.
\end{itemize}

\subsection{Design Verification Plan}

The design verification plan outlines the strategies and procedures that the team will use to verify the correctness and reliability of the design of our X-ray analysis and disease prediction application. This plan will serve as a guideline during the testing phase to ensure that the design meets the intended requirements and can mitigate potential risks identified by the team. The following procedures will be undertaken by the testing team during the verification process:

\subsubsection{Document Review}
The system's design documentation and related materials will be reviewed by each member of the testing team after the initial draft is produced. During the document review process, the testing team will:
\begin{itemize}
\item Ensure that the design of the system aligns with all functional and non-functional requirements, particularly those related to medical accuracy, data security, and regulatory compliance.
\item Assess if the documentation accurately describes the intended functionalities and behavior of the system, including its ability to analyze X-rays, classify diseases, and provide accurate predictions. Any discrepancies between the design and requirements should be recorded and discussed further.
\end{itemize}

\subsubsection{Prototype Testing}
After the scheduled Proof of Concept (POC) demonstration, a prototype for each major system component should be completed. The prototypes will be assembled for system testing. During this phase, the testing team will:
\begin{itemize}
\item Verify the application’s ability to process X-ray images accurately and consistently.
Assess system performance under different load conditions, such as batch processing multiple images simultaneously.
\item Test the robustness of disease prediction algorithms under various scenarios, including rare and complex cases.
\item Evaluate the application's user interface for healthcare professionals, ensuring it is intuitive, provides clear predictions, and integrates well with clinical workflows.
Simulate failure conditions to verify error-handling mechanisms, such as incomplete data, corrupted files, or system downtime.
\end{itemize}


\subsubsection{Design Verification Checklist}

\textbf{Document Review}
    \begin{itemize}
        \item Ensure design aligns with all functional and non-functional requirements.
        \item Confirm documentation accuracy for image processing, classification, and prediction.
        \item Record any discrepancies from requirements.
    \end{itemize}
\textbf{Prototype Testing}
\begin{itemize}
    \item \textbf{System Performance}
    \begin{itemize}
        \item Verify accurate X-ray image processing.
        \item Test performance under various load conditions.
        \item Check response time for predictions.
        \item Ensure scalability for future usage.
    \end{itemize}

\item \textbf{Disease Prediction Accuracy}
    \begin{itemize}
        \item Test accuracy with a benchmark dataset.
        \item Evaluate algorithm on rare and complex cases.
        \item Check consistency of predictions across runs.
    \end{itemize}

    \item \textbf{User Interface (UI)}
    \begin{itemize}
        \item Confirm UI is intuitive for healthcare professionals.
        \item Ensure clear display of disease predictions.
        \item Verify data visualization tools are easy to interpret.
        \item Check seamless integration with clinical systems.
    \end{itemize}

    \item \textbf{Failure Handling}
    \begin{itemize}
        \item Test response to incomplete or corrupted data.
        \item Verify clarity of error messages.
        \item Simulate network failures for recovery and integrity.
        \item Check logging for error tracking.
    \end{itemize}

    \item \textbf{Security and Compliance}
    \begin{itemize}
        \item Confirm encryption and patient privacy protections.
        \item Verify compliance with healthcare data regulations.
        \item Ensure access control mechanisms are in place.
    \end{itemize}
\end{itemize}
\subsection{Verification and Validation Plan Verification Plan}

This section provides guideline for Quality Assurance team to validate the Verification and Validation Plan document. The following steps will be used:
\begin{itemize}
  \item{Cross checking with Functional Requirements from SRS document: VnV documentation must objectively cover all business use-cases and functional requirements verification plan.}
  \item{Review uncertenties with stakeholders: VnV plan must be reviewed by the stakeholders to ensure that the plan is feasible, realistic, and complete.}
  \item{Each test case mentioned will have to state the following:}
  \begin{itemize}
    \item{What is the expected input and output of the test case.}
    \item{How the test case is going to be implemented (Using what tools, techniques).}
    \item{Which functional requirements, or non-functional requirements (from SRS document) does this test case belongs to.}
    \item{Which system is being tested (Front-end, Back-end, ML model, etc.).}
    \item{Which team is responsible for the test case. For example, a end-to-end test case might require everyone's effort}
  \end{itemize}
\end{itemize}

\subsection{Implementation Verification Plan}
The following testing techniques will be used to verify the implementation our system:
\begin{itemize}
  \item{\textbf{Static Code and Documentation walkthroughs}}
  \begin{itemize}
    \item{Documentation walkthroughs will be performed by the team members, as a group, to verify that business use-cases and functional requirements are all covered.}
    \item{Code walkthroughs/ Peer reviews will be performed by the team to ensure that the code is written as per the design documents (VNV Plan, SRS, and Development Plan).}
  \end{itemize}

  \item{\textbf{Unit Testing}}
  \begin{itemize}
    \item{Individual units of the software are tested in isolation from the rest of a particular system. This is done to ensure that each unit of the software is working as designed in the business use-cases.}
    \item{Examples of unit-test includes: testing the input validation, testing the output of a function, testing button clicks, testing authenticated web pages, etc.}
    \item{All unit tests must cover the happy path and edge cases (boundary conditions, invalid input, etc.).}
  \end{itemize}
  
  \item{\textbf{Integration Testing}}
  \begin{itemize}
    \item{Two Individual Systems's Integration are tested. This is done to ensure that the system is working together as designed in the business use-cases.}
    \item{In the context of Neuralanalyzer, required integrations to test are: the front-end and the back-end, back-end with AWS Cognito, back-end with Neuralanalyzer ML model, back-end with TorchXRayVision etc.}
    \item{Examples of system tests include: Testing back-end responses with different input from front-end , Testing ML prediction with different images input onto backend, etc.}
    \item{All Integration tests are ideally to be tested in isolation, for example: front-end to back-end integration test should mock the ML model's responses.}
    \item{All system tests must cover the happy path and edge cases (boundary conditions, invalid input, etc.).}
  \end{itemize}
  \item{\textbf{System/ End-to-End Testing}}
  \begin{itemize}
    \item{System testing is done to ensure that all systems are working together as designed in the business use-cases.}
    \item{Example of 1 System test cases: Sign up on the front-end and validating the existing user on the back-end, etc.}
    \item{All end-to-end tests must cover all business use-cases (Log-in, sign-up, image-upload, prediction, etc.), and Functional Requirements.}
  \end{itemize}
  \item{\textbf{Periodic Health Checks}}
  \begin{itemize}
    \item{The front-end, back-end, and ML model will be checked periodically to measure system's uptime. Measured uptime must satisfy the agrred-upon SLA (95\%).}
    \item{Periodic Health Checks will be performed regardless of the system' software version and state.}
  \end{itemize}
  \item{\textbf{Manual Testing}}
  \begin{itemize}
    \item{Manual testing will be performed by the team to ensure intuitive and simple UI/UX to the end-user.}
    \item{Manual testing will only include interaction with the front-end on different devices (mobile, tablet, desktop).}
    \item{All manual tests must cover all business use-cases (Log-in, sign-up, image-upload, prediction, etc.).}
  \end{itemize}
  \item{\textbf{Machine Learning Model Validation and Testing}}
  \begin{itemize}
    \item{All the Chest X-Ray data available will be split into 3 diffrent sets: Training, Validation, and Testing.}
    \item{The ML model will be trained on the Training set, and validated on the Validation set.}
    \item{The ML model will be tested on the Testing set, and the accuracy will be measured. The accuracy of the models will be aggreed upon by the team, and the stakeholders.}
  \end{itemize}
  \item{\textbf{User Acceptance Testing}}
  \begin{itemize}
    \item{After the MVP is ready, the team will invite selected stakeholders to test out the system.}
    \item{Based on the feedback, the team will make necessary changes to the system via GitHub Issues.}
  \end{itemize}
\end{itemize}

\subsection{Automated Testing and Verification Tools}
With many testing techniques in place, the following tools will be used to automate the testing process:
\begin{itemize}
  \item{\textbf{Static Code/ Document Review via Github Pull Requests}}
  \begin{itemize}
    \item{During development phase, every line of code ideally would have to be reviewd, per \textbf{Github Pull Requests}, by the team. As mentioned, pull requests onto main branch must be approved by all 4 team members.}
    \item{If a pull request is way too long (ie: more than 500 lines of code), the pull request owner will have to break down the pull request into smaller pull requests, or the team will have a meeting to review the code.}
    \item{Since the documentation is also version controlled, the team will review documents similar to code review via \textbf{Github Pull Requests}.}
  \end{itemize}

  \item{\textbf{Static Code Analysis Tools}}
  \begin{itemize}
    \item{Utilizing \textbf{Pylint} for Python code, and \textbf{ESLint} for JavaScript code To ensure that the code is written as per the PEP8 standards.}
    \item{Pylint and ESLint will be integrated into the development environment, and will be run on every commit to the main branch via \textbf{Gtihub Actions}.}
  \end{itemize}

  \item{\textbf{Tools for Unit Testing}}
  \begin{itemize}
    \item{\textbf{Front-end}: Utilizing \textbf{Jest} for testing React components, and front-end data fetching functions. Jest is capable of mocking API calls and other front-end modules, isolating front-end units at test-execution time, making it a perfect tool for React unit testing.}
    \item{\textbf{Back-end and ML plugins}: Utilizing \textbf{Pytest} for testing individual functions and layers. Pytest is capable of mocking front-end input, external service calls, and other back-end modules.}
  \end{itemize}

  \item{\textbf{Tools for Integration Testing}}
  \begin{itemize}
    \item{Front-end and back-end integration will be tested with \textbf{Cypress}. Certain function that requires ML model's responses can be mocked with Cypress at the api layer.}
    \item{Back-end and ML Plugins integration will be tested with \textbf{Pytest}. Certain function that requires Front-end input can be mocked at the API layer.}
    \item{Back-end and AWS Cognito integration will be tested with \textbf{Pytest}. Certain functions that requires Front-end input and ML models output can be mocked.}
  \end{itemize}

  \item{\textbf{Tools for End-to-End Testing}}
  \begin{itemize}
    \item{End-to-end testing can be done with \textbf{Cypress}. Without mocking any functions, end-to-end testing can be performed in a fashion that mimics the user's interaction with the system.}
    \item{For example Cypress can create an account on the front-end, and validate if the account is created on AWS Cognito.}
  \end{itemize}

  \item{\textbf{Tools for Periodic Health Checks}}
  \begin{itemize}
    \item{\textbf{Github Actions Schedules} will be used to run health checks on running front-end and back-end.}
    \item{Since AWS Cognito's uptime is fully mamanged by AWS, perioid health check for this service is unnecessary.}
  \end{itemize}

  \item{\textbf{Test Automation Framework}}
  \begin{itemize}
    \item{\textbf{GitHub Actions} will be utilized to automate the testing process.}
    \item{GitHub Actions will be configured to run all tests (All unit tests, integration tests, and system tests) on every Pull Request, and on every commit to the main branch.}
  \end{itemize}

  \item{\textbf{Machine Learning Model Validation and Testing}}
  \begin{itemize}
    \item{\textbf{Pytorch} will be used to test the ML model's accuracy on the Testing set.}
  \end{itemize}
\end{itemize}

\subsection{Software Validation Plan}
\begin{itemize}
  \item{\textbf{Developers}: After a feature is developed, the internal validation plan will includes the following steps}
  \begin{itemize}
    \item{Developers will write tests, and come up with acceptance criteria for the feature.}
    \item{The test caes then will be reviewd (on Github), and ran on Github Actions on every Pull Requests.}
    \item{The software is consider validated, if all tests pass, including all old tests, and newly written tests.}
    \item{If the test cases failed, the developer will need to fix the code, and notify the team to re-approve the Pull Requests.}
  \end{itemize}
  \item{\textbf{Dr.Mehdi Moradi}: Since Dr.Mehdi Moradi is the primary stakeholder of the project, the team will report project progress to him on a bi-weekly basis. The team will also invite Dr.Mehdi to internal testing sessions.}
  \item{\textbf{Other Stakeholders}: During the development cycle of the application, User Acceptance Testing (UAT) will be performed periodically by the team. UATs will be performed to ensure that the system is working as expected and is ready for the stakeholders to test. The UAT will be performed in the following steps:}
  \begin{itemize}
    \item{Coming up with business use-cases, by stakeholders, that the system must satisfy.}
    \item{Selected stakeholders will be invited to use and test the beta system.}
    \item{Stakeholders interaction with the system will be observed and feedback will be collected.}
    \item{If the interactions and feedbacks are positive, the current system will be deployed to production. However, if the feedbacks are mostly negative, the team will make necessary changes to the system via GitHub Issues, and perform another round of UAT.}
  \end{itemize}
\end{itemize}


\section{Unit Test Description}

\wss{This section should not be filled in until after the MIS (detailed design
  document) has been completed.}

\wss{Reference your MIS (detailed design document) and explain your overall
philosophy for test case selection.}  
\\ Document is not completed yet.

\wss{To save space and time, it may be an option to provide less detail in this section.  
For the unit tests you can potentially layout your testing strategy here.  That is, you 
can explain how tests will be selected for each module.  For instance, your test building 
approach could be test cases for each access program, including one test for normal behaviour 
and as many tests as needed for edge cases.  Rather than create the details of the input 
and output here, you could point to the unit testing code.  For this to work, you code 
needs to be well-documented, with meaningful names for all of the tests.}

\subsection{Unit Testing Scope}

\wss{What modules are outside of the scope.  If there are modules that are
  developed by someone else, then you would say here if you aren't planning on
  verifying them.  There may also be modules that are part of your software, but
  have a lower priority for verification than others.  If this is the case,
  explain your rationale for the ranking of module importance.}
\subsection{Unit Testing Scope}
The modules which the unit test focuses on will be divided into two categories: \textbf{Front-end} and \textbf{Back-end}.\\\\
The following modules are considered outside the scope of unit testing:\\\\
\textbf{Third-Party Modules}: 
Any third-party libraries or frameworks integrated into the application will not be subjected to unit testing. This includes modules developed by external sources, such as libraries for data visualization or authentication. These modules will be relied upon for their documented functionality, but no direct verification will be performed.


\subsection{Rationale for Ranking}
The prioritization of modules for testing is based on the following criteria:
\begin{itemize}
    \item \textbf{Impact on Core Functionality}: Modules that are integral to the application’s main features will receive higher priority for unit testing.
    \item \textbf{User Experience}: Features that directly affect user interactions and overall experience will be tested more rigorously.
    \item \textbf{Risk Assessment}: Modules with higher associated risks, such as security and data handling, will be prioritized to mitigate potential issues.
\end{itemize}

By focusing our testing efforts on high-impact modules, we aim to ensure a robust and reliable application while efficiently utilizing our resources.


\subsection{Tests for Functional Requirements}

\wss{Most of the verification will be through automated unit testing.  If
  appropriate specific modules can be verified by a non-testing based
  technique.  That can also be documented in this section.}\\\\
For unit testing, we’re dividing the process into frontend and backend sections. UI/UX interactions, such as buttons and links, will be tested manually to ensure they function as expected. Backend functions, especially those involving data fetching, will be tested automatically.\\\\
Frontend testing: UI/UX elements such as buttons, links, and other interactive components will undergo manual testing to confirm that they work as intended from a user perspective. This includes verifying that each interactive element responds accurately and meets usability standards.\\\\
Backend testing: functions that handle data fetching and processing will be validated through automated tests. These tests will simulate various scenarios to ensure that data retrieval, processing, and any associated logic work as expected, helping us catch issues early and maintain data integrity. This combination of manual and automated testing ensures comprehensive coverage and reliability across the application. (Possibly by Pytest library)

\subsubsection{Front End}

\wss{Include a blurb here to explain why the subsections below cover the module.
  References to the MIS would be good.  You will want tests from a black box
  perspective and from a white box perspective.  Explain to the reader how the
  tests were selected.}\\\\
  The frontend testing will encompass all possible user interactions, covering any element that the user can click or interact with. This includes, but is not limited to, buttons, links, dropdowns, modals, and form fields. Each interactive element will be manually tested to confirm that it behaves as expected in various scenarios, such as different screen sizes and user flows. This ensures that the user experience is smooth, intuitive, and functional across all devices and use cases. Anything that deals with username/password/upload/download should be securely designed.
\begin{enumerate}

\item{FR-FE-1\\} (FR-7, FR-13)

Type: Manual
			
Initial State:  Automated
					
Input: Username \& Password
					
Output: User is successfully logged in and redirected to the home page if the username and password are correct. If incorrect, an error message is displayed, preventing login.

Test Case Derivation: The output is expected to confirm whether the login process works correctly. A successful login with correct credentials should redirect the user to the home page, while incorrect credentials should display an error message, preventing access. This expected outcome ensures that the system only grants access to users with valid login information, maintaining security and usability standards.

How test will be performed: An automated

[Refrence to Login Form UI]

\item{FR-FE-2\\} (FR-7, FR-13)

Type: Automated
					
Initial State: Register Page
					
Input: All form fields: Include but not limited to username/email, first name, last name, occupation, organization, location, password and confirmed password.
					
Output: User is successfully registered and redirected to the login page if all fields are valid; otherwise, appropriate error messages are displayed for invalid inputs.

Test Case Derivation: This test verifies that the registration functionality correctly processes all required fields. The expected output is based on whether the provided data meets validation criteria and does not contain duplicates.

How test will be performed: Automated scripts will fill in all form fields with valid and invalid data, submitting the registration form. The outputs will be checked to ensure successful registration for valid data and the correct error messages for invalid inputs.
[Refrence to Register Form UI]

\item{FR-FE-3\\} (FR-1, FR-12, FR-15)

Type: Automated
					
Initial State: Upload Page 
					
Input: One or more pictures in various formats (e.g., JPG, PNG, etc.) and within constrained size.
					
Output: Images are successfully uploaded, and confirmation messages are displayed. For unsupported formats, appropriate error messages are shown.

Test Case Derivation: This test verifies the image upload functionality, ensuring that the system accepts valid image formats while rejecting unsupported ones. The expected outcome is determined by the format and size of the uploaded images.

How test will be performed: Automated scripts will upload a mix of supported and unsupported image formats and sizes to the upload page. The results will be checked for successful uploads or correct error messages for unsupported formats.
[Refrence to Upload Page UI]

\item{FR-FE-4\\} (FR-3)

Type: Automated
					
Initial State: Upload Page (After Upload)
					
Input: Symptoms
					
Output: Symptoms are successfully recorded and displayed on the user interface, confirming submission; an error message appears for invalid input.

Test Case Derivation: This test verifies that the system correctly processes and records user-submitted symptoms after an image upload.

How test will be performed: Automated scripts will input various symptom descriptions, including valid and invalid examples, to the appropriate field. The results will be checked to ensure successful recording or the display of correct error messages for invalid entries.

\item{FR-FE-5\\} (FR-1, FR-12, FR-15)

Type: Automated
					
Initial State: Upload Page (Any)
					
Input: Delete, Save or Submit
					
Output: User’s symptoms and uploaded images are deleted, User’s symptoms and uploaded images are saved under profile; User's symptoms and uploaded images are submitted to backed. Confirmation message will be displayed or if there are validation errors, appropriate error messages are shown.

Test Case Derivation: The expected output depends on the action taken, ensuring that data is either saved locally or submitted to the backend as appropriate.

How test will be performed: Automated scripts will execute the Save or Submit actions and subsequently assess the actual outcomes against the expected results to ensure compliance with the specified requirements.

\item{FR-FE-6\\} (FR-13)

Type: Manual
					
Initial State: Home Page 
					
Input: Actions to navigate to other pages. (Upload, Profile, Logout-redirects to Login page etc)
					
Output: The user is successfully redirected to the selected page.

Test Case Derivation: This test ensures that navigation actions from the Home Page function correctly, leading users to the appropriate pages without errors.

How test will be performed: All the buttons will be manually tested, and observed for functionality.

\item{FR-FE-7\\} (FR-1, FR-12, FR-15)

Type: Automated
					
Initial State: Report Page (Saved)
					
Input: Delete or Submit
					
Output: Report page will be deleted or Report page will go to submitted state.

Test Case Derivation: This test verifies that the system correctly processes the actions of deleting or submitting a report, ensuring appropriate status changes and user feedback.

How test will be performed: Automated scripts will simulate the selection of "Delete" and "Submit" actions, checking for the expected outcomes—either the removal of the report or the confirmation of submission.

\item{FR-FE-8\\} (FR-5 FR-6 FR-7 FR-11 FR-13)

Type: Automated
					
Initial State: Report Page (Submitted)
					
Input: N/A
					
Output: Fetch and display the report from the database; otherwise, an error message indicates an invalid or expired session.

Test Case Derivation: This test ensures the system retrieves and displays reports correctly and handles invalid Session ID scenarios (Security).

How test will be performed: Automated scripts will input valid and invalid datasets, verifying the correct report display.

\item{FR-FE-9\\} (FR-1, FR-12, FR-15)

Type: Automated
					
Initial State: Report Page (Saved)
					
Input: Delete or Submit
					
Output: If 'Delete' is selected, the report page is removed with confirmation. If 'Submit' is selected, it transitions to a submitted state with confirmation.

Test Case Derivation: This test verifies that the system correctly processes the actions of deleting or submitting a report, ensuring appropriate status changes and user feedback.

How test will be performed:Automated scripts will simulate the selection of Delete and Submit actions, checking for the expected outcomes—either the removal of the report or the confirmation of submission.

\item{FR-FE-10\\} (FR-7 FR-10 FR-13)

Type: Automated
					
Initial State: Profile Page (Patient)
					
Input: N/A
					
Output: Displays all relevant information, such as the treatment plan and reports.

Test Case Derivation: This test verifies that the Profile Page correctly displays all pertinent patient information, ensuring data accuracy and completeness.

How test will be performed: Automated scripts will access the Profile Page and check that all expected information, including the treatment plan and reports, is displayed correctly.

\item{FR-FE-11\\} (FR-4 FR-7 FR-8 FR-10 FR-13)

Type: Automated
					
Initial State: Profile Page (Doctor)
					
Input: Actions to edit a patient's profile, including treatment plan, comments, etc.
					
Output: Displays all patients and their respective information.

Test Case Derivation: This test ensures that the Profile Page for the doctor accurately displays all patients and their details after any editing actions are performed.

How test will be performed: Automated scripts will simulate actions to edit patient profiles and verify that the updated information is displayed correctly for all patients.

\end{enumerate}

\subsubsection{Back End}

The backend testing will focus on validating the accuracy, reliability, and performance of all server-side functions and data processes. This includes testing data-fetching endpoints, ensuring secure and correct handling of requests and responses, and verifying data integrity in various scenarios. Automated tests will simulate multiple use cases to ensure that business logic, such as data processing, authentication, and error handling, works as expected. Performance tests will also be conducted to confirm that the backend maintains speed and efficiency under load, supporting seamless interactions between the frontend and backend components. This comprehensive backend testing aims to prevent data inconsistencies and ensure robust, secure, and scalable operations.

\begin{enumerate}

\item{FR-BE-1\\} (FR-16)

Type: Automated 
					
Initial State: Start
					
Input: N/A
					
Output: Initialize backed systems and returns result.

Test Case Derivation: This test verifies that the backend systems initialize correctly and that the expected result  is returned, indicating successful setup. (Go to S0)

How test will be performed: Automated scripts will execute the initialization process and validate that the system returns the expected results without errors.

\item{FR-BE-2\\} (FR-1 FR-2 FR-12)

Type: Automated 
					
Initial State: S0
					
Input: Unloaded Form data (Images, symptoms etc)
					
Output: Transition to S1 if the upload is successful; otherwise, remain at S0.

Test Case Derivation: This test verifies that the system successfully transitions from S0 to S1 upon a successful upload of form data and stays at S0 if the upload fails.

How test will be performed: Automated scripts will simulate the upload of form data and check for the correct transition to S1 when the upload is successful or confirm that the system remains at S0 when the upload fails.

\item{FR-BE-3\\} (FR-7 FR-15)

Type: Automated 
					
Initial State: S1
					
Input: Uploaded Form data
					
Output: If the form data is valid, transition to S2; otherwise, return to S0.

Test Case Derivation: This test verifies that the upload form processes data correctly, ensuring successful transitions between states based on the upload outcome.

How test will be performed: Automated scripts will simulate form submissions with valid and invalid data, checking that the system transitions and returns the appropriate error code for failures.

\item{FR-BE-4\\} (FR-14)

Type: Automated 
					
Initial State: S2
					
Input: Valid Form data
					
Output: Processed and normalized form data (in the case of image data) if success transition into S3, else failure state.

Test Case Derivation: This test verifies that the system accurately processes and normalizes valid form data, ensuring that the output meets the specified requirements for further use.

How test will be performed: Automated scripts will input valid form data and verify that the system produces correctly processed and normalized output, checking for data integrity and conformity to expected formats.

\item{FR-BE-5\\} (FR-3)

Type: Automated
					
Initial State: S3
					
Input: Normalized data
					
Output: Diagnosis prediction data if success transition into S4, else failure state.

Test Case Derivation: If the normalized data is successfully processed by the model, transition to S4; otherwise, enter a failure state.

How test will be performed: Automated scripts will input normalized data into the model and verify that the diagnosis prediction data is produced correctly. The scripts will check for a successful transition to S4 or confirm entry into a failure state based on the processing outcome.

\item{FR-BE-6\\} (FR-4 FR-5 FR-6 FR-8)

Type: Automated 
					
Initial State: S4
					
Input: Diagnosis prediction data
					
Output: Generated report data if sucess transition into S5, else failure state.

Test Case Derivation: This test will generate a concise report based on the diagnosis prediction data provided.

How test will be performed: Automated scripts will input the diagnosis prediction data and verify that a report is generated successfully. The scripts will check for a successful transition to S5 or confirm entry into a failure state if the report generation fails.

\item{FR-BE-7\\} (FR3)

Type: Automated 
					
Initial State: Failure
					
Input: Error code
					
Output: Appropriate error message and system logs.

Test Case Derivation: This is the exit state of the process, indicating that an error has occurred, and the system should provide relevant feedback based on the error code received.

How test will be performed: Automated scripts will input various error codes and verify that the system generates the correct error messages and logs the details appropriately. The test will ensure that the system behaves as expected in the failure state.

\item{FR-BE-8\\} (FR-11 FR-13)

Type: Automated 
					
Initial State: S5
					
Input: Generated report data
					
Output: A confidence score; if low, transition to S6 (if the user is under a doctor); otherwise, proceed to End (if the doctor has approved or the user is not under any doctor).

Test Case Derivation: The confidence score is derived from the generated report data, determining the subsequent state based on its value and the user's doctor status.

How test will be performed: Automated scripts will process the generated report data, calculate the confidence score, and verify the correct state transition based on the score and user status.

\item{FR-BE-9\\} (FR-9 FR-13)

Type: Automated
					
Initial State: S6
					
Input: Edited data (Doctor)
					
Output: JSON data containing relevant details of reports, diagnosis data, and status.

Test Case Derivation: The output will depend on the doctor's assessment of the confidence score and the generated report data, resulting in manually annotated information.

How test will be performed: Automated scripts will submit the confidence score and generated report data for manual review, verifying that the doctor’s annotations are accurately recorded and linked to the corresponding report.

\item{FR-BE-10\\} (FR-3 FR7)

Type: Automated 
					
Initial State: End
					
Input: Approval/Disapproval (1/0)
					
Output: JSON data containing relevant details of reports, diagnosis data, and status.

Test Case Derivation: The output will vary based on the input (1 or 0), returns nothing or include all pertinent report and diagnosis information.

How test will be performed: Automated scripts will input approval/disapproval and validate that the returned JSON data matches the expected format and includes all relevant details.

\end{enumerate}
\subsection{Tests for Non-Functional Requirements}

\subsection{Traceability Between Test Cases and Modules}
This section provides evidence that all modules have been considered in the test case design. Each test case is mapped to its corresponding module to ensure comprehensive coverage and verification of functionality.

\begin{landscape}
\begin{table}[h]
    \centering
    \caption{Traceability Matrix}
    \begin{tabular}{|c|c|c|c|}
        \hline
        \textbf{Test Case ID} & \textbf{Module} & \textbf{Description} & \textbf{Status} \\
        \hline
        test-id1 & Module A (e.g., Upload) & Tests the upload functionality & Pending \\
        \hline
        test-id2 & Module B (e.g., Profile) & Verifies profile editing capabilities & In Progress \\
        \hline
        test-id3 & Module C (e.g., Reports) & Confirms report generation accuracy & Completed \\
        \hline
        test-id4 & Module D (e.g., Diagnosis) & Assesses diagnosis prediction output & Pending \\
        \hline
    \end{tabular}
\end{table}
\end{landscape}


\textbf{Analysis:}
\begin{itemize}
    \item Each test case directly corresponds to specific modules within the application, ensuring that all critical functionalities are covered.
    \item The status of each test case reflects its current progress, allowing for easy tracking of testing efforts.
    \item System logs will be generated during the execution of each test case to provide a detailed record of operations. These logs will capture input parameters, output results, and any errors encountered, aiding in debugging and validation.
    \item By correlating the test case outcomes with system logs, we can ensure accountability and traceability in the testing process, making it easier to identify and address issues.
\end{itemize}
				
\bibliographystyle{plainnat}

\bibliography{../../refs/References}

\newpage

\section{Appendix}

This is where you can place additional information.

\subsection{Symbolic Parameters}

The definition of the test cases will call for SYMBOLIC\_CONSTANTS.
Their values are defined in this section for easy maintenance.

\subsection{Usability Survey Questions?}

\wss{This is a section that would be appropriate for some projects.}

\newpage{}
\section*{Appendix --- Reflection}

\wss{This section is not required for CAS 741}

The information in this section will be used to evaluate the team members on the
graduate attribute of Lifelong Learning.


\begin{enumerate}
  \item What went well while writing this deliverable? 
  \item What pain points did you experience during this deliverable, and how
    did you resolve them?
  \item What knowledge and skills will the team collectively need to acquire to
  successfully complete the verification and validation of your project?
  Examples of possible knowledge and skills include dynamic testing knowledge,
  static testing knowledge, specific tool usage, Valgrind etc.  You should look to
  identify at least one item for each team member.
  \item For each of the knowledge areas and skills identified in the previous
  question, what are at least two approaches to acquiring the knowledge or
  mastering the skill?  Of the identified approaches, which will each team
  member pursue, and why did they make this choice?
\end{enumerate}

\end{document}