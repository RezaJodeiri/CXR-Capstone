\documentclass[12pt, titlepage]{article}
\usepackage{lscape}
\usepackage{booktabs}
\usepackage{tabularx}
\usepackage{hyperref}
\hypersetup{
    colorlinks,
    citecolor=blue,
    filecolor=black,
    linkcolor=red,
    urlcolor=blue
}
\usepackage[round]{natbib}

%% Comments

\usepackage{color}

\newif\ifcomments\commentstrue %displays comments
%\newif\ifcomments\commentsfalse %so that comments do not display

\ifcomments
\newcommand{\authornote}[3]{\textcolor{#1}{[#3 ---#2]}}
\newcommand{\todo}[1]{\textcolor{red}{[TODO: #1]}}
\else
\newcommand{\authornote}[3]{}
\newcommand{\todo}[1]{}
\fi

\newcommand{\wss}[1]{\authornote{blue}{SS}{#1}} 
\newcommand{\plt}[1]{\authornote{magenta}{TPLT}{#1}} %For explanation of the template
\newcommand{\an}[1]{\authornote{cyan}{Author}{#1}}

%% Common Parts

\newcommand{\progname}{ProgName} % PUT YOUR PROGRAM NAME HERE
\newcommand{\authname}{Team \#, Team Name
\\ Student 1 name
\\ Student 2 name
\\ Student 3 name
\\ Student 4 name} % AUTHOR NAMES                  

\usepackage{hyperref}
    \hypersetup{colorlinks=true, linkcolor=blue, citecolor=blue, filecolor=blue,
                urlcolor=blue, unicode=false}
    \urlstyle{same}
                                


\begin{document}

\title{System Verification and Validation Plan for \progname{}} 
\author{\authname}
\date{\today}
	
\maketitle

\pagenumbering{roman}

\section*{Revision History}

\begin{tabularx}{\textwidth}{p{3cm}p{2cm}X}
\toprule {\bf Date} & {\bf Version} & {\bf Notes}\\
\midrule
Date 1 & 1.0 & Notes\\
Date 2 & 1.1 & Notes\\
\bottomrule
\end{tabularx}

~\\
\wss{The intention of the VnV plan is to increase confidence in the software.
However, this does not mean listing every verification and validation technique
that has ever been devised.  The VnV plan should also be a \textbf{feasible}
plan. Execution of the plan should be possible with the time and team available.
If the full plan cannot be completed during the time available, it can either be
modified to ``fake it'', or a better solution is to add a section describing
what work has been completed and what work is still planned for the future.}

\wss{The VnV plan is typically started after the requirements stage, but before
the design stage.  This means that the sections related to unit testing cannot
initially be completed.  The sections will be filled in after the design stage
is complete.  the final version of the VnV plan should have all sections filled
in.}

\newpage

\tableofcontents

\listoftables
\wss{Remove this section if it isn't needed}

\listoffigures
\wss{Remove this section if it isn't needed}

\newpage

\section{Symbols, Abbreviations, and Acronyms}

\renewcommand{\arraystretch}{1.2}
\begin{tabular}{l l} 
  \toprule		
  \textbf{symbol} & \textbf{description}\\
  \midrule 
  T & Test\\
  \bottomrule
\end{tabular}\\

\wss{symbols, abbreviations, or acronyms --- you can simply reference the SRS
  \citep{SRS} tables, if appropriate}

\wss{Remove this section if it isn't needed}

\newpage

\pagenumbering{arabic}

This document ... \wss{provide an introductory blurb and roadmap of the
  Verification and Validation plan}

\section{General Information}

\subsection{Summary}

\wss{Say what software is being tested.  Give its name and a brief overview of
  its general functions.}

\subsection{Objectives}

\wss{State what is intended to be accomplished.  The objective will be around
  the qualities that are most important for your project.  You might have
  something like: ``build confidence in the software correctness,''
  ``demonstrate adequate usability.'' etc.  You won't list all of the qualities,
  just those that are most important.}

\wss{You should also list the objectives that are out of scope.  You don't have 
the resources to do everything, so what will you be leaving out.  For instance, 
if you are not going to verify the quality of usability, state this.  It is also 
worthwhile to justify why the objectives are left out.}

\wss{The objectives are important because they highlight that you are aware of 
limitations in your resources for verification and validation.  You can't do everything, 
so what are you going to prioritize?  As an example, if your system depends on an 
external library, you can explicitly state that you will assume that external library 
has already been verified by its implementation team.}

\subsection{Challenge Level and Extras}

\wss{State the challenge level (advanced, general, basic) for your project.
Your challenge level should exactly match what is included in your problem
statement.  This should be the challenge level agreed on between you and the
course instructor.  You can use a pull request to update your challenge level
(in TeamComposition.csv or Repos.csv) if your plan changes as a result of the
VnV planning exercise.}

\wss{Summarize the extras (if any) that were tackled by this project.  Extras
can include usability testing, code walkthroughs, user documentation, formal
proof, GenderMag personas, Design Thinking, etc.  Extras should have already
been approved by the course instructor as included in your problem statement.
You can use a pull request to update your extras (in TeamComposition.csv or
Repos.csv) if your plan changes as a result of the VnV planning exercise.}

\subsection{Relevant Documentation}

\wss{Reference relevant documentation.  This will definitely include your SRS
  and your other project documents (design documents, like MG, MIS, etc).  You
  can include these even before they are written, since by the time the project
  is done, they will be written.  You can create BibTeX entries for your
  documents and within those entries include a hyperlink to the documents.}

\citet{SRS}

\wss{Don't just list the other documents.  You should explain why they are relevant and 
how they relate to your VnV efforts.}

\section{Plan}

\wss{Introduce this section.  You can provide a roadmap of the sections to
  come.}

\subsection{Verification and Validation Team}

\wss{Your teammates.  Maybe your supervisor.
  You should do more than list names.  You should say what each person's role is
  for the project's verification.  A table is a good way to summarize this information.}

\subsection{SRS Verification Plan}

\wss{List any approaches you intend to use for SRS verification.  This may
  include ad hoc feedback from reviewers, like your classmates (like your
  primary reviewer), or you may plan for something more rigorous/systematic.}

\wss{If you have a supervisor for the project, you shouldn't just say they will
read over the SRS.  You should explain your structured approach to the review.
Will you have a meeting?  What will you present?  What questions will you ask?
Will you give them instructions for a task-based inspection?  Will you use your
issue tracker?}

\wss{Maybe create an SRS checklist?}

\subsection{Design Verification Plan}

\wss{Plans for design verification}

\wss{The review will include reviews by your classmates}

\wss{Create a checklists?}

\subsection{Verification and Validation Plan Verification Plan}

\wss{The verification and validation plan is an artifact that should also be
verified.  Techniques for this include review and mutation testing.}

\wss{The review will include reviews by your classmates}

\wss{Create a checklists?}

\subsection{Implementation Verification Plan}

\wss{You should at least point to the tests listed in this document and the unit
  testing plan.}

\wss{In this section you would also give any details of any plans for static
  verification of the implementation.  Potential techniques include code
  walkthroughs, code inspection, static analyzers, etc.}

\wss{The final class presentation in CAS 741 could be used as a code
walkthrough.  There is also a possibility of using the final presentation (in
CAS741) for a partial usability survey.}

\subsection{Automated Testing and Verification Tools}

\wss{What tools are you using for automated testing.  Likely a unit testing
  framework and maybe a profiling tool, like ValGrind.  Other possible tools
  include a static analyzer, make, continuous integration tools, test coverage
  tools, etc.  Explain your plans for summarizing code coverage metrics.
  Linters are another important class of tools.  For the programming language
  you select, you should look at the available linters.  There may also be tools
  that verify that coding standards have been respected, like flake9 for
  Python.}

\wss{If you have already done this in the development plan, you can point to
that document.}

\wss{The details of this section will likely evolve as you get closer to the
  implementation.}

\subsection{Software Validation Plan}

\wss{If there is any external data that can be used for validation, you should
  point to it here.  If there are no plans for validation, you should state that
  here.}

\wss{You might want to use review sessions with the stakeholder to check that
the requirements document captures the right requirements.  Maybe task based
inspection?}

\wss{For those capstone teams with an external supervisor, the Rev 0 demo should 
be used as an opportunity to validate the requirements.  You should plan on 
demonstrating your project to your supervisor shortly after the scheduled Rev 0 demo.  
The feedback from your supervisor will be very useful for improving your project.}

\wss{For teams without an external supervisor, user testing can serve the same purpose 
as a Rev 0 demo for the supervisor.}

\wss{This section might reference back to the SRS verification section.}

\section{Unit Test Description}

\wss{This section should not be filled in until after the MIS (detailed design
  document) has been completed.}

\wss{Reference your MIS (detailed design document) and explain your overall
philosophy for test case selection.}  
\\ Document is not completed yet.

\wss{To save space and time, it may be an option to provide less detail in this section.  
For the unit tests you can potentially layout your testing strategy here.  That is, you 
can explain how tests will be selected for each module.  For instance, your test building 
approach could be test cases for each access program, including one test for normal behaviour 
and as many tests as needed for edge cases.  Rather than create the details of the input 
and output here, you could point to the unit testing code.  For this to work, you code 
needs to be well-documented, with meaningful names for all of the tests.}

\subsection{Unit Testing Scope}

\wss{What modules are outside of the scope.  If there are modules that are
  developed by someone else, then you would say here if you aren't planning on
  verifying them.  There may also be modules that are part of your software, but
  have a lower priority for verification than others.  If this is the case,
  explain your rationale for the ranking of module importance.}
\subsection{Unit Testing Scope}
The modules which the unit test focuses on will be divided into two categories: \textbf{Front-end} and \textbf{Back-end}.\\\\
The following modules are considered outside the scope of unit testing:\\\\
\textbf{Third-Party Modules}: 
Any third-party libraries or frameworks integrated into the application will not be subjected to unit testing. This includes modules developed by external sources, such as libraries for data visualization or authentication. These modules will be relied upon for their documented functionality, but no direct verification will be performed.


\subsection{Rationale for Ranking}
The prioritization of modules for testing is based on the following criteria:
\begin{itemize}
    \item \textbf{Impact on Core Functionality}: Modules that are integral to the application’s main features will receive higher priority for unit testing.
    \item \textbf{User Experience}: Features that directly affect user interactions and overall experience will be tested more rigorously.
    \item \textbf{Risk Assessment}: Modules with higher associated risks, such as security and data handling, will be prioritized to mitigate potential issues.
\end{itemize}

By focusing our testing efforts on high-impact modules, we aim to ensure a robust and reliable application while efficiently utilizing our resources.


\subsection{Tests for Functional Requirements}

\wss{Most of the verification will be through automated unit testing.  If
  appropriate specific modules can be verified by a non-testing based
  technique.  That can also be documented in this section.}\\\\
For unit testing, we’re dividing the process into frontend and backend sections. UI/UX interactions, such as buttons and links, will be tested manually to ensure they function as expected. Backend functions, especially those involving data fetching, will be tested automatically.\\\\
Frontend testing: UI/UX elements such as buttons, links, and other interactive components will undergo manual testing to confirm that they work as intended from a user perspective. This includes verifying that each interactive element responds accurately and meets usability standards.\\\\
Backend testing: functions that handle data fetching and processing will be validated through automated tests. These tests will simulate various scenarios to ensure that data retrieval, processing, and any associated logic work as expected, helping us catch issues early and maintain data integrity. This combination of manual and automated testing ensures comprehensive coverage and reliability across the application. (Possibly by Pytest library)

\subsubsection{Front End}

\wss{Include a blurb here to explain why the subsections below cover the module.
  References to the MIS would be good.  You will want tests from a black box
  perspective and from a white box perspective.  Explain to the reader how the
  tests were selected.}\\\\
  The frontend testing will encompass all possible user interactions, covering any element that the user can click or interact with. This includes, but is not limited to, buttons, links, dropdowns, modals, and form fields. Each interactive element will be manually tested to confirm that it behaves as expected in various scenarios, such as different screen sizes and user flows. This ensures that the user experience is smooth, intuitive, and functional across all devices and use cases. Anything that deals with username/password/upload/download should be securely designed.
\begin{enumerate}

\item{FR-FE-1\\} (FR-7, FR-13)

Type: Manual
			
Initial State:  Automated
					
Input: Username \& Password
					
Output: User is successfully logged in and redirected to the home page if the username and password are correct. If incorrect, an error message is displayed, preventing login.

Test Case Derivation: The output is expected to confirm whether the login process works correctly. A successful login with correct credentials should redirect the user to the home page, while incorrect credentials should display an error message, preventing access. This expected outcome ensures that the system only grants access to users with valid login information, maintaining security and usability standards.

How test will be performed: An automated

[Refrence to Login Form UI]

\item{FR-FE-2\\} (FR-7, FR-13)

Type: Automated
					
Initial State: Register Page
					
Input: All form fields: Include but not limited to username/email, first name, last name, occupation, organization, location, password and confirmed password.
					
Output: User is successfully registered and redirected to the login page if all fields are valid; otherwise, appropriate error messages are displayed for invalid inputs.

Test Case Derivation: This test verifies that the registration functionality correctly processes all required fields. The expected output is based on whether the provided data meets validation criteria and does not contain duplicates.

How test will be performed: Automated scripts will fill in all form fields with valid and invalid data, submitting the registration form. The outputs will be checked to ensure successful registration for valid data and the correct error messages for invalid inputs.
[Refrence to Register Form UI]

\item{FR-FE-3\\} (FR-1, FR-12, FR-15)

Type: Automated
					
Initial State: Upload Page 
					
Input: One or more pictures in various formats (e.g., JPG, PNG, etc.) and within constrained size.
					
Output: Images are successfully uploaded, and confirmation messages are displayed. For unsupported formats, appropriate error messages are shown.

Test Case Derivation: This test verifies the image upload functionality, ensuring that the system accepts valid image formats while rejecting unsupported ones. The expected outcome is determined by the format and size of the uploaded images.

How test will be performed: Automated scripts will upload a mix of supported and unsupported image formats and sizes to the upload page. The results will be checked for successful uploads or correct error messages for unsupported formats.
[Refrence to Upload Page UI]

\item{FR-FE-4\\} (FR-3)

Type: Automated
					
Initial State: Upload Page (After Upload)
					
Input: Symptoms
					
Output: Symptoms are successfully recorded and displayed on the user interface, confirming submission; an error message appears for invalid input.

Test Case Derivation: This test verifies that the system correctly processes and records user-submitted symptoms after an image upload.

How test will be performed: Automated scripts will input various symptom descriptions, including valid and invalid examples, to the appropriate field. The results will be checked to ensure successful recording or the display of correct error messages for invalid entries.

\item{FR-FE-5\\} (FR-1, FR-12, FR-15)

Type: Automated
					
Initial State: Upload Page (Any)
					
Input: Delete, Save or Submit
					
Output: User’s symptoms and uploaded images are deleted, User’s symptoms and uploaded images are saved under profile; User's symptoms and uploaded images are submitted to backed. Confirmation message will be displayed or if there are validation errors, appropriate error messages are shown.

Test Case Derivation: The expected output depends on the action taken, ensuring that data is either saved locally or submitted to the backend as appropriate.

How test will be performed: Automated scripts will execute the Save or Submit actions and subsequently assess the actual outcomes against the expected results to ensure compliance with the specified requirements.

\item{FR-FE-6\\} (FR-13)

Type: Manual
					
Initial State: Home Page 
					
Input: Actions to navigate to other pages. (Upload, Profile, Logout-redirects to Login page etc)
					
Output: The user is successfully redirected to the selected page.

Test Case Derivation: This test ensures that navigation actions from the Home Page function correctly, leading users to the appropriate pages without errors.

How test will be performed: All the buttons will be manually tested, and observed for functionality.

\item{FR-FE-7\\} (FR-1, FR-12, FR-15)

Type: Automated
					
Initial State: Report Page (Saved)
					
Input: Delete or Submit
					
Output: Report page will be deleted or Report page will go to submitted state.

Test Case Derivation: This test verifies that the system correctly processes the actions of deleting or submitting a report, ensuring appropriate status changes and user feedback.

How test will be performed: Automated scripts will simulate the selection of "Delete" and "Submit" actions, checking for the expected outcomes—either the removal of the report or the confirmation of submission.

\item{FR-FE-8\\} (FR-5 FR-6 FR-7 FR-11 FR-13)

Type: Automated
					
Initial State: Report Page (Submitted)
					
Input: N/A
					
Output: Fetch and display the report from the database; otherwise, an error message indicates an invalid or expired session.

Test Case Derivation: This test ensures the system retrieves and displays reports correctly and handles invalid Session ID scenarios (Security).

How test will be performed: Automated scripts will input valid and invalid datasets, verifying the correct report display.

\item{FR-FE-9\\} (FR-1, FR-12, FR-15)

Type: Automated
					
Initial State: Report Page (Saved)
					
Input: Delete or Submit
					
Output: If 'Delete' is selected, the report page is removed with confirmation. If 'Submit' is selected, it transitions to a submitted state with confirmation.

Test Case Derivation: This test verifies that the system correctly processes the actions of deleting or submitting a report, ensuring appropriate status changes and user feedback.

How test will be performed:Automated scripts will simulate the selection of Delete and Submit actions, checking for the expected outcomes—either the removal of the report or the confirmation of submission.

\item{FR-FE-10\\} (FR-7 FR-10 FR-13)

Type: Automated
					
Initial State: Profile Page (Patient)
					
Input: N/A
					
Output: Displays all relevant information, such as the treatment plan and reports.

Test Case Derivation: This test verifies that the Profile Page correctly displays all pertinent patient information, ensuring data accuracy and completeness.

How test will be performed: Automated scripts will access the Profile Page and check that all expected information, including the treatment plan and reports, is displayed correctly.

\item{FR-FE-11\\} (FR-4 FR-7 FR-8 FR-10 FR-13)

Type: Automated
					
Initial State: Profile Page (Doctor)
					
Input: Actions to edit a patient's profile, including treatment plan, comments, etc.
					
Output: Displays all patients and their respective information.

Test Case Derivation: This test ensures that the Profile Page for the doctor accurately displays all patients and their details after any editing actions are performed.

How test will be performed: Automated scripts will simulate actions to edit patient profiles and verify that the updated information is displayed correctly for all patients.

\end{enumerate}

\subsubsection{Back End}

The backend testing will focus on validating the accuracy, reliability, and performance of all server-side functions and data processes. This includes testing data-fetching endpoints, ensuring secure and correct handling of requests and responses, and verifying data integrity in various scenarios. Automated tests will simulate multiple use cases to ensure that business logic, such as data processing, authentication, and error handling, works as expected. Performance tests will also be conducted to confirm that the backend maintains speed and efficiency under load, supporting seamless interactions between the frontend and backend components. This comprehensive backend testing aims to prevent data inconsistencies and ensure robust, secure, and scalable operations.

\begin{enumerate}

\item{FR-BE-1\\} (FR-16)

Type: Automated 
					
Initial State: Start
					
Input: N/A
					
Output: Initialize backed systems and returns result.

Test Case Derivation: This test verifies that the backend systems initialize correctly and that the expected result  is returned, indicating successful setup. (Go to S0)

How test will be performed: Automated scripts will execute the initialization process and validate that the system returns the expected results without errors.

\item{FR-BE-2\\} (FR-1 FR-2 FR-12)

Type: Automated 
					
Initial State: S0
					
Input: Unloaded Form data (Images, symptoms etc)
					
Output: Transition to S1 if the upload is successful; otherwise, remain at S0.

Test Case Derivation: This test verifies that the system successfully transitions from S0 to S1 upon a successful upload of form data and stays at S0 if the upload fails.

How test will be performed: Automated scripts will simulate the upload of form data and check for the correct transition to S1 when the upload is successful or confirm that the system remains at S0 when the upload fails.

\item{FR-BE-3\\} (FR-7 FR-15)

Type: Automated 
					
Initial State: S1
					
Input: Uploaded Form data
					
Output: If the form data is valid, transition to S2; otherwise, return to S0.

Test Case Derivation: This test verifies that the upload form processes data correctly, ensuring successful transitions between states based on the upload outcome.

How test will be performed: Automated scripts will simulate form submissions with valid and invalid data, checking that the system transitions and returns the appropriate error code for failures.

\item{FR-BE-4\\} (FR-14)

Type: Automated 
					
Initial State: S2
					
Input: Valid Form data
					
Output: Processed and normalized form data (in the case of image data) if success transition into S3, else failure state.

Test Case Derivation: This test verifies that the system accurately processes and normalizes valid form data, ensuring that the output meets the specified requirements for further use.

How test will be performed: Automated scripts will input valid form data and verify that the system produces correctly processed and normalized output, checking for data integrity and conformity to expected formats.

\item{FR-BE-5\\} (FR-3)

Type: Automated
					
Initial State: S3
					
Input: Normalized data
					
Output: Diagnosis prediction data if success transition into S4, else failure state.

Test Case Derivation: If the normalized data is successfully processed by the model, transition to S4; otherwise, enter a failure state.

How test will be performed: Automated scripts will input normalized data into the model and verify that the diagnosis prediction data is produced correctly. The scripts will check for a successful transition to S4 or confirm entry into a failure state based on the processing outcome.

\item{FR-BE-6\\} (FR-4 FR-5 FR-6 FR-8)

Type: Automated 
					
Initial State: S4
					
Input: Diagnosis prediction data
					
Output: Generated report data if sucess transition into S5, else failure state.

Test Case Derivation: This test will generate a concise report based on the diagnosis prediction data provided.

How test will be performed: Automated scripts will input the diagnosis prediction data and verify that a report is generated successfully. The scripts will check for a successful transition to S5 or confirm entry into a failure state if the report generation fails.

\item{FR-BE-7\\} (FR3)

Type: Automated 
					
Initial State: Failure
					
Input: Error code
					
Output: Appropriate error message and system logs.

Test Case Derivation: This is the exit state of the process, indicating that an error has occurred, and the system should provide relevant feedback based on the error code received.

How test will be performed: Automated scripts will input various error codes and verify that the system generates the correct error messages and logs the details appropriately. The test will ensure that the system behaves as expected in the failure state.

\item{FR-BE-8\\} (FR-11 FR-13)

Type: Automated 
					
Initial State: S5
					
Input: Generated report data
					
Output: A confidence score; if low, transition to S6 (if the user is under a doctor); otherwise, proceed to End (if the doctor has approved or the user is not under any doctor).

Test Case Derivation: The confidence score is derived from the generated report data, determining the subsequent state based on its value and the user's doctor status.

How test will be performed: Automated scripts will process the generated report data, calculate the confidence score, and verify the correct state transition based on the score and user status.

\item{FR-BE-9\\} (FR-9 FR-13)

Type: Automated
					
Initial State: S6
					
Input: Edited data (Doctor)
					
Output: JSON data containing relevant details of reports, diagnosis data, and status.

Test Case Derivation: The output will depend on the doctor's assessment of the confidence score and the generated report data, resulting in manually annotated information.

How test will be performed: Automated scripts will submit the confidence score and generated report data for manual review, verifying that the doctor’s annotations are accurately recorded and linked to the corresponding report.

\item{FR-BE-10\\} (FR-3 FR7)

Type: Automated 
					
Initial State: End
					
Input: Approval/Disapproval (1/0)
					
Output: JSON data containing relevant details of reports, diagnosis data, and status.

Test Case Derivation: The output will vary based on the input (1 or 0), returns nothing or include all pertinent report and diagnosis information.

How test will be performed: Automated scripts will input approval/disapproval and validate that the returned JSON data matches the expected format and includes all relevant details.

\end{enumerate}
\subsection{Tests for Non-Functional Requirements}

\subsection{Traceability Between Test Cases and Modules}
This section provides evidence that all modules have been considered in the test case design. Each test case is mapped to its corresponding module to ensure comprehensive coverage and verification of functionality.

\begin{landscape}
\begin{table}[h]
    \centering
    \caption{Traceability Matrix}
    \begin{tabular}{|c|c|c|c|}
        \hline
        \textbf{Test Case ID} & \textbf{Module} & \textbf{Description} & \textbf{Status} \\
        \hline
        test-id1 & Module A (e.g., Upload) & Tests the upload functionality & Pending \\
        \hline
        test-id2 & Module B (e.g., Profile) & Verifies profile editing capabilities & In Progress \\
        \hline
        test-id3 & Module C (e.g., Reports) & Confirms report generation accuracy & Completed \\
        \hline
        test-id4 & Module D (e.g., Diagnosis) & Assesses diagnosis prediction output & Pending \\
        \hline
    \end{tabular}
\end{table}
\end{landscape}


\textbf{Analysis:}
\begin{itemize}
    \item Each test case directly corresponds to specific modules within the application, ensuring that all critical functionalities are covered.
    \item The status of each test case reflects its current progress, allowing for easy tracking of testing efforts.
    \item System logs will be generated during the execution of each test case to provide a detailed record of operations. These logs will capture input parameters, output results, and any errors encountered, aiding in debugging and validation.
    \item By correlating the test case outcomes with system logs, we can ensure accountability and traceability in the testing process, making it easier to identify and address issues.
\end{itemize}
				
\bibliographystyle{plainnat}

\bibliography{../../refs/References}

\newpage

\section{Appendix}

This is where you can place additional information.

\subsection{Symbolic Parameters}

The definition of the test cases will call for SYMBOLIC\_CONSTANTS.
Their values are defined in this section for easy maintenance.

\subsection{Usability Survey Questions?}

\wss{This is a section that would be appropriate for some projects.}

\newpage{}
\section*{Appendix --- Reflection}

\wss{This section is not required for CAS 741}

The information in this section will be used to evaluate the team members on the
graduate attribute of Lifelong Learning.

The purpose of reflection questions is to give you a chance to assess your own
learning and that of your group as a whole, and to find ways to improve in the
future. Reflection is an important part of the learning process.  Reflection is
also an essential component of a successful software development process.  

Reflections are most interesting and useful when they're honest, even if the
stories they tell are imperfect. You will be marked based on your depth of
thought and analysis, and not based on the content of the reflections
themselves. Thus, for full marks we encourage you to answer openly and honestly
and to avoid simply writing ``what you think the evaluator wants to hear.''

Please answer the following questions.  Some questions can be answered on the
team level, but where appropriate, each team member should write their own
response:


\begin{enumerate}
  \item What went well while writing this deliverable? 
  \item What pain points did you experience during this deliverable, and how
    did you resolve them?
  \item What knowledge and skills will the team collectively need to acquire to
  successfully complete the verification and validation of your project?
  Examples of possible knowledge and skills include dynamic testing knowledge,
  static testing knowledge, specific tool usage, Valgrind etc.  You should look to
  identify at least one item for each team member.
  \item For each of the knowledge areas and skills identified in the previous
  question, what are at least two approaches to acquiring the knowledge or
  mastering the skill?  Of the identified approaches, which will each team
  member pursue, and why did they make this choice?
\end{enumerate}

\end{document}