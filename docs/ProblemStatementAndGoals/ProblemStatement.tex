\documentclass{article}

\usepackage{tabularx}
\usepackage{booktabs}
\usepackage{float}

\title{Problem Statement and Goals\\\progname}

\author{\authname}

\date{}

%% Comments

\usepackage{color}

\newif\ifcomments\commentstrue %displays comments
%\newif\ifcomments\commentsfalse %so that comments do not display

\ifcomments
\newcommand{\authornote}[3]{\textcolor{#1}{[#3 ---#2]}}
\newcommand{\todo}[1]{\textcolor{red}{[TODO: #1]}}
\else
\newcommand{\authornote}[3]{}
\newcommand{\todo}[1]{}
\fi

\newcommand{\wss}[1]{\authornote{blue}{SS}{#1}} 
\newcommand{\plt}[1]{\authornote{magenta}{TPLT}{#1}} %For explanation of the template
\newcommand{\an}[1]{\authornote{cyan}{Author}{#1}}

%% Common Parts

\newcommand{\progname}{ProgName} % PUT YOUR PROGRAM NAME HERE
\newcommand{\authname}{Team \#, Team Name
\\ Student 1 name
\\ Student 2 name
\\ Student 3 name
\\ Student 4 name} % AUTHOR NAMES                  

\usepackage{hyperref}
    \hypersetup{colorlinks=true, linkcolor=blue, citecolor=blue, filecolor=blue,
                urlcolor=blue, unicode=false}
    \urlstyle{same}
                                

The purpose of reflection questions is to give you a chance to assess your own
learning and that of your group as a whole, and to find ways to improve in the
future. Reflection is an important part of the learning process.  Reflection is
also an essential component of a successful software development process.  

Reflections are most interesting and useful when they're honest, even if the
stories they tell are imperfect. You will be marked based on your depth of
thought and analysis, and not based on the content of the reflections
themselves. Thus, for full marks we encourage you to answer openly and honestly
and to avoid simply writing ``what you think the evaluator wants to hear.''

Please answer the following questions.  Some questions can be answered on the
team level, but where appropriate, each team member should write their own
response:


\begin{document}

\maketitle

\begin{table}[hp]
\caption{Revision History} \label{TblRevisionHistory}
\begin{tabularx}{\textwidth}{llX}
\toprule
\textbf{Date} & \textbf{Developer(s)} & \textbf{Change}\\
\midrule
Sep 21, 2024 & Kelly & Add section Stakeholders, Environment\\
Sep 21, 2024 & Reza & Sec1 Problem statement and I/O\\
Sep 22, 2024 & Ayman & Fixed Readme, Fixed License, Added Challenge Level for PS\\
Sep 22, 2024 & Kelly & Change team number\\
Sep 23, 2024 & Reza & Added goals and stretch goals\\
Sep 23, 2024 & Nathan & Added stakeholder, and spellcheck\\
\bottomrule
\end{tabularx}
\end{table}
\newpage
\section{Problem Statement}
\subsection{Problem}
The rapid and accurate detection of lung and cardiac conditions using chest X-rays is critical in clinical settings. However, current methods for disease detection and monitoring rely heavily on manual interpretation and patient monitoring by radiologists, leading to inconsistent assessments and time delays, particularly when evaluating disease progression over time.To address this challenge, our team aims to develop an automated system that not only detects the presence of diseases in chest X-rays but also tracks changes in patient health across a series of X-rays. This system will enable early detection of decline or improvement in a patient's condition, allowing clinicians to make more informed treatment decisions. This is essential because conventional AI models are limited to binary classifications of disease presence, without providing insights into the progression or regression of conditions.
\subsection{Inputs and Outputs}
\subsubsection{Inputs}

\begin{itemize}
    \item \textbf{Medical Images:} A set of chest X-rays provided over time for analysis.
    \item \textbf{Patient Information:} Basic information such as patient ID and scan dates to track progression.
    \item \textbf{Patient/Physician Input:} Additional inputs from patients or doctors describing symptoms like cough, chest pain, or fever.
\end{itemize}

\subsubsection{Outputs}

\begin{itemize}
    \item \textbf{Disease Detection:} The system outputs whether a particular disease is present or absent in the X-rays.
    \item \textbf{Progression Analysis:} The system tracks changes over time, indicating if the patient’s condition has improved, worsened, or remained stable.
    \item \textbf{Visual Aids:} Graphical representation of affected areas on the X-ray images.
    \item \textbf{Structured Report Generation:} A summary report with key findings, disease detection results, progression status, and confidence levels.

\end{itemize}



\subsection{Stakeholders}
1. Physicians \\
Physicians are experts in interpreting X-ray images and their background knowledge can help validate the AI model's accuracy. They will be the primary users of the application as it would help them enhance efficiency and reduce error in analyzing chest X-ray images. \\
\\
2. Patients\\
Patients are the end-users of the AI model as they undergo X-ray examinations.They can upload the X-ray by themselves as well to the application for self-diagnosing purposes. This would provide patients with fast and high-accuracy diagnoses which ensures that they get timely treatment, improving their health outcome.\\
\\
3. Healthcare provider\\
This refers to  hospitals, clinics, and other medical facilities that will implement the AI model. Adapting this AI model to their system will provide a more streamlined workflow, providing patients a better experience and ease the work for physicians. \\
\\
4. Technical support\\
This refers to people who provide maintenance and support for the application throughout its lifespan, ensuring a smooth user experience and integration into existing systems.
\\
5. Developers\\
This refers to people who write, maintain and operate the software, ensuring the capture of bugs, and possible feature-enhancement in the future.
\\
\subsection{Environment}
This software will be compatible with Windows 10 or higher and macOS 12 or higher. Compatibility with other version of Windows, macOS is likely but will not be guaranteed nor tested.\\

\newpage
\section{Goals}
\begin{table}[H] % Change from [ht] to [H] for stricter control
    \hspace*{-2cm}
    \centering
    \begin{tabular}{|p{3cm}|p{6cm}|p{6cm}|}
    \hline
    \textbf{Goal} & \textbf{Explanation} & \textbf{Reasoning} \\ \hline
    Accurate Disease Detection & The system must detect diseases such as pneumonia and other cardiac conditions in chest X-rays with a minimum accuracy of 85\%. & High accuracy in disease detection is essential to ensure the system’s reliability and to support clinical decisions. \\ \hline

    Automated Report Generation & The system will generate structured, human-readable reports that summarize key findings, confidence levels, and disease progression status for each patient. These reports will be easily interpretable by clinicians and require no additional formatting. & Automating the report generation process streamlines clinical workflows by providing a ready-to-use summary of results. This minimizes the time clinicians spend manually reviewing raw data and allows them to focus on decision-making, improving overall efficiency in medical settings. \\ \hline
    
    Disease Progression Tracking Mechanism & A feature will be implemented that tracks disease progression or improvement over multiple chest X-rays taken from the same patient within a specific time frame, providing actionable insights on changes in the patient’s condition. &Tracking disease progression is crucial in offering a longitudinal analysis of the patient's health. By analyzing X-rays over time, the system will help clinicians assess the effectiveness of treatments and adjust strategies accordingly, ensuring that care is continuously optimized.\\ \hline

    Intuitive User Interface & The design of the user interface for this system must be simple and intuitive, allowing healthcare professionals with varying technical backgrounds to use the platform efficiently. It is essential that the interface enables users to easily upload chest X-rays, view results, and generate reports without requiring extensive training. &The user base for this project includes clinicians and radiologists, many of whom may have limited experience with AI tools or complex interfaces. A streamlined and user-friendly design will ensure that users can navigate the system quickly, minimizing errors and maximizing adoption in clinical settings. This will help ensure that the system can be integrated seamlessly into everyday workflows.  \\ \hline
    
    Robust Data Privacy  & The system will ensure that users feel secure when uploading their chest X-ray images by implementing strong data encryption and offering enhanced security features. Users will have the option to enable additional security measures, such as two-factor authentication, to provide them with full control over their data protection. & By offering users an additional verification layer, such as two-factor authentication, the system not only ensures compliance with privacy regulations but also builds trust, allowing users to feel confident that their sensitive health data is fully secured and only accessible by authorized individuals. \\ \hline
    \end{tabular}
    \caption{Project Goals}
\end{table}
\section{Stretch Goals}
\begin{table}[ht]
    \hspace*{-2cm}
    \centering
    \begin{tabular}{|p{3cm}|p{6cm}|p{6cm}|}
    \hline
    \textbf{Stretch Goal} & \textbf{Explanation} & \textbf{Reasoning} \\ \hline

    Alert System for Immediate Actions & An alert mechanism will flag critical cases, such as severe pneumonia or rapid disease progression, and trigger automatic notifications to clinicians for immediate intervention. & Early detection and prompt response can be critical in life-threatening conditions. An alert system ensures that urgent cases are prioritized, improving patient outcomes. \\ \hline

    Demographic-Specific Model Tuning & The model will be fine-tuned to account for variations in patient demographics like age, gender, and race, with adjustments made to minimize bias in predictions across different groups. & Tailoring the model to recognize demographic differences promotes fairness and accuracy across diverse populations, addressing ethical concerns and improving clinical outcomes for all patients. \\ \hline

    Integration of External Medical Datasets & Additional datasets (such as MIMIC-CXR or NIH Chest X-rays) will be integrated to expand the training set and validate the model’s performance on a broader range of chest X-ray images and disease conditions. & Incorporating external data enhances the model’s robustness and generalizability, ensuring it performs well on a variety of real-world cases, making it more applicable in different healthcare environments. \\ \hline

    \end{tabular}
    \caption{Stretch Goals}
\end{table}
\section{Challenge Level and Extras}

\textbf{Challenge Level: Advanced} \\ \\ 
We believe the challenge level of this project to be advanced due to the complexity of domain knowledge and implementation required for this project. Firstly, to work with chest X-ray images for disease diagnosis it requires a deep understanding of medical imaging. Interpreting radio logical data, dealing with imbalanced datasets, and ensuring proper data labelling are significant obstacles. Also, estimating patient demographics such as race, age, and gender introduces further complexity, as these features may present subtle visual cues that could risk introducing bias into the model. Addressing disease prediction and demographic classification from X-ray images requires careful handling and validation, making the task more challenging.
Furthermore, using PyTorch to build deep learning models for multi-task classification adds to the difficulty. Pre-processing medical images, as we need to understand a pattern and handle large datasets, to design a well-designed model. The project also faces technical hurdles such as multi-label classification, where the model needs to predict multiple outputs simultaneously (disease, race, age, gender), increasing the complexity of training and fine-tuning the network. Additionally, addressing ethical concerns, such as ensuring fairness and minimizing bias in predictions related to race and gender, adds another layer of difficulty to the project, making it a highly challenging and nuanced task.

\section*{Appendix --- Reflection}


\begin{enumerate}
    \item What went well while writing this deliverable? \\\\
While writing this deliverable our Team was able to get a better understanding of the project’s technical requirements and create a helpful time management system. Identifying the challenges related to multi-label classification, transfer learning, and ethical considerations helped outline the advanced nature of the project. Furthermore, PyTorch as a framework for implementing neural networks was a strong choice, as it provided flexibility for the complex modelling tasks at hand. We find that is very well documented and learning PyTorch comprehensively will be a great asset in this project.  
    \item What pain points did you experience during this deliverable, and how \\\\
    The biggest challenge our team encountered was to limit and understand the scope of the project to ensure it remained feasible within the constraints of a Capstone while maintaining sufficient complexity. Initially, the project was overly ambitious, especially regarding the number of tasks (disease detection, demographic predictions) and the potential data processing workload. This was addressed by narrowing the focus to the most critical diseases and using existing datasets to avoid time-consuming data collection. We also considered using transfer learning to aide our models to reduce the training time and improve performance on small medical datasets.
    did you resolve them?
    \item How did you and your team adjust the scope of your goals to ensure
    they are suitable for a Capstone project (not overly ambitious but also of
    appropriate complexity for a senior design project)? \\\\
    Our team ensured our scope for our Capstone was appropriate by scaling down to the diseases that are highly prevalent in chest X-rays and focusing on obtaining accuracy in one dataset. Through this decision, we can reduce the dataset complexity and the workload required for both training and model validation substantially. In addition, including the demographic predictions (age, race, and gender) as a secondary goal, and prioritizing disease classification would ensure we maintain a challenging and comprehensive project. By setting clear milestones and focusing on realistic, high-impact deliverables, we made sure the project was ambitious enough but also achievable within the given timeframe.
\end{enumerate}  

\end{document}