\documentclass{article}

\usepackage{tabularx}
\usepackage{booktabs}
\usepackage{longtable}

\title{Reflection and Traceability Report on \progname}

\author{\authname}

\date{}

%% Comments

\usepackage{color}

\newif\ifcomments\commentstrue %displays comments
%\newif\ifcomments\commentsfalse %so that comments do not display

\ifcomments
\newcommand{\authornote}[3]{\textcolor{#1}{[#3 ---#2]}}
\newcommand{\todo}[1]{\textcolor{red}{[TODO: #1]}}
\else
\newcommand{\authornote}[3]{}
\newcommand{\todo}[1]{}
\fi

\newcommand{\wss}[1]{\authornote{blue}{SS}{#1}} 
\newcommand{\plt}[1]{\authornote{magenta}{TPLT}{#1}} %For explanation of the template
\newcommand{\an}[1]{\authornote{cyan}{Author}{#1}}

%% Common Parts

\newcommand{\progname}{ProgName} % PUT YOUR PROGRAM NAME HERE
\newcommand{\authname}{Team \#, Team Name
\\ Student 1 name
\\ Student 2 name
\\ Student 3 name
\\ Student 4 name} % AUTHOR NAMES                  

\usepackage{hyperref}
    \hypersetup{colorlinks=true, linkcolor=blue, citecolor=blue, filecolor=blue,
                urlcolor=blue, unicode=false}
    \urlstyle{same}
                                


\begin{document}

\maketitle

\plt{Reflection is an important component of getting the full benefits from a
learning experience.  Besides the intrinsic benefits of reflection, this
document will be used to help the TAs grade how well your team responded to
feedback.  Therefore, traceability between Revision 0 and Revision 1 is and
important part of the reflection exercise.  In addition, several CEAB (Canadian
Engineering Accreditation Board) Learning Outcomes (LOs) will be assessed based
on your reflections.}

\section{Changes in Response to Feedback}

\subsection{SRS and Hazard Analysis}

Here is the feedback we received on the SRS and Hazard Analysis documents, and the changes we made in response to that feedback.

\begin{longtable}{| p{0.2\textwidth} | p{0.2\textwidth} | p{0.3\textwidth} | p{0.1\textwidth} |}
    \caption{Feedback and Changes for SRS Documentation} \\
    \hline
    \textbf{Feedback Source} & \textbf{Feedback Item} & \textbf{Response} & \textbf{Issue} \\
    \hline
    \endfirsthead
    \hline
    \textbf{Feedback Source} & \textbf{Feedback Item} & \textbf{Response} & \textbf{Issue} \\
    \hline
    \endhead
    \hline
    \endfoot
    TA Feedback & Formatting and Style & Mention figures in paragraphs and fix title & \href{https://github.com/RezaJodeiri/CXR-Capstone/issues/125}{\#125}\\
    \hline
    TA Feedback & What not How(Abstract) & Improve constraints details & \href{https://github.com/RezaJodeiri/CXR-Capstone/issues/126}{\#126} \\
    \hline
    TA Feedback & Complete, Correct and Unambiguous & Template explanation & \href{https://github.com/RezaJodeiri/CXR-Capstone/issues/124}{\#124}\\
    \hline
    TA Feedback & Traceable Requirements & Fix referencing for section 5.2. & \href{https://github.com/RezaJodeiri/CXR-Capstone/issues/123}{\#123} \\
    \hline
    TA Feedback & Document Content & Fix functional requirements. & \href{https://github.com/RezaJodeiri/CXR-Capstone/issues/122}{\#122} \\
    \hline
    Peer Review & Project Goals & Goal Statements Inconsistency. & \href{https://github.com/RezaJodeiri/CXR-Capstone/issues/55}{\#55} \\
    \hline
    Team Feedback & Document Content & Fix FR and NFR to align with the current scope of project & \href{https://github.com/RezaJodeiri/CXR-Capstone/issues/201}{\#201}\\
    \hline

\end{longtable}

\subsection{Design and Design Documentation}

\subsection{VnV Plan and Report}

Here is the feedback we received on the VNV Plan and VNV  Report, and the changes we made in response to that feedback.

\begin{longtable}{| p{0.2\textwidth} | p{0.2\textwidth} | p{0.3\textwidth} | p{0.1\textwidth} |}
    \caption{Feedback and Changes for VNV Plan} \\
    \hline
    \textbf{Feedback Source} & \textbf{Feedback Item} & \textbf{Response} & \textbf{Issue} \\
    \hline
    \endfirsthead
    \hline
    \textbf{Feedback Source} & \textbf{Feedback Item} & \textbf{Response} & \textbf{Issue} \\
    \hline
    \endhead
    \hline
    \endfoot
    TA Feedback & Nondynamic testing used as necessary & Improve Testing & \href{https://github.com/RezaJodeiri/CXR-Capstone/issues/196}{\#196} \\
    \hline
    TA Feedback & General Information & Objective mismatching & \href{https://github.com/RezaJodeiri/CXR-Capstone/issues/195}{\#195} \\
    \hline
    Team Feedback & System Tests for Functional / Nonfunctional Requirements are specific & \href{https://github.com/RezaJodeiri/CXR-Capstone/issues/203}{\#203}\\
    \hline

\end{longtable}

\section{Challenge Level and Extras}

\subsection{Challenge Level}

\plt{State the challenge level (advanced, general, basic) for your project.  Your challenge level should exactly match what is included in your problem statement.  This should be the challenge level agreed on between you and the course instructor.}

\subsection{Extras}

\plt{Summarize the extras (if any) that were tackled by this project.  Extras
can include usability testing, code walkthroughs, user documentation, formal
proof, GenderMag personas, Design Thinking, etc.  Extras should have already
been approved by the course instructor as included in your problem statement.}

\section{Design Iteration (LO11 (PrototypeIterate))}

\plt{Explain how you arrived at your final design and implementation.  How did
the design evolve from the first version to the final version?} 

\plt{Don't just say what you changed, say why you changed it.  The needs of the
client should be part of the explanation.  For example, if you made changes in
response to usability testing, explain what the testing found and what changes
it led to.}

\section{Design Decisions (LO12)}

\plt{Reflect and justify your design decisions.  How did limitations,
 assumptions, and constraints influence your decisions?  Discuss each of these
 separately.}

\section{Economic Considerations (LO23)}

\plt{Is there a market for your product? What would be involved in marketing your 
product? What is your estimate of the cost to produce a version that you could 
sell?  What would you charge for your product?  How many units would you have to 
sell to make money? If your product isn't something that would be sold, like an 
open source project, how would you go about attracting users?  How many potential 
users currently exist?}

\section{Reflection on Project Management (LO24)}

\plt{This question focuses on processes and tools used for project management.}

\subsection{How Does Your Project Management Compare to Your Development Plan}

\plt{Did you follow your Development plan, with respect to the team meeting plan, 
team communication plan, team member roles and workflow plan.  Did you use the 
technology you planned on using?}

\subsection{What Went Well?}

\plt{What went well for your project management in terms of processes and 
technology?}

\subsection{What Went Wrong?}

\plt{What went wrong in terms of processes and technology?}

\subsection{What Would you Do Differently Next Time?}

\plt{What will you do differently for your next project?}

\section{Reflection on Capstone}

\plt{This question focuses on what you learned during the course of the capstone project.}

\subsection{Which Courses Were Relevant}

\plt{Which of the courses you have taken were relevant for the capstone project?}

\subsection{Knowledge/Skills Outside of Courses}

\plt{What skills/knowledge did you need to acquire for your capstone project
that was outside of the courses you took?}

\end{document}