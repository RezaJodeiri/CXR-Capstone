\documentclass{article}

\usepackage{float}
\restylefloat{table}

\usepackage{booktabs}

\title{Team Contributions: Rev 0\\\progname}

\author{\authname}

\date{}

%% Comments

\usepackage{color}

\newif\ifcomments\commentstrue %displays comments
%\newif\ifcomments\commentsfalse %so that comments do not display

\ifcomments
\newcommand{\authornote}[3]{\textcolor{#1}{[#3 ---#2]}}
\newcommand{\todo}[1]{\textcolor{red}{[TODO: #1]}}
\else
\newcommand{\authornote}[3]{}
\newcommand{\todo}[1]{}
\fi

\newcommand{\wss}[1]{\authornote{blue}{SS}{#1}} 
\newcommand{\plt}[1]{\authornote{magenta}{TPLT}{#1}} %For explanation of the template
\newcommand{\an}[1]{\authornote{cyan}{Author}{#1}}

%% Common Parts

\newcommand{\progname}{ProgName} % PUT YOUR PROGRAM NAME HERE
\newcommand{\authname}{Team \#, Team Name
\\ Student 1 name
\\ Student 2 name
\\ Student 3 name
\\ Student 4 name} % AUTHOR NAMES                  

\usepackage{hyperref}
    \hypersetup{colorlinks=true, linkcolor=blue, citecolor=blue, filecolor=blue,
                urlcolor=blue, unicode=false}
    \urlstyle{same}
                                


\begin{document}

\maketitle

This document summarizes the contributions of each team member for the Rev 0
Demo.  The time period of interest is the time between the POC demo and the Rev
0 demo.

\section{Demo Plans}

\wss{What will you be demonstrating}

\section{Team Meeting Attendance}

\wss{For each team member how many team meetings have they attended over the
time period of interest.  This number should be determined from the meeting
issues in the team's repo.  The first entry in the table should be the total
number of team meetings held by the team.}

\begin{table}[H]
\centering
\begin{tabular}{ll}
\toprule
\textbf{Student} & \textbf{Meetings}\\
\midrule
Total & 10\\
Ayman Akhras & 6\\
Nathan Luong & 8\\
Patrick Zhou & 8\\
Kelly Deng & 7\\
Reza Jodeiri & 8\\
\bottomrule
\end{tabular}
\end{table}

All students had their own difficult problems to solve. People with higher meeting counts tended to meet with each other more often.
\section{Supervisor/Stakeholder Meeting Attendance}

\wss{For each team member how many supervisor/stakeholder team meetings have
they attended over the time period of interest.  This number should be determined
from the supervisor meeting issues in the team's repo.  The first entry in the
table should be the total number of supervisor and team meetings held by the
team.  If there is no supervisor, there will usually be meetings with
stakeholders (potential users) that can serve a similar purpose.}

\begin{table}[H]
\centering
\begin{tabular}{ll}
\toprule
\textbf{Student} & \textbf{Meetings}\\
\midrule
Total & 2\\
Ayman Akhras & 2\\
Nathan Luong & 2\\
Patrick Zhou & 2\\
Kelly Deng & 2\\
Reza Jodeiri & 2\\
\bottomrule
\end{tabular}
\end{table}

\wss{If needed, an explanation for the counts can be provided here.}

\section{Lecture Attendance}

\wss{For each team member how many lectures have they attended over the time
period of interest.  This number should be determined from the lecture issues in
the team's repo.  The first entry in the table should be the total number of
lectures since the beginning of the term.}

\begin{table}[H]
\centering
\begin{tabular}{ll}
\toprule
\textbf{Student} & \textbf{Lectures}\\
\midrule
Total & 15\\
Ayman Akhras & 1\\
Nathan Luong & 1\\
Patrick Zhou & 2\\
Kelly Deng & 2\\
Reza Jodeiri & 9\\
\bottomrule
\end{tabular}
\end{table}

\wss{If needed, an explanation for the lecture attendance can be provided here.}

\section{TA Document Discussion Attendance}

\wss{For each team member how many of the informal document discussion meetings
with the TA were attended over the time period of interest.}

\begin{table}[H]
\centering
\begin{tabular}{ll}
\toprule
\textbf{Student} & \textbf{Lectures}\\
\midrule
Total & 24\\
Ayman Akhras & 8\\
Nathan Luong & 1\\
Patrick Zhou & 2\\
Kelly Deng & 2\\
Reza Jodeiri & 6\\
\bottomrule
\end{tabular}
\end{table}

For the purpose of this report, any form of communication with the TA, including both formal meetings and informal messaging, is considered a "meeting." This includes questions asked and answered via email, messaging platforms, or any other medium of communication. The following table summarizes the number of such communications for each team member over the time period of interest.

\section{Commits}


\begin{table}[H]
\centering
\begin{tabular}{lll}
\toprule
\textbf{Student} & \textbf{Commits} & \textbf{Percent}\\
\midrule
Total & 327 & 100\% \\
Ayman Akhras & 59 & 18.04\%\\
Nathan Luong & 53 & 16.21\%\\
Patrick Zhou & 24 & 7.34\%\\
Kelly Deng & 45 & 13.76\%\\
Reza Jodeiri & 146 & 44.65\%\\
\bottomrule
\end{tabular}
\end{table}



\section{Issue Tracker}

\wss{For each team member how many issues have they authored (including open and
closed issues (O+C)) and how many have they been assigned (only counting closed
issues (C only)) over the time period of interest.}

\begin{table}[H]
\centering
\begin{tabular}{lll}
\toprule
\textbf{Student} & \textbf{Authored (O+C)} & \textbf{Assigned (C only)}\\
\midrule
Ayman Akhras  & 0 & 3 \\
Nathan Luong & 3 & 3 \\
Patrick Zhou & 1 & 3 \\
Kelly Deng & 0 & 3 \\
Reza Jodeiri & 25 & 10 \\
\bottomrule
\end{tabular}
\end{table}

\wss{If needed, an explanation for the counts can be provided here.}

\section{CICD}

Our CI/CD pipeline ensures our project's code quality and automates essential processes like testing and deployment. It begins with setting up the environment, where the latest code is checked out from our repository, Python is configured, and dependencies are installed while utilizing cached pip packages for efficiency. Next, the pipeline performs linting and type checks using tools such as flake8 and mypy to maintain code consistency and catch potential issues early. Finally, it runs unit and integration tests using a framework like pytest or unittest to verify that changes do not break existing functionality. This automated workflow is triggered whenever code is pushed or merged, streamlining our development process, reducing human errors, and ensuring a stable and reliable codebase.




\end{document}