\documentclass[12pt, titlepage]{article}

\usepackage{booktabs}
\usepackage{tabularx}
\usepackage{hyperref}
\hypersetup{
    colorlinks,
    citecolor=black,
    filecolor=black,
    linkcolor=red,
    urlcolor=blue
}
\usepackage[round]{natbib}

%% Comments

\usepackage{color}

\newif\ifcomments\commentstrue %displays comments
%\newif\ifcomments\commentsfalse %so that comments do not display

\ifcomments
\newcommand{\authornote}[3]{\textcolor{#1}{[#3 ---#2]}}
\newcommand{\todo}[1]{\textcolor{red}{[TODO: #1]}}
\else
\newcommand{\authornote}[3]{}
\newcommand{\todo}[1]{}
\fi

\newcommand{\wss}[1]{\authornote{blue}{SS}{#1}} 
\newcommand{\plt}[1]{\authornote{magenta}{TPLT}{#1}} %For explanation of the template
\newcommand{\an}[1]{\authornote{cyan}{Author}{#1}}

%% Common Parts

\newcommand{\progname}{ProgName} % PUT YOUR PROGRAM NAME HERE
\newcommand{\authname}{Team \#, Team Name
\\ Student 1 name
\\ Student 2 name
\\ Student 3 name
\\ Student 4 name} % AUTHOR NAMES                  

\usepackage{hyperref}
    \hypersetup{colorlinks=true, linkcolor=blue, citecolor=blue, filecolor=blue,
                urlcolor=blue, unicode=false}
    \urlstyle{same}
                                


\begin{document}

\title{Verification and Validation Report: \progname} 
\author{\authname}
\date{\today}
	
\maketitle

\pagenumbering{roman}

\section{Revision History}

\begin{tabularx}{\textwidth}{p{3cm}p{2cm}X}
\toprule {\bf Date} & {\bf Version} & {\bf Notes}\\
\midrule
Date 1 & 1.0 & Notes\\
Date 2 & 1.1 & Notes\\
\bottomrule
\end{tabularx}

~\newpage

\section{Symbols, Abbreviations and Acronyms}

\renewcommand{\arraystretch}{1.2}
\begin{tabular}{l l} 
  \toprule		
  \textbf{symbol} & \textbf{description}\\
  \midrule 
  T & Test\\
  \bottomrule
\end{tabular}\\

\wss{symbols, abbreviations or acronyms -- you can reference the SRS tables if needed}

\newpage

\tableofcontents

\listoftables %if appropriate

\listoffigures %if appropriate

\newpage

\pagenumbering{arabic}

This document ...

\section{Functional Requirements Evaluation}
\subsection{Functional Requirement 1}
\begin{itemize}
  \item Description: The system shall accept chest X-ray images as input from authorized users.
  \item Type: Manual
  \item Verification: Calling HTTP Server Module API to get a presigned URL for uploading the image with and without authorization token.
  \item Validation: Presigned URL doesn\'t return if the user is not authorized.
  \item Result: Pass
\end{itemize}

\subsection{Functional Requirement 2}
\begin{itemize}
  \item Description: The system shall enable users to input additional patient symptoms, such as cough, chest pain, or fever
  \item Type: Automated
  \item Verification: Execute integration tests on the Medical Record Management Module to include additional patient symptoms.
  \item Validation: Check if the additional patient symptoms are included in the medical record from the Data Persistent Module.
  \item Result: Pass
\end{itemize}

\subsection{Functional Requirement 3}
\begin{itemize}  
  \item Description: The system shall analyze chest X-ray images to detect the presence or absence of specific diseases with an accuracy of 85\% or higher.
  \item Type: Automated
  \item Verification: Execute Validation Script on the Disease Detection Module to check the accuracy of the AI model.
  \item Validation: Check if the accuracy of the AI model is 85\% or higher.
  \item Result: Pass
\end{itemize}

\subsection{Functional Requirement 4}
\begin{itemize}  
  \item Description: The system shall indicate whether a patient\'s condition has improved, worsened, or remained stable between scans.
  \item Type: Automated
  \item Verification: Execute integration tests on the Disease Progression Module, with pre-defined patient conditions.
  \item Validation: Validated the output of the Module to include improved, worsened, or remained stable conditions.
  \item Result: Pass
\end{itemize}

\subsection{Functional Requirement 5}
\begin{itemize}  
  \item Description: The system shall generate visual aids by highlighting affected areas on the chest X-ray images.
  \item Type: Manual
  \item Verification: Manually create a medical record on the X-Ray Report View Module.
  \item Validation: Validated the output of the Module to include highlighted affected areas.
  \item Result: Pass
\end{itemize}

\subsection{Functional Requirement 6}
\begin{itemize}  
  \item Description: The system shall produce a structured, human-readable report summarizing key findings, disease detection results, and progression status.
  \item Type: Automated
  \item Verification: Execute integration tests on the Internal Report Generation Service, with pre-defined patient conditions.
  \item Validation: Validated the output of the Service to include key findings, disease detection results, and progression status.
  \item Result: Pass
\end{itemize}

\subsection{Functional Requirement 7}
\begin{itemize}  
  \item Description: The system shall store patient data, including images and reports, in a secure database for future reference.
  \item Type: Automated
  \item Verification: Execute integration tests on Medical Record Management Module to create a medical record with images and reports.
  \item Validation: Validated that all medical records, findings, and X-Ray images are stored in the Data Persistent Module.
  \item Result: Pass
\end{itemize}

\subsection{Functional Requirement 8}
\begin{itemize}  
  \item Description: The system shall provide alerts to clinicians if significant changes in a patient\'s condition are detected between scans.
  \item Type: Manual
  \item Verification: Manually create a medical record on the Disease Progression Module with significant changes in a patient\'s condition.
  \item Validation: Validated that the drastic changes in a patient\'s condition are presence on the UI of Patient List View Module and Patient Overview Module.
  \item Result: Pass
\end{itemize}

\subsection{Functional Requirement 9}
\begin{itemize}  
  \item Description: The system shall allow healthcare professionals to adjust treatment plans based on the X-ray analysis results
  \item Type: Manual
  \item Verification: Manually edit a medical record on the X-Ray Report View Module with adjusted treatment plans.
  \item Validation: Validated that the adjusted treatment plans are presence on the UI of Patient List View Module and Data Persistent Module.
  \item Result: Pass
\end{itemize}

\subsection{Functional Requirement 10}
\begin{itemize}  
  \item Description: The system shall display confidence levels for disease detection and progression analysis results through an intuitive user interface that requires minimal training to operate.
  \item Type: Manual
  \item Verification: Manually create a medical record on the X-Ray Report View Module with known disease confidence levels.
  \item Validation: Validated that the confidence levels are accurately shown on the UI of X-Ray Report View Module.
  \item Result: Pass
\end{itemize}

\subsection{Functional Requirement 11}
\begin{itemize}  
  \item Description: The system shall support multiple user roles with appropriate access levels (e.g., physician, patient).
  \item Type: Automated
  \item Verification: Execute integration tests on the User Management Module to create doctors and patients.
  \item Validation: Validated that created doctors has the is_doctor flag set to true and newly created patient always attached to a doctor via the doctor_id field.
  \item Result: Pass
\end{itemize}

\subsection{Functional Requirement 12}
\begin{itemize}  
  \item Description: The system shall create a new copy of a patient\'s X-ray before running the AI model for analysis.
  \item Type: Automated
  \item Verification: Execute integration tests on the Medical Record Management Module to make a new record.
  \item Validation: Validated that patient\'s X-ray is already copied into the Data Persistent Module before running through the Disease Detection and Progression Module.
  \item Result: Pass
\end{itemize}

\subsection{Functional Requirement 13}
\begin{itemize}  
  \item Description: The system shall support regular updates to the AI model to incorporate new data and improve accuracy over time.
  \item Type: Automated
  \item Verification: Execute integration tests on the AI Model Update Module to ensure updates are applied correctly.
  \item Validation: Validate that the AI model\'s accuracy improves over time with new data.
  \item Result: Pass
\end{itemize}

\section{Nonfunctional Requirements Evaluation}

\subsection{Usability}
		
\subsection{Performance}

\subsection{etc.}
	
\section{Comparison to Existing Implementation}	

This section will not be appropriate for every project.

\section{Unit Testing}

\section{Changes Due to Testing}

\wss{This section should highlight how feedback from the users and from 
the supervisor (when one exists) shaped the final product.  In particular 
the feedback from the Rev 0 demo to the supervisor (or to potential users) 
should be highlighted.}

\section{Automated Testing}
		
\section{Trace to Requirements}
		
\section{Trace to Modules}		

\section{Code Coverage Metrics}

\bibliographystyle{plainnat}
\bibliography{../../refs/References}

\newpage{}
\section*{Appendix --- Reflection}

The information in this section will be used to evaluate the team members on the
graduate attribute of Reflection.

The purpose of reflection questions is to give you a chance to assess your own
learning and that of your group as a whole, and to find ways to improve in the
future. Reflection is an important part of the learning process.  Reflection is
also an essential component of a successful software development process.  

Reflections are most interesting and useful when they're honest, even if the
stories they tell are imperfect. You will be marked based on your depth of
thought and analysis, and not based on the content of the reflections
themselves. Thus, for full marks we encourage you to answer openly and honestly
and to avoid simply writing ``what you think the evaluator wants to hear.''

Please answer the following questions.  Some questions can be answered on the
team level, but where appropriate, each team member should write their own
response:


\begin{enumerate}
  \item What went well while writing this deliverable? 
  \item What pain points did you experience during this deliverable, and how
    did you resolve them?
  \item Which parts of this document stemmed from speaking to your client(s) or
  a proxy (e.g. your peers)? Which ones were not, and why?
  \item In what ways was the Verification and Validation (VnV) Plan different
  from the activities that were actually conducted for VnV?  If there were
  differences, what changes required the modification in the plan?  Why did
  these changes occur?  Would you be able to anticipate these changes in future
  projects?  If there weren't any differences, how was your team able to clearly
  predict a feasible amount of effort and the right tasks needed to build the
  evidence that demonstrates the required quality?  (It is expected that most
  teams will have had to deviate from their original VnV Plan.)
\end{enumerate}

\end{document}