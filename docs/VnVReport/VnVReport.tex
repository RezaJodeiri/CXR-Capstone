\documentclass[12pt, titlepage]{article}
\usepackage{pdflscape}
\usepackage{longtable}
\usepackage{booktabs}
\usepackage{tabularx}
\usepackage{hyperref}
\hypersetup{
    colorlinks,
    citecolor=black,
    filecolor=black,
    linkcolor=red,
    urlcolor=blue
}
\usepackage[round]{natbib}

\input{../packages/Comments.tex}
%% Common Parts

\newcommand{\progname}{CXR} % PUT YOUR PROGRAM NAME HERE
\newcommand{\authname}{Team 27, Neuralyzers
\\ Ayman Akhras 
\\ Nathan Luong
\\ Patrick Zhou
\\ Kelly Deng
\\ Reza Jodeiri} % AUTHOR NAMES                  

\usepackage{hyperref}
    \hypersetup{colorlinks=true, linkcolor=blue, citecolor=blue, filecolor=blue,
                urlcolor=blue, unicode=false}
    \urlstyle{same}
                                

\input{../packages/Reflection.tex}

\begin{document}

\title{Verification and Validation Report: \progname} 
\author{\authname}
\date{\today}
	
\maketitle

\pagenumbering{roman}

\section{Revision History}

\begin{tabularx}{\textwidth}{p{3cm}p{2cm}X}
\toprule {\bf Date} & {\bf Version} & {\bf Notes}\\
\midrule
Date 1 & 1.0 & Notes\\
Date 2 & 1.1 & Notes\\
\bottomrule
\end{tabularx}

~\newpage

\section{Symbols, Abbreviations, and Acronyms}
This section records information for easy reference and aims to reduce ambiguity in understanding key concepts used in the project.

\subsection{Table of Units}

Throughout this document, SI (Système International d'Unités) is employed as the unit system. In addition to basic units, several derived units are used as described below. For each unit, the symbol is given, followed by a description of the unit and the SI name.

\renewcommand{\arraystretch}{1.2}
\noindent \begin{tabular}{l l l} 
    \toprule		
    \textbf{Symbol} & \textbf{Unit} & \textbf{SI}\\
    \midrule 
    \si{s} & Time & Second\\
    \si{GB} & Data & Gigabyte\\
    \si{MB} & Data & Megabyte\\
    \si{LOC} & Quantity & Lines of Code\\
    \bottomrule
\end{tabular}

\subsection{Definitions}
This subsection provides a list of terms that are used in the subsequent sections and their meanings, with the purpose of reducing ambiguity and making it easier to correctly understand the requirements:

\begin{itemize}
    
    
    \item[-] \textbf{Artificial Intelligence (AI) Model}: A program that analyzes datasets to identify patterns and make predictions. Used extensively in medical image analysis for automating diagnostics. 
    
    \item[-] \textbf{Convolutional Neural Network (CNN)}: A deep learning algorithm that processes images by assigning weights and biases, allowing it to identify patterns and features in medical images such as chest X-rays.
    
    \item[-] \textbf{Detection Transformer (DETR)}: A transformer-based neural network model for object detection. It uses an encoder-decoder transformer architecture to directly predict bounding boxes and class labels from an image, simplifying the detection process.
    
    \item[-] \textbf{DICOM (Digital Imaging and Communications in Medicine)}: The international standard for medical images, defining formats for image exchange that ensure clinical quality.
    
    \item[-] \textbf{Containerized Application}: A portable version of an application that can be run on a container run-time, such as Docker.
  
    \item[-] \textbf{Machine Learning (ML)}: A subset of AI focusing on using data and algorithms to mimic human learning, improving accuracy over time.
    \item[-] \textbf{Picture Archiving and Communication System (PACS)}: A system for acquiring, storing, transmitting, and displaying medical images digitally, providing a filmless clinical environment.
    \item[-] \textbf{PHI}: Personal Health Information - Private and confidential data that must be protected under the HIPPA act.
    \item[-] \textbf{HIPPA}: Health Insurance Portability and Accountability Act, a set of standards protecting sensitive health information from disclosure without patient's consent.
    \item[-] \textbf{AWS - Amazon Web Services}: A public cloud provider, offering all HIPAA-compliance cloud services that helps Neuralanalyzer host, manage, and scale our application.
    \item[-] \textbf{AWS ECS}: AWS Elastic Cloud Service: - An AWS managed service for managing and maintaining application containers at run-time.
    \item[-] \textbf{AWS ECR}: AWS Elastic Container Registry - An AWS managed service for storing and managing container images.
    \item[-] \textbf{AWS Fargate}: An AWS managed service for running containerized applications.
    \item[-] \textbf{AWS Cognito}: An AWS managed service for authentication logic, handling user and password management.
    \item[-] \textbf{React}: A web front-end framework, written in Javascript.
    \item[-] \textbf{Flask}: An HTTP-based server framework, written in Python.
    \item[-] \textbf{Finite State Machine (FSM)}: A computation model that simulates sequential logic using state transitions, applied in processes like user authentication and backend workflows.
    
    \item[-] \textbf{ROC Curve (Receiver Operating Characteristic Curve)}: A graph that shows the performance of a classification model by plotting the true positive rate against the false positive rate at various threshold levels.
    
    \item[-] \textbf{Service-Level Agreement (SLA)}: Defines the guaranteed uptime of the system, such as ensuring the availability of the AI service for 99.99\% of operational hours.
    
    \item[-] \textbf{Software as a Medical Device (SaMD)}: Software classified as a medical device under regulatory frameworks, such as those defined by the Food and Drugs Act.
    
    \item[-] \textbf{TorchXRAYVision}: An open-source library for classifying diseases based on chest X-ray images, offering pre-trained models to accelerate the development process.
    
    \item[-] \textbf{X-ray}: A form of high-energy electromagnetic radiation used in medical imaging to produce images of the inside of the body, enabling the diagnosis of conditions through radiographic film or digital detectors.
    
    \item[-] \textbf{MIT License}: An open-source software license that allows for the free use, modification, and distribution of software.
    
    \item[-] \textbf{Training Data}: Refers to the dataset of labeled chest X-ray images used to train the AI model. In this project, the dataset size is approximately 471.12 GB.
    
\end{itemize}

\subsection{Abbreviations and Acronyms}
\renewcommand{\arraystretch}{1.3}
\noindent \begin{tabular}{l l} 
  \toprule		
  \textbf{Symbol} & \textbf{Description}\\
  \midrule 
  SRS & Software Requirements Specification\\
  AI & Artificial Intelligence\\
  CNN & Convolutional Neural Network\\
  DICOM & Digital Imaging and Communications in Medicine\\
  DETR & Detection Transformer\\
  ViT & Vision Transformer\\
  VnV & Verification and Validation\\
  ML & Machine Learning\\
  PACS & Picture Archiving and Communication System\\
  SaMD & Software as a Medical Device\\
  ROC & Receiver Operating Characteristic Curve\\
  SLA & Service-Level Agreement\\
  FR & Functional Requirement\\
  NFR & Non-Functional Requirement\\
  FSM & Finite State Machine\\
  CXR & Chest X-Ray Project\\
  POC & Proof of Concept\\
  TM & Theoretical Model\\
  AWS & Amazon Web Services\\
  ECS & Elastic Container Service\\
  ECR & Elastic Container Registry\\
  CI/CD & Continuous integration and Continuous deployment\\
  HTTP & Hypertext Transfer Protocol\\
  
  \bottomrule
\end{tabular}

\newpage

\tableofcontents

\listoftables %if appropriate

\listoffigures %if appropriate

\newpage

\pagenumbering{arabic}

This document ...

\section{Functional Requirements Evaluation}

\section{Nonfunctional Requirements Evaluation}

\subsection{Usability}
		
\subsection{Performance}

\subsection{etc.}
	
\section{Comparison to Existing Implementation}	

\subsection{Existing Projects}
\begin{itemize}
    \item \href{https://github.com/harrisonchiu/xray/tree/main}{harrisonchiu/xray}
    \item \href{https://github.com/N8THEPL8/ChestLenseAI/tree/main}{N8THEPL8/ChestLenseAI}
    \item \href{https://github.com/PLAN-Lab/CheXRelFormer/tree/main}{PLAN-Lab/CheXRelFormer}
\end{itemize}

\subsection{Project Structure}

\begin{longtable}{|l|l|l|l|l|l|}
\hline
\textbf{Project Name}   & \textbf{FrontEnd}       & \textbf{BackEnd}   & \textbf{Data Base} & \textbf{Cloud} & \textbf{Deployment} \\ \hline
\endfirsthead
\hline
\textbf{Project Name}   & \textbf{FrontEnd}       & \textbf{BackEnd}   & \textbf{Data Base} & \textbf{Cloud} & \textbf{Deployment} \\ \hline
\endhead
\hline
\endfoot

\textbf{Our Capstone}    & React.js               & Flask API          & Amazon S3          & AWS             & Docker             \\ \hline
xray                    & N/A                    & N/A                & N/A                & N/A             & Jupyter Notebook   \\ \hline
ChestLenseAI            & HTML \& CSS             & Flask API          & N/A                & N/A             & Python             \\ \hline
CheXRelFormer           & N/A                    & N/A                & N/A                & N/A             & Shell Command      \\ \hline

\end{longtable}

\newpage
\begin{landscape}
\subsection{Datasets}
\begin{longtable}{|l|p{3.5cm}|p{1.8cm}|p{6cm}|p{4cm}|}
\hline
\textbf{Project Name}   & \textbf{Dataset}                                                                                             & \textbf{Size}                                              & \textbf{Classes}                                                                                                                                                                                                                                                                                              & \textbf{Link}                                                   \\ \hline
\endfirsthead
\hline
\textbf{Project Name}   & \textbf{Dataset}                                                                                             & \textbf{Size}                                              & \textbf{Classes}                                                                                                                                                                                                                                                                                              & \textbf{Link}                                                   \\ \hline
\endhead
\hline
\endfoot

\textbf{Our Capstone}    & MIMIC-CXR-JPG 2.0.0                                                                                           & 557.6 GB                                                   & 9: Lung Opacity, Pleural Effusion, Atelectasis, Enlarged Cardiac Silhouette, Pulmonary Edema/Hazy Opacity, Pneumothorax, Consolidation, Fluid Overload/Heart Failure, Pneumonia. \textbf{Progress}: No Change, Improved, Worsened. & \url{https://physionet.org/content/mimic-cxr-jpg/2.0.0/}        \\ \hline
\textbf{xray}           & Chest-xray14                                                                                                  & 42.0 GB                                                    & 14: Atelectasis, Cardiomegaly, Consolidation, Edema, Effusion, Emphysema, Fibrosis, Hernia, Infiltration, Mass, Nodule, Pleural Thickening, Pneumonia, Pneumothorax.                                                     & \url{https://nihcc.app.box.com/v/ChestXray-NIHCC/folder/37178474737} \\ \hline
\textbf{ChestLenseAI}   & MIMIC-CXR-JPG 2.0.0                                                                                           & 557.6 GB                                                   & 6: Atelectasis, Cardiomegaly, Consolidation, Edema, No Finding, Pleural Effusion.                                                              & \url{https://physionet.org/content/mimic-cxr-jpg/2.0.0/}        \\ \hline
\textbf{CheXRelFormer}  & MIMIC-CXR-JPG 2.0.0, MIMIC-III 1.4                                                                           & 557.6 GB, 6.2 GB                                          & 3: No Change, Improved, Worsened                                                                                                                                                                                                            & \url{https://physionet.org/content/mimic-cxr-jpg/2.0.0/}        \\
\end{longtable}

\subsection{Architectures}

\begin{longtable}{|l|l|l|p{10cm}|}
\hline
\textbf{Project Name}   & \textbf{Neural Network}  & \textbf{Configuration}  & \textbf{Link (graph) (paper)}  \\ \hline
\endfirsthead
\hline
\textbf{Project Name}   & \textbf{Neural Network}  & \textbf{Configuration}  & \textbf{Link (graph) (paper)}  \\ \hline
\endhead
\hline
\endfoot

\textbf{Our Capstone}    & DETR                    & MLP                     & \url{https://viso.ai/wp-content/uploads/2024/02/DETR-Architecture.jpg} \\ 
                         &                         &                         & \url{https://arxiv.org/pdf/2005.12872} \\ \hline
\textbf{xray}           & ResNet                  & ResNet-50               & \url{https://i.ytimg.com/vi/woEs7UCaITo/maxresdefault.jpg} \\ 
                         &                         &                         & \url{https://arxiv.org/pdf/1512.03385} \\ \hline
\textbf{ChestLenseAI}   & DenseNet                & DenseNet-121            & \url{https://pytorch.org/assets/images/densenet1.png} \\ 
                         &                         &                         & \url{https://arxiv.org/pdf/1608.06993} \\ \hline
\textbf{CheXRelFormer}  & ViT                     & MLP                     & \url{https://www.researchgate.net/publication/383905431/figure/fig3/AS:11431281290331182@1731595235002/Structure-of-the-backbone-PVTv2.ppm} \\ 
                         &                         &                         & \url{https://arxiv.org/pdf/2106.13797} \\ \hline

\end{longtable}

\newpage

\subsection{Summary}
\begin{longtable}{|l|p{10cm}|p{4cm}|p{4cm}|}
\hline
\textbf{Project Name}   & \textbf{Summary}                                                                 & \textbf{Reference Project}                              & \textbf{Paper}                                                \\ \hline
\endfirsthead
\hline
\textbf{Project Name}   & \textbf{Summary}                                                                 & \textbf{Reference Project}                              & \textbf{Paper}                                                \\ \hline
\endhead
\hline
\endfoot

\textbf{Our Capstone}    & This project encompasses the entire product lifecycle, integrating Frontend, Backend, Cloud Infrastructure, and a CI/CD pipeline. It is designed to support the detection of diseases, track the progression of conditions over time, and generate AI-driven diagnostic reports for healthcare professionals. & \url{https://github.com/McMasterAIHLab/CheXDetector}         & \url{https://papers.miccai.org/miccai-2024/paper/3269_paper.pdf} \\ \hline
xray                    & This project only contains backend structure for the AI model, which uses a smaller dataset compared to other projects and uses the resnet-50 network for disease classification. & \url{https://github.com/LalehSeyyed/CheXclusion}            & \url{https://arxiv.org/pdf/2003.00827v2}                        \\ \hline
ChestLenseAI            & This project uses a minimal Frontend and applies the DenseNet-121 pre-trained model to detect diseases in chest X-ray images.  & \url{https://github.com/LaurentVeyssier/Chest-X-Ray-Medical-Diagnosis-with-Deep-Learning/tree/main}  & \url{https://arxiv.org/pdf/1711.05225}                          \\ \hline
CheXRelFormer           & This project is purely focused on the backend ViT model, it does not use a pre-trained network for disease progress between two x-ray images. It is a very advanced (PhD) project which is developed with no prior related research.  & N/A                                                      & N/A                                                          \\ \hline

\end{longtable}

\end{landscape}

\section{Unit Testing}

\section{Changes Due to Testing}


\section{Automated Testing}
		
\section{Trace to Requirements}
		
\section{Trace to Modules}		

\section{Code Coverage Metrics}

\bibliographystyle{plainnat}
\bibliography{../../refs/References}



\newpage{}
\section*{Appendix --- Reflection}

The information in this section will be used to evaluate the team members on the
graduate attribute of Reflection.

\input{../Reflection.tex}

\begin{enumerate}
  \item What went well while writing this deliverable? 
  \item What pain points did you experience during this deliverable, and how
    did you resolve them?
  \item Which parts of this document stemmed from speaking to your client(s) or
  a proxy (e.g. your peers)? Which ones were not, and why?
  \item In what ways was the Verification and Validation (VnV) Plan different
  from the activities that were actually conducted for VnV?  If there were
  differences, what changes required the modification in the plan?  Why did
  these changes occur?  Would you be able to anticipate these changes in future
  projects?  If there weren't any differences, how was your team able to clearly
  predict a feasible amount of effort and the right tasks needed to build the
  evidence that demonstrates the required quality?  (It is expected that most
  teams will have had to deviate from their original VnV Plan.)
\end{enumerate}

\end{document}