\documentclass{article}

\usepackage{booktabs}
\usepackage{tabularx}

\title{Development Plan\\\progname}

\author{\authname}

\date{}

%% Comments

\usepackage{color}

\newif\ifcomments\commentstrue %displays comments
%\newif\ifcomments\commentsfalse %so that comments do not display

\ifcomments
\newcommand{\authornote}[3]{\textcolor{#1}{[#3 ---#2]}}
\newcommand{\todo}[1]{\textcolor{red}{[TODO: #1]}}
\else
\newcommand{\authornote}[3]{}
\newcommand{\todo}[1]{}
\fi

\newcommand{\wss}[1]{\authornote{blue}{SS}{#1}} 
\newcommand{\plt}[1]{\authornote{magenta}{TPLT}{#1}} %For explanation of the template
\newcommand{\an}[1]{\authornote{cyan}{Author}{#1}}

%% Common Parts

\newcommand{\progname}{ProgName} % PUT YOUR PROGRAM NAME HERE
\newcommand{\authname}{Team \#, Team Name
\\ Student 1 name
\\ Student 2 name
\\ Student 3 name
\\ Student 4 name} % AUTHOR NAMES                  

\usepackage{hyperref}
    \hypersetup{colorlinks=true, linkcolor=blue, citecolor=blue, filecolor=blue,
                urlcolor=blue, unicode=false}
    \urlstyle{same}
                                

The purpose of reflection questions is to give you a chance to assess your own
learning and that of your group as a whole, and to find ways to improve in the
future. Reflection is an important part of the learning process.  Reflection is
also an essential component of a successful software development process.  

Reflections are most interesting and useful when they're honest, even if the
stories they tell are imperfect. You will be marked based on your depth of
thought and analysis, and not based on the content of the reflections
themselves. Thus, for full marks we encourage you to answer openly and honestly
and to avoid simply writing ``what you think the evaluator wants to hear.''

Please answer the following questions.  Some questions can be answered on the
team level, but where appropriate, each team member should write their own
response:


\begin{document}

\maketitle

\begin{table}[hp]
\caption{Revision History} \label{TblRevisionHistory}
\begin{tabularx}{\textwidth}{llX}
\toprule
\textbf{Date} & \textbf{Developer (s)} & \textbf{Change}\\
\midrule
Date1 & Name (s) & Description of changes\\
Date2 & Name (s) & Description of changes\\
... & ... & ...\\
\bottomrule
\end{tabularx}
\end{table}

\newpage{}

\wss{Put your introductory blurb here.  Often the blurb is a brief roadmap of
what is contained in the report.}

\wss{Additional information on the development plan can be found in the
\href{https://gitlab.cas.mcmaster.ca/courses/capstone/-/blob/main/Lectures/L02b_POCAndDevPlan/POCAndDevPlan.pdf?ref_type=heads}
{lecture slides}.}

\section{Confidential Information?}

\wss{State whether your project has confidential information from industry, or
not.  If there is confidential information, point to the agreement you have in
place.}

\wss{For most teams this section will just state that there is no confidential
information to protect.}
\section{IP to Protect}

\wss{State whether there is IP to protect.  If there is, point to the agreement.
All students who are working on a project that requires an IP agreement are also
required to sign the ``Intellectual Property Guide Acknowledgement.''}

\section{Copyright License}

Our Team is adopting an MIT License, which is a permissive open-source license. This license  allows users to freely use, copy, modify, merge, publish, distribute, sublicense, and even sell copies of the software, provided that the original copyright notice and the license text are included with all copies or substantial portions of the software. The license can be found in our github repository at the 

\begin{itemize}
    \item Link: \url{https://github.com/RezaJodeiri/CXR-Capstone/blob/main/LICENSE}
\end{itemize}

\section{Team Meeting Plan}

\wss{How often will you meet? where?}

\wss{If the meeting is a physical location (not virtual), out of an abundance of
caution for safety reasons you shouldn't put the location online}

\wss{How often will you meet with your industry advisor?  when?  where?}

\wss{Will meetings be virtual?  At least some meetings should likely be
in-person.}

\wss{How will the meetings be structured?  There should be a chair for all meetings.  There should be an agenda for all meetings.}


\section{Team Communication Plan}

Neuralanalyzers plans to hold in person and virutual meetings as a method of communication between the team. 

\subsection{Discord}
Discord will be used as the team's main mehtod of communication as its great at message logging, file sharing, creating threads for issues, real time communication and creating channels to separate information. Discord will be used by team members to update, and send key information or documentation links such as GitHub links, YouTube links, and code blocks to help keep the team up to date as well as ensure note taking is kept at a high standard. 
\subsection{GitHub}
GitHub is a great resource that Neuralanalyzers will be utilizing to create branchs and track code whilst ensuring clean merging of code between members. On github, issue requests can be created for the team to be notified and add input, to fix following issues as well as update the project as a whole. This will be the key center point where all the documentation and code will be saved. 

\section{Team Member Roles}

\begin{itemize}
\item Nathan Luong
    \begin{itemize}
        \item Scrum Master
        \item Developer
        \item Machine Learning Expert: Will do research on how to effectively read the X-Ray Image and create models that can be used to read race, age, and details of the patient. 
    \end{itemize}
    \item Ayman
    \begin{itemize}
        \item Developer
        \item Note Taker
        \item Computer Vision Expert: Will do research on how using Pytorch or Cuda could be utlized to ensure best performance of Application. 
    \end{itemize}
\item Patrick Zhou
    \begin{itemize}
        \item Developer
        \item Reviewer
        \item Python Expert: Will be in charge of Documentation of Python code Labeling each Function as well as, keeping the programming style consistent. Will be able to transform our pseudocode  into Python code. 
    \end{itemize}
        
\item Kelly Deng
    \begin{itemize}
        \item Developer
        \item Meeting Chair
        \item Machine Learning Expert: Will do research on how to effectively read the X-Ray Image and create models that can be used to read race, age, and details of the patient. 
    \end{itemize}
\item Reza Jodeiri
    \begin{itemize}
        \item Developer
        \item Leader 
        \item Chest X-ray Expert: Will lead the research and guide the team towards achieving the project’s objectives through their deep understanding of chest X-rays. Their expertise will be instrumental in ensuring that the X-ray images are accurately interpreted, identifying key anomalies and common disease patterns. This knowledge will directly contribute to training the machine learning model by selecting the most relevant features and markers for disease detection, such as abnormalities in lung structure, nodules, or lesions. 
    \end{itemize}
\end{itemize}
  
\section{Workflow Plan}

\begin{itemize}
	\item How will you be using git, including branches, pull request, etc.?
	\item How will you be managing issues, including template issues, issue
	classification, etc.?
  \item Use of CI/CD
\end{itemize}

\section{Project Decomposition and Scheduling}

\wss{How will the project be scheduled?  This is the big picture schedule, not
details. You will need to reproduce information that is in the course outline
for deadlines.}

\begin{itemize}
    \item How will you be using GitHub projects? 
    
    \begin{itemize}
        \item We will be using GitHub projects to keep track of the progress of the project. We will create a project board with columns such as To Do, In Progress, and Done. Each task will be represented as an issue on the project board. The project board will be updated regularly to reflect the current status of the project.
    \end{itemize}
    \item Include a link to your GitHub project 
    \begin{itemize}
        \item \url{https://github.com/RezaJodeiri/CXR-Capstone}
    \end{itemize}
\end{itemize}



\section{Proof of Concept Demonstration Plan}

What is the main risk, or risks, for the success of your project?  What will you
demonstrate during your proof of concept demonstration to convince yourself that
you will be able to overcome this risk?

\section{Expected Technology}

\wss{What programming language or languages do you expect to use?  What external
libraries?  What frameworks?  What technologies.  Are there major components of
the implementation that you expect you will implement, despite the existence of
libraries that provide the required functionality.  For projects with machine
learning, will you use pre-trained models, or be training your own model?  }

\wss{The implementation decisions can, and likely will, change over the course
of the project.  The initial documentation should be written in an abstract way;
it should be agnostic of the implementation choices, unless the implementation
choices are project constraints.  However, recording our initial thoughts on
implementation helps understand the challenge level and feasibility of a
project.  It may also help with early identification of areas where project
members will need to augment their training.}

Topics to discuss include the following:

\begin{itemize}
\item Specific programming language
\item Specific libraries
\item Pre-trained models
\item Specific linter tool (if appropriate)
\item Specific unit testing framework
\item Investigation of code coverage measuring tools
\item Specific plans for Continuous Integration (CI), or an explanation that CI
  is not being done
\item Specific performance measuring tools (like Valgrind), if
  appropriate
\item Tools you will likely be using?
\end{itemize}

\wss{git, GitHub and GitHub projects should be part of your technology.}

\section{Coding Standard}

\wss{What coding standard will you adopt?}

\newpage{}

\section*{Appendix --- Reflection}

\wss{Not required for CAS 741}


\begin{enumerate}
    \item Why is it important to create a development plan prior to starting the
    project?
    \begin{itemize}
        \item It is crucial to create a development plan prior to starting the project as it aims to ensure that the project is completed on time and within budget. The development plan outlines the tasks that need to be completed, the resources that are required, and the timeline for completion. By creating a development plan, the team can identify potential risks and issues early on and develop strategies to mitigate them. The development plan also helps to ensure that all team members are on the same page and working towards the same goals.
    \end{itemize}
    \item In your opinion, what are the advantages and disadvantages of using
    CI/CD? \\ \\
    % \begin{itemize}
        Advantages of using CI/CD include:
        \begin{itemize}
        \item Improved code quality: CI/CD runs automated tests on the code to identify bugs and issues early on, which can help to improve the overall quality of the code.
        \item Faster Development: By using CI/CD we can encourage/create faster development cycles that automate the process of building, testing, and deploying code, which can help to speed up the development process.
        \item Reduced Human Error: By making the deployment process automated, the chance manual errors when deploying is significantly reduced, ensuring reliability and consistency in releases.
        \end{itemize}
    % \end{itemize}
    \item What disagreements did your group have in this deliverable, if any,
    and how did you resolve them?
    \begin{itemize}
        \item Our group did not have any disagreements in this deliverable.
    \end{itemize}
\end{enumerate}

\newpage{}

\section*{Appendix --- Team Charter}

\wss{borrows from
\href{https://engineering.up.edu/industry_partnerships/files/team-charter.pdf}
{University of Portland Team Charter}}

\subsection*{External Goals}

\wss{What are your team's external goals for this project? These are not the
goals related to the functionality or quality fo the project.  These are the
goals on what the team wishes to achieve with the project.  Potential goals are
to win a prize at the Capstone EXPO, or to have something to talk about in
interviews, or to get an A+, etc.}

\subsection*{Attendance}

\subsubsection*{Expectations}

\wss{What are your team's expectations regarding meeting attendance (being on
time, leaving early, missing meetings, etc.)?}

\subsubsection*{Acceptable Excuse}

\wss{What constitutes an acceptable excuse for missing a meeting or a deadline?
What types of excuses will not be considered acceptable?}

\subsubsection*{In Case of Emergency}

\wss{What process will team members follow if they have an emergency and cannot
attend a team meeting or complete their individual work promised for a team
deliverable?}

\subsection*{Accountability and Teamwork}

\subsubsection*{Quality} 

\wss{What are your team's expectations regarding the quality
of team members' preparation for team meetings and the quality of the
deliverables that members bring to the team?}

\subsubsection*{Attitude}

\wss{What are your team's expectations regarding team members' ideas,
interactions with the team, cooperation, attitudes, and anything else regarding
team member contributions?  Do you want to introduce a code of conduct?  Do you
want a conflict resolution plan?  Can adopt existing codes of conduct.}

\subsubsection*{Stay on Track}

\wss{What methods will be used to keep the team on track? How will your team
ensure that members contribute as expected to the team and that the team
performs as expected? How will your team reward members who do well and manage
members whose performance is below expectations?  What are the consequences for
someone not contributing their fair share?}

\wss{You may wish to use the project management metrics collected for the TA and
instructor for this.}

\wss{You can set target metrics for attendance, commits, etc.  What are the
consequences if someone doesn't hit their targets?  Do they need to bring the
coffee to the next team meeting?  Does the team need to make an appointment with
their TA, or the instructor?  Are there incentives for reaching targets early?}

\subsubsection*{Team Building}

\wss{How will you build team cohesion (fun time, group rituals, etc.)? }

\subsubsection*{Decision Making} 

\wss{How will you make decisions in your group? Consensus?  Vote? How will you
handle disagreements? }

\end{document}