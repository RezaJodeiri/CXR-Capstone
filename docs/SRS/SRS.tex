% THIS DOCUMENT IS TAILORED TO REQUIREMENTS FOR SCIENTIFIC COMPUTING.  IT SHOULDN'T
% BE USED FOR NON-SCIENTIFIC COMPUTING PROJECTS
\documentclass[12pt]{article}

\usepackage{amsmath, mathtools}
\usepackage{amsfonts}
\usepackage{amssymb}
\usepackage{graphicx}
\usepackage{colortbl}
\usepackage{xr}
\usepackage{hyperref}
\usepackage{longtable}
\usepackage{xfrac}
\usepackage{tabularx}
\usepackage{float}
\usepackage{siunitx}
\usepackage{booktabs}
\usepackage{caption}
\usepackage{pdflscape}
\usepackage{afterpage}
\usepackage[table,xcdraw]{xcolor}
\usepackage{geometry}
\usepackage{array}
\usepackage{booktabs}
\usepackage{longtable}
\usepackage{enumitem}
\usepackage{graphicx}
\graphicspath{ {../assets/} }

\usepackage[round]{natbib}

%\usepackage{refcheck}

\hypersetup{
    bookmarks=true,         % show bookmarks bar?
      colorlinks=true,       % false: boxed links; true: colored links
    linkcolor=red,          % color of internal links (change box color with linkbordercolor)
    citecolor=green,        % color of links to bibliography
    filecolor=magenta,      % color of file links
    urlcolor=cyan           % color of external links
}

\input{../Comments}
%% Common Parts

\newcommand{\progname}{CXR} % PUT YOUR PROGRAM NAME HERE
\newcommand{\authname}{Team 27, Neuralyzers
\\ Ayman Akhras 
\\ Nathan Luong
\\ Patrick Zhou
\\ Kelly Deng
\\ Reza Jodeiri} % AUTHOR NAMES                  

\usepackage{hyperref}
    \hypersetup{colorlinks=true, linkcolor=blue, citecolor=blue, filecolor=blue,
                urlcolor=blue, unicode=false}
    \urlstyle{same}
                                


% For easy change of table widths
\newcommand{\colZwidth}{1.0\textwidth}
\newcommand{\colAwidth}{0.13\textwidth}
\newcommand{\colBwidth}{0.82\textwidth}
\newcommand{\colCwidth}{0.1\textwidth}
\newcommand{\colDwidth}{0.05\textwidth}
\newcommand{\colEwidth}{0.8\textwidth}
\newcommand{\colFwidth}{0.17\textwidth}
\newcommand{\colGwidth}{0.5\textwidth}
\newcommand{\colHwidth}{0.28\textwidth}

% Used so that cross-references have a meaningful prefix
\newcounter{defnum} %Definition Number
\newcommand{\dthedefnum}{GD\thedefnum}
\newcommand{\dref}[1]{GD\ref{#1}}
\newcounter{datadefnum} %Datadefinition Number
\newcommand{\ddthedatadefnum}{DD\thedatadefnum}
\newcommand{\ddref}[1]{DD\ref{#1}}
\newcounter{theorynum} %Theory Number
\newcommand{\tthetheorynum}{TM\thetheorynum}
\newcommand{\tref}[1]{TM\ref{#1}}
\newcounter{tablenum} %Table Number
\newcommand{\tbthetablenum}{TB\thetablenum}
\newcommand{\tbref}[1]{TB\ref{#1}}
\newcounter{assumpnum} %Assumption Number
\newcommand{\atheassumpnum}{A\theassumpnum}
\newcommand{\aref}[1]{A\ref{#1}}
\newcounter{goalnum} %Goal Number
\newcommand{\gthegoalnum}{GS\thegoalnum}
\newcommand{\gsref}[1]{GS\ref{#1}}
\newcounter{instnum} %Instance Number
\newcommand{\itheinstnum}{IM\theinstnum}
\newcommand{\iref}[1]{IM\ref{#1}}
\newcounter{reqnum} %Requirement Number
\newcommand{\rthereqnum}{R\thereqnum}
\newcommand{\rref}[1]{R\ref{#1}}
\newcounter{nfrnum} %NFR Number
\newcommand{\rthenfrnum}{NFR\thenfrnum}
\newcommand{\nfrref}[1]{NFR\ref{#1}}
\newcounter{lcnum} %Likely change number
\newcommand{\lthelcnum}{LC\thelcnum}
\newcommand{\lcref}[1]{LC\ref{#1}}

\usepackage{fullpage}

\newcommand{\deftheory}[9][Not Applicable]
{
\newpage
\noindent \rule{\textwidth}{0.5mm}

\paragraph{RefName: } \textbf{#2} \phantomsection 
\label{#2}

\paragraph{Label:} #3

\noindent \rule{\textwidth}{0.5mm}

\paragraph{Equation:}

#4

\paragraph{Description:}

#5

\paragraph{Notes:}

#6

\paragraph{Source:}

#7

\paragraph{Ref.\ By:}

#8

\paragraph{Preconditions for \hyperref[#2]{#2}:}
\label{#2_precond}

#9

\paragraph{Derivation for \hyperref[#2]{#2}:}
\label{#2_deriv}

#1

\noindent \rule{\textwidth}{0.5mm}

}

\begin{document}

\title{Software Requirements Specification for \progname: subtitle describing software} 
\author{\authname}
\date{\today}
	
\maketitle

~\newpage

\pagenumbering{roman}

\tableofcontents

~\newpage

\section*{Revision History}

\begin{tabularx}{\textwidth}{p{3cm}p{2cm}X}
\toprule {\bf Date} & {\bf Version} & {\bf Notes}\\
\midrule
Date 1 & 1.0 & Notes\\
Date 2 & 1.1 & Notes\\
\bottomrule
\end{tabularx}

~\\
\plt{This template is intended for use by CAS 741.  For CAS 741 the template
should be used exactly as given, except the Reflection Appendix can be deleted.
For the capstone course it is a source of ideas, but shouldn't be followed
exactly.  The exception is the reflection appendix.  All capstone SRS documents
should have a refelection appendix.}

~\newpage

\section{Reference Material}

This section records information for easy reference.

\subsection{Table of Units}

Throughout this document SI (Syst\`{e}me International d'Unit\'{e}s) is employed
as the unit system.  In addition to the basic units, several derived units are
used as described below.  For each unit, the symbol is given followed by a
description of the unit and the SI name.
~\newline

\renewcommand{\arraystretch}{1.2}
%\begin{table}[ht]
  \noindent \begin{tabular}{l l l} 
    \toprule		
    \textbf{symbol} & \textbf{unit} & \textbf{SI}\\
    \midrule 
    \si{\second} & Time  & second\\
    \si{min} & Time  & minute\\
    \si{imgs} & Image & Imagines\\
    \si{px} & Image  & Pixel\\
    \si{GB} & Data 	& Gigabyte\\
    \si{MB} & Data  & Megabyte\\
    \si{LOC} & Quantity & Lines of Code\\
    \bottomrule
  \end{tabular}
  %	\caption{Provide a caption}
%\end{table}
\\
\subsection{Definitions}
\renewcommand{\arraystretch}{1.2}
  \noindent \begin{tabular}{l l} 
    \toprule		
    \textbf{Terminology} & \textbf{Definition}\\
    \midrule 
    \si{\second} & Time\\
    \si{Containerized Application} &  portable version of an application that can be run
on a container run-time, such as Docker.\\
    \si{Amazon Web Services} & A public cloud provider, offering all HIPAA-
compliance cloud services that helps Neuralanalyzer host, manage, and scale our ap-
plication.\\
    \si{AWS ECS} & AWS Elastic Cloud Service: - An AWS managed service for managing
and maintaining application containers at run-time.\\
    \si{AWS ECR} & AWS Elastic Container Registry - An AWS managed service for storing
and managing container images.\\
    \si{AWS Fargate} & An AWS managed service for running containerized applications.\\
    \si{AWS Cognito} & An AWS managed service for authentication logic, handling user and
password management\\
    \si{React} & A web front-end framework, written in Javascript.\\
    \si{Flask} & An HTTP-based server framework, written in Python.\\
    \si{TorchXRAY Vision} & An open-source classifier model that can classify diseases based
on chest X-Ray images\\
    \si{SLA} &  Shorten for Service-Level Agreement. A percentage metrics to signify the overall
up-time of a system. For example, an SLA of 99\% means the system must be up 99\%
of it’s operational time.\\
    \bottomrule
  \end{tabular}
  
\subsection{Abbreviations and Acronyms}
\renewcommand{\arraystretch}{1.3}
\begin{tabular}{l l} 
  \toprule		
  \textbf{symbol} & \textbf{description}\\
  \midrule 
  CAS & Computing and Software\\
  SRS & Software Requirements Specification\\
  AI & Artificial Intelligence\\
  ML & Machine Laerning\\
  FR & Functional Requirement\\
  NFR & Non Functional Requirement\\
  FSM & Finite State Machine\\
  CXR & Chest X-Ray Project\\
  POC & Proof of concept\\
  TM & Theoretical Model\\
  AWS & Amazon Web Serivices\\
  UI/UX & User Interface and User Experience\\
  OS & Operating System\\
  CT & Computed Tomography \\
  HIPAA & Health Insurance Portability and Accountability Act\\
  PIPEDA & Personal Information Protection and Electronic Documents Act\\
  PHI & Personal Health Information\\
  MVP & Minimum viable product\\
  IOS & International Organization for Standardization\\
  MRI & Magnetic resonance imaging\\
  EHR & electronic health record\\
  AES & Advanced Encryption Standard\\
  PACS & picture archiving and communication system\\
  DICOM & Digital Imaging and Communications\\
  CAD & Canadian Dollars\\
  N/A & Not applicable, not available, not assessed, or no answer\\
  \bottomrule
\end{tabular}\\

\newpage
\section{Introduction}

Chest X-rays are one of the most common diagnostic imaging techniques used in clinical practice for evaluating respiratory and cardiac conditions. However, the interpretation of chest X-ray images can be complex and time-consuming that requires high-level expertise and experience. Misinterpretations can lead to wrong diagnoses and inappropriate treatment plans, increasing the healthcare costs.\\

As healthcare professionals are facing challenges in accurately and efficiently diagnosing abnormalities from chest X-ray images, there is a urgent need for a reliable, efficient and user-friendly AI/ML tool that helps them with interpreting chest X-ray images.\\

In this section, the document will go through the purpose of this SRS document, scope of requirements, characteristics of intended readers and organization of this SRS.\\



\subsection{Purpose of Document}

The purpose of this SRS document is to provide a foundation that guides the design, development and implementation of the software. The primary purposes of the SRS document includes providing a shared understanding of the software requirements between stakeholders and developers. It defines the scope of the software project and details the what functionalities will be included and what will not be included. It serves as a comprehensive guide for developers and outlines the system constraints, properties of a correct solution, functional and non-functional requirements. It also states the potential changes to the software project, and provides a structured approach to manage changes in the software project.\\

\subsection{Scope of Requirements} 

The scope of this project is to take in a chest X-ray image and output if any diseases / abnormalities are detected. In the training process, labels need to include both disease categories and areas of abnormality. The datasets also should represent diverse patient demographics (age, gender etc) and include different scenarios including rare diseases. Input X-rays also should be of high resolution and captured with standard machines. The project is not designed for reading all types of medical imaging and is only specialized for chest X-rays. The project is not meant to fully automate X-ray reading but only to assist with specific tasks like identifying diseases / abnormalities and supporting decision-making. Certain rare / high complex diseases may fall outside the system's training data which lead to limitations in detection. The system does not account for patient history, or other test results that are crucial for diagnosis. The system cannot determine what cause the diseases / abnormalities or predict long-term outcomes. \\

\subsection{Characteristics of Intended Reader} \label{sec_IntendedReader}

The reader of this SRS document should possess undergraduate level of software development skill as it specifies the system constraints, functional and non-functional requirements. Reader should understand the theories of software development as this SRS contains sections about theoretical models and instance models. Readers should also have some basic experience with project management to understand the development plan and the traceability matrix. Ideally, user should have some fundamental understanding of Artificial Intelligence but this is not a required characteristic as the SRS focuses more on the requirements instead of the technical side. To understand input data types and other constraints on data, reader should have at least 1 year programming experiences.\\

\subsection{Stakeholders}
1. Healthcare Providers: they are the primary uses of the AI system, as the software will help them in interpreting chest X-rays more efficiently by detecting abnormalities and analyzing potential diseases. The use of this software as part of the diagnostic process will help improve efficiency and accuracy.
\\
\\
2. Patients: they benefit from faster, more accurate diagnoses with a improved healthcare accessibility. They also can get early detection of life-threatening diseases.\\
\\
3. Hospitals and Clinics: Hospitals and clinics can integrate the software into their hospital systems, which can help them to improve efficiency, reduce diagnostic errors, manage costs.\\
\\
4. Technical Support team: they are responsible for ensuring seamless connectivity with existing systems, maintaining workflow, and ensuring data security and privacy. \\

\section{General System Description}

This section provides general information about the system.  It identifies the
interfaces between the system and its environment, describes the user
characteristics and lists the system constraints. \\

\subsection{System Context}

The below system context diagram shows that the system will take in an input (chest X-ray image) from the user after they login. The user upload the chest X-ray image to the system and then system will annotate and provide the interpretation of the X-ray image back to the user.



\begin{itemize}
\item User Responsibilities:
\begin{itemize}
\item {Need to ensure that the chest X-ray image is of high resolution. }
\item {Need to ensure that the chest X-ray image is a chest X-ray image, not other types of medical images.}
\item {Need to ensure that the chest X-ray image taken by standard machines. Images with poor lighting, noises or does not follow standard protocols will not be accepted. }
\end{itemize}
\item Program Responsibilities:
\begin{itemize}
\item {Detect abnormalities / diseases in the chest X-ray images.}
\item {Classify abnormalities / diseases into specific categories.}
\item {Quantify size / degree / volume of abnormalities.}
\item {Produce report with possiblee diagnosis and the above information.}
\end{itemize}
\end{itemize}

The software is typically used in medical and clinical contexts to support healthcare professionals and physicians to speed up X-ray readings and improve accuracy in diagnosis. It is a safety critical application as failure could lead to delay in treatment, misdiagnosis and potentially severe outcomes for the patient.


\subsection{User Characteristics} \label{SecUserCharacteristics}

The user of this software application will primarily be patients and physicians. The end user should be able to read / understand middle school level English. The end user should also have a basic understanding of chest-related diseases, advanced knowledge is preferred but not required. Users can consult professionals if they have trouble interpreting the result. The user should also know to to upload X-ray images via web app from local machine and have basic knowledge of how to use computer / laptop. \\

\subsection{System Constraints}

1. Computational resources: Limited processing power and memory might affect the speed and efficiency of the AI algorithms.\\
\\
2. Data quality: Biased or blurry data can lead to inaccurate results.\\
\\
3. Compatibility: The software will only be supported on Windows 10 or higher and macOS 12 or higher. \\
\\
4. Ethical considerations: If the training data lacks diversity, the AI may perform poorly for certain populations. \\
\\
5. Regulatory constraints: Delays in obtaining necessary approvals can hinder deployment and development stages.\\

\subsection{Use Case Diagram}
\begin{figure}
    \centering
    \includegraphics[width=1\linewidth]{UseCase.png}
    \caption{Use Case Diagram of AI for chest X-ray read}
    \label{fig:enter-label}
\end{figure}

\section{Specific System Description}
This section first presents the problem description, which gives a high-level
view of the problem to be solved.  This is followed by the solution characteristics
specification, which presents the assumptions, theories, definitions and finally
the instance models.

\subsection{Problem Description}
Disease prediction via medical imaging continues to be one of the most important aspect of health-care. However, the health-care industry is still slowly catching up with current technology, making disease prediction and prevention processes labour-intensive and inefficient. With a prolong history in operation, countless images has been captured from CT Scanners and X-Ray machines, for medical and health-care research, but yet to be applied in hospital settings between physicians and patients.

Our team, Neuralanalyzer, saw an opportunity to transform traditional health-care processes, by applying Machine Learning, and modern software engineering techniques; aiming to provide an optimized, unified, and HIPAA-compliant platform to predict diseases from chest X-Ray scans.

\subsubsection{Goal Statements}
\noindent Given the inputs and the constraints of the system, the goal statements are:

\begin{itemize}

\item[GS\refstepcounter{goalnum}\thegoalnum \label{G_prediction_accuracy}:] \textbf{High Prediction Accuracy: }Machine Learning Model used by Neuralanalyzer must have a high prediction accuracy for Chest X-Ray images. We are aiming for a minimum of 75\% accuracy.
\item[GS\refstepcounter{goalnum}\thegoalnum \label{G_ease_of_ML_intro}:] \textbf{Simple to introduce more ML models: }New classifier models must be simple and intuitive to be integrated with the system, with short time-to-market. This goal will require a containerized ML-plugin code template (in Python) and guidelines documentations for new models to follow.
\item[GS\refstepcounter{goalnum}\thegoalnum \label{G_fault-tolerance}:] \textbf{High Fault-Tolerance: }Neuralanalyzer's system (front-end, back-end, ML-plugins) must handle run-time errors as clever as possible, to prevents run-time crashes, and unwanted downtime. Comprehensive unit and integration testings must be present, to reduce errors and bugs within the system.
\item[GS\refstepcounter{goalnum}\thegoalnum \label{G_good_UI}:] \textbf{Intuitive UI/UX: }Neuralanalyzer's front-end must be user-friendly and straight-forward to be used. Front-end's core features such as login, account registration, and image-prediction should require as few clicks as possible.
\item[GS\refstepcounter{goalnum}\thegoalnum \label{G_dev_work_automation}:] \textbf{High automation of Dev-workflow: }Repetitive, and crucial development workflows must be implemented within Github Actions. Example including: Unit \& Integration Testing, Linting Check, Docker Image build, Deployment on AWS.
\item[GS\refstepcounter{goalnum}\thegoalnum \label{G_avaialblity}:] \textbf{High Availability: }Neuralanalyzer application must be highly available, with very minimal downtime. Our team aiming to have our targeted SLA as 99\%.
\item[GS\refstepcounter{goalnum}\thegoalnum \label{G_data_security}:] \textbf{Secure Data Transfer \& Hosting: }Image-related operations, such as storing, transferring via front-end, ML-training must be authenticated, authorized, and encrypted. This requires an introduction of a authentication service, which our team will be using AWS Cognito.

\end{itemize}

\subsection{Solution Characteristics Specification}
\subsubsection{Architectural Decisions}
\begin{itemize}
\item A. Splitting up the AI-Models from the core Neuralanalyzer back-end service.
    \begin{itemize}
        \item Benefits:
        \begin{itemize}
            \item Allowing separate development and deployment of AI-Models, and other application logic (authentication, databases, etc.)
            \item Enabling the ability to connect multiple models, via plugins, onto the system, providing Neuralanalyzer with more flexibility.
        \end{itemize}
        \item Drawbacks:
        \begin{itemize}
            \item Increase complexity in managing source code, and at run-time. For example: runtime security, availability, service discovery, etc.
        \end{itemize}
    \end{itemize}
    
\item B. Training Neuralanalyzer ML Model on CAS Department's servers.
    \begin{itemize}
        \item Benefits:
        \begin{itemize}
            \item Massively speeds up research and development since department's servers are already set-up with student's MacID.
            \item Reducing cost for AI-Model training.
        \end{itemize}
        \item Drawbacks:
        \begin{itemize}
            \item This separate training process makes this particular development workflow hard and time-consuming to automate.
        \end{itemize}
    \end{itemize}
    
\item C. Including TorchXRayVision as a back-up model.
    \begin{itemize}
        \item Benefits:
        \begin{itemize}
            \item Ability to quickly create a working and stable MVP, which contains the front-end, back-end, and the ML model.
            \item Ability to apply transfer-learning on to this model to fine-tune its accuracy.
            \item Ability to change the source code of this model to fit's our team's need.
        \end{itemize}
        \item Drawbacks:
        \begin{itemize}
            \item Un-ability to re-train the model.
            \item Out-of-the-box model has low accuraccy score.
        \end{itemize}
    \end{itemize}
\end{itemize}
\subsubsection{Assumptions} \label{sec_assumpt}
\begin{itemize}

\item[A\refstepcounter{assumpnum}\theassumpnum \label{A_meaningfulLabel}:] Machine Learning Models Performance will be tracked via a standardized method, and this method will be publicly published after being developed. Reason being the accuracy of the ML model is vital for the end-user (physicians, patients), therefore this validation method needs transparency to the stakeholders, and users.
\item[A\refstepcounter{assumpnum}\theassumpnum \label{A_meaningfulLabel}:] All health-related data that are not publicly available, needs to be handled in a HIPPA-compliance manner. Reason being private PHI are sensitive are are protected under this act.
\item[A\refstepcounter{assumpnum}\theassumpnum \label{A_meaningfulLabel}:] McMaster Department's servers will be available for our group members to perform model training and development. Under another assumption that our group members, as users of department resources must follows department's guidelines.

\end{itemize}

\subsubsection{Input Data Constraints} \label{sec_DataConstraints} 
From the system perspective, the input data can be any image, of any size, in any format. Our system will only output the correct label for images of chest x-ray scan. Example below.
\begin{center}
    \includegraphics[scale=0.3]{chest-x-ray.png}
\end{center}

\subsubsection{Properties of a Correct Solution} \label{sec_CorrectSolution}
In the context of our machine learning model, the absolute correct solution can not be calculated at run-time. However, A solution is considered good enough when during in the training and validation process,  the error rate of the system has been reduced to an agreeable number.

\section{Required Behaviour Overview}
Neuralanalyzer's system is composed of 3 main sub-systems:
\begin{itemize}
    \item \textbf{Neuralanalyzer Front-end}: Responsible for handling user's interaction via a web-based UI. The front-end will be responsible for core functionalities such as login, account registration, image uploading.
    \item \textbf{Neuralanalyzer Back-end}: Responsible for handling front-end calls, authentication logic, and ML-plugins interactions. The back-end will follow a stateless design pattern, optimized for horizontal scalling.
    \item \textbf{Neuralanalyzer ML Plugins}: Responsible for taking in images, and returns a prediction vector with high accuracy and high-throughput. The ML-plugins will also follow a stateless design pattern, optimized for horizontal scalling.
\end{itemize}

\section{Requirements}
\subsection{Formal Mathematical Specification}

\subsubsection{Formal Definitions}
\begin{itemize}
  \item Let \( X = \{x_1, x_2, \ldots, x_n\} \) be the set of chest X-ray images, where \( x_i \) represents each individual image.
  \item Let \( Y = \{y_1, y_2, \ldots, y_n\} \) be the set of labels, where \( y_i \in \{0, 1\} \). Here, \( y_i = 1 \) indicates the presence of a disease, and \( y_i = 0 \) indicates its absence.
  \item Let \( M \) be the machine learning model used for disease detection.
  \item Let \( \theta \) represent the parameters of model \( M \), which are learned during training.
  \item Let \( P(x_i) \) represent the probability that an image \( x_i \) is classified as containing a disease by the model \( M \).
\end{itemize}

\subsubsection{Formal Expressions}
\begin{enumerate}
  \item \textbf{Training the Model:}
    \begin{align*}
      \theta^* &= \text{arg min}_\theta \sum_{i=1}^n (M(x_i; \theta) - y_i)^2 \\
      &\text{where } M(x_i; \theta) \text{ is the predicted value for } x_i.
    \end{align*}
    The model parameters \( \theta \) are adjusted during training to minimize the difference between the predicted values and the actual labels.

  \item \textbf{Prediction:}
    \begin{align*}
      \hat{y}_i &= \begin{cases}
        1 & \text{if } P(x_i) \geq 0.5 \\
        0 & \text{if } P(x_i) < 0.5
      \end{cases}
    \end{align*}
    The model predicts \( \hat{y}_i = 1 \) (disease present) if the probability \( P(x_i) \) is greater than or equal to 0.5, and \( \hat{y}_i = 0 \) (disease absent) otherwise.

  \item \textbf{Evaluation:}
    \begin{align*}
      \text{Accuracy} &= \frac{\sum_{i=1}^n \mathbb{I}(\hat{y}_i = y_i)}{n} \\
      &\text{where } \mathbb{I}(\hat{y}_i = y_i) \text{ equals 1 if the prediction is correct, otherwise 0.}
    \end{align*}
    Accuracy measures the proportion of images for which the model's predictions match the actual labels.

  \item \textbf{Loss Function:}
    \begin{align*}
      L(\theta) &= \sum_{i=1}^n (M(x_i; \theta) - y_i)^2
    \end{align*}
    The loss function measures how far the model's predictions are from the actual labels. The model aims to minimize this function during training.
\end{enumerate}

\subsubsection{Model Constraints}
\begin{itemize}
  \item The model \( M \) should achieve an accuracy of at least 85\% on the validation dataset.
  \item Each input image \( x_i \) should be preprocessed to a fixed size (e.g., 224x224 pixels) to ensure compatibility with the model.
  \item The probability threshold for classification is set at 0.5, but it can be adjusted based on the desired sensitivity of the model.
\end{itemize}
\subsection{Finite State Machine Diagram}
\newpage
\subsection{Functional Requirements}
\begin{table}[h!]
\centering
\rowcolors{2}{white}{white} % All rows except the first will be white
\begin{tabular}{|p{3.5cm}|p{11.5cm}|}
\hline
\rowcolor{gray!30} % Light gray for the first row (header)
\textbf{FR1} & The system shall accept chest X-ray images as input from authorized users, including healthcare professionals and patients. \\
\hline
\textbf{Rationale} & To perform analysis, the system requires chest X-ray images to be uploaded by users in supported formats (DICOM, JPEG, PNG). \\
\hline
\textbf{Verification} & Test the upload functionality with different image formats and ensure the system can process the images without errors. \\
\hline
\textbf{Priority} & High \\
\hline
\textbf{Traceability} & LF1, LF2, UH1, FR2 (for input handling), FR7 (data storage) \\
\hline
\end{tabular}
\caption{Functional Requirement FR1}
\centering
\rowcolors{2}{white}{white}
\begin{tabular}{|p{3.5cm}|p{11.5cm}|}
\hline
\rowcolor{gray!30}
\textbf{FR2} & The system shall enable users to input additional patient symptoms, such as cough, chest pain, or fever. \\
\hline
\textbf{Rationale} & Including patient symptoms helps provide more context for the AI model to enhance disease detection accuracy. Our goal is to make the AI more useful by giving it more information to work with. \\
\hline
\textbf{Verification} & Verify that users can enter patient symptoms and that they are correctly linked to the associated X-ray images. \\
\hline
\textbf{Priority} & Medium \\
\hline
\textbf{Traceability} & UH1, UH2, MS1, FR1 (input dependency), FR6 (reporting of symptoms) \\
\hline
\end{tabular}
\caption{Functional Requirement FR2}
\end{table}
\begin{table}[h!]
\centering
\rowcolors{2}{white}{white}
\begin{tabular}{|p{3.5cm}|p{11.5cm}|}
\hline
\rowcolor{gray!30}
\textbf{FR3} & The system shall analyze chest X-ray images to detect the presence or absence of specific diseases with an accuracy of 85\% or higher. \\
\hline
\textbf{Rationale} & One of the core functionalities of our system is its ability to detect diseases like pneumonia from X-rays. This ensures the AI model is effective in clinical settings. \\
\hline
\textbf{Verification} & Evaluate the accuracy of the disease detection model against a test dataset and confirm it meets or exceeds the 85\% accuracy threshold. \\
\hline
\textbf{Priority} & High \\
\hline
\textbf{Traceability} & PR1, HS1, SR1, FR1 (input of images), FR6 (reporting) \\
\hline
\end{tabular}
\caption{Functional Requirement FR3}
\end{table}
\begin{table}[h!]
\centering
\rowcolors{2}{white}{white}
\begin{tabular}{|p{3.5cm}|p{11.5cm}|}
\hline
\rowcolor{gray!30}
\textbf{FR4} & The system shall indicate whether a patient's condition has improved, worsened, or remained stable between scans. \\
\hline
\textbf{Rationale} & This feature helps clinicians monitor disease progression or regression, which is vital for evaluating the effectiveness of treatments over time.\\
\hline
\textbf{Verification} & Verify that the system provides a status of the patient's condition after comparing multiple scans. \\
\hline
\textbf{Priority} & High \\
\hline
\textbf{Traceability} & PR1, OE2, HS1, FR3 (analysis results), FR8 (alerts) \\
\hline
\end{tabular}
\caption{Functional Requirement FR4}
\end{table}
\begin{table}[h!]
\centering
\rowcolors{2}{white}{white}
\begin{tabular}{|p{3.5cm}|p{11.5cm}|}
\hline
\rowcolor{gray!30}
\textbf{FR5} & The system shall generate visual aids by highlighting affected areas on the chest X-ray images. \\
\hline
\textbf{Rationale} & By highlighting affected regions, our system makes it easier for clinicians to identify the areas of concern in a chest X-ray. This improves their ability to act quickly. \\
\hline
\textbf{Verification} & Ensure that the system overlays the appropriate markers on the X-ray images to highlight abnormal areas. \\
\hline
\textbf{Priority} & Medium\\
\hline
\textbf{Traceability} & PR1, PR3, LF1, FR3 (analysis results), FR6 (report visualization) \\
\hline
\end{tabular}
\caption{Functional Requirement FR5}
\end{table}
\begin{table}[h!]
\centering
\rowcolors{2}{white}{white}
\begin{tabular}{|p{3.5cm}|p{11.5cm}|}
\hline
\rowcolor{gray!30}
\textbf{FR6} & The system shall produce a structured, human-readable report summarizing key findings, disease detection results, and progression status. \\
\hline
\textbf{Rationale} & A comprehensive report makes it easier for clinicians to interpret the analysis results, especially in fast-paced environments where time is critical. \\
\hline
\textbf{Verification} & Verify that the system generates a report containing all necessary details in an organized format. \\
\hline
\textbf{Priority} & High \\
\hline
\textbf{Traceability} &  LF1, UH2, HS1, FR3 (analysis), FR4 (condition status). \\
\hline
\end{tabular}
\caption{Functional Requirement FR6}
\end{table}
\begin{table}[h!]
\centering
\rowcolors{2}{white}{white}
\begin{tabular}{|p{3.5cm}|p{11.5cm}|}
\hline
\rowcolor{gray!30}
\textbf{FR7} & The system shall store patient data, including images and reports, in a secure database for future reference. \\
\hline
\textbf{Rationale} & Keeping patient records secure and accessible for future consultations or research is important for patient care continuity. \\
\hline
\textbf{Verification} & Verify that the system correctly stores and retrieves patient data from the database. \\
\hline
\textbf{Priority} & Medium \\
\hline
\textbf{Traceability} & PR2, PR3, SR1, LR2, FR1 (data input), FR6 (reporting) \\
\hline
\end{tabular}
\caption{Functional Requirement FR7}
\end{table}

\begin{table}[h!]
\centering
\rowcolors{2}{white}{white}
\begin{tabular}{|p{3.5cm}|p{11.5cm}|}
\hline
\rowcolor{gray!30}
\textbf{FR8} & The system shall provide alerts to clinicians if significant changes in a patient's condition are detected between scans. \\
\hline
\textbf{Rationale} & Automated alerts can save time and ensure that urgent cases are handled quickly, improving patient outcomes. \\
\hline
\textbf{Verification} & Test that alerts are generated when the system detects a significant change in disease severity or condition. \\
\hline
\textbf{Priority} & Low \\
\hline
\textbf{Traceability} & PR1, SR2, FR4 (condition changes), FR6 (reporting) \\
\hline
\end{tabular}
\caption{Functional Requirement FR8}
\end{table}
\begin{table}[h!]
  \centering
  \rowcolors{2}{white}{white}
  \begin{tabular}{|p{3.5cm}|p{11.5cm}|}
  \hline
  \rowcolor{gray!30}
  \textbf{FR9} & The system shall allow healthcare professionals to adjust treatment plans based on the X-ray analysis results. \\
  \hline
  \textbf{Rationale} & Allowing clinicians to modify treatment plans based on X-ray results helps to tailor patient care more effectively. This ensures that the system supports decision-making in real-time clinical settings. \\
  \hline
  \textbf{Verification} & Verify that healthcare professionals can modify treatment plans directly in the system, and that these changes are linked to the corresponding X-ray analysis results. \\
  \hline
  \textbf{Priority} & High \\
  \hline
  \textbf{Traceability} & MS1, SR2, FR6 (reporting adjustments), FR7 (data updates) \\
  \hline
  \end{tabular}
  \caption{Functional Requirement FR9}
  \end{table}

\begin{table}[h!]
\centering
\rowcolors{2}{white}{white}
\begin{tabular}{|p{3.5cm}|p{11.5cm}|}
\hline
\rowcolor{gray!30}
\textbf{FR10} & The system shall allow integration with electronic health record (EHR) systems to import and export patient data. \\
\hline
\textbf{Rationale} & Integrating with existing EHR systems improves workflow efficiency and reduces the need for manual data entry, making the system more useful in a clinical setting. \\
\hline
\textbf{Verification} & Test the system’s ability to connect with standard EHR systems and exchange data using appropriate protocols. \\
\hline
\textbf{Priority} & Low \\
\hline
\textbf{Traceability} & OE1, MS1, LR2, FR7 (data storage), FR9 (treatment adjustment) \\
\hline
\end{tabular}
\caption{Functional Requirement FR10}
\end{table}
\begin{table}[h!]
\centering
\rowcolors{2}{white}{white}
\begin{tabular}{|p{3.5cm}|p{11.5cm}|}
\hline
\rowcolor{gray!30}
\textbf{FR11} & The system shall display confidence levels for disease detection and progression analysis results through an intuitive user interface that requires minimal training to operate. \\
\hline
\textbf{Rationale} & Providing confidence levels helps clinicians assess the reliability of the results. \\
\hline
\textbf{Verification} & Check that the system displays a confidence score alongside each analysis result. \\
\hline
\textbf{Priority} & High \\
\hline
\textbf{Traceability} & UH2, PR1, FR3 (analysis results), FR5 (visual aids) \\
\hline
\end{tabular}
\caption{Functional Requirement FR11}
\end{table}

\begin{table}[h!]
\centering
\rowcolors{2}{white}{white}
\begin{tabular}{|p{3.5cm}|p{11.5cm}|}
\hline
\rowcolor{gray!30}
\textbf{FR12} & The system shall allow patients to upload their own chest X-ray images for self-diagnosis purposes, subject to appropriate disclaimers. \\
\hline
\textbf{Rationale} & Allowing patients to upload their own images increases engagement in their healthcare journey, though it must be made clear that professional interpretation is still necessary. \\
\hline
\textbf{Verification} & Ensure that patients can upload images and that the system displays appropriate disclaimers before analysis. \\
\hline
\textbf{Priority} & Low \\
\hline
\textbf{Traceability} & SR1, SR2, HS2, FR1 (image upload), FR13 (role management) \\
\hline
\end{tabular}
\caption{Functional Requirement FR12}
\end{table}

\begin{table}[h!]
\centering
\rowcolors{2}{white}{white}
\begin{tabular}{|p{3.5cm}|p{11.5cm}|}
\hline
\rowcolor{gray!30}
\textbf{FR13} & The system shall support multiple user roles with appropriate access levels (e.g., physician, radiologist, patient). \\
\hline
\textbf{Rationale} & Different users need access to different features and data, so assigning roles ensures that each user can only perform actions they are authorized to. \\
\hline
\textbf{Verification} & Verify that users are assigned roles and that their access is restricted to the appropriate features based on their role. \\
\hline
\textbf{Priority} & High \\
\hline
\textbf{Traceability} &  SR2, LR1, FR7 (data storage), FR12 (patient data access) \\
\hline
\end{tabular}
\caption{Functional Requirement FR13}
\end{table}
\begin{table}[h!]
\centering
\rowcolors{2}{white}{white}
\begin{tabular}{|p{3.5cm}|p{11.5cm}|}
\hline
\rowcolor{gray!30}
\textbf{FR14} & The system shall create a new copy of a patient’s X-ray before running the AI model for analysis. \\
\hline
\textbf{Rationale} & Creating a new copy ensures that the original X-ray remains unchanged for future reference, while the AI model works on the new version for tagging and analysis. \\
\hline
\textbf{Verification} & Verify that the system creates a new X-ray copy, preserving the original file while the AI model runs on the duplicate. \\
\hline
\textbf{Priority} & High \\
\hline
\textbf{Traceability} & MS2, OE1, FR1 (image upload), FR3 (analysis) \\
\hline
\end{tabular}
\caption{Functional Requirement FR14}
\end{table}
\begin{table}[h!]
\centering
\rowcolors{2}{white}{white}
\begin{tabular}{|p{3.5cm}|p{11.5cm}|}
\hline
\rowcolor{gray!30}
\textbf{FR15} & The system shall support additional medical imaging modalities, such as CT scans and MRIs, for comprehensive analysis. \\
\hline
\textbf{Rationale} & Supporting multiple imaging types expands the system's utility in various clinical scenarios. \\
\hline
\textbf{Verification} & Test the system's ability to accept and analyze different image types (e.g., CT, MRI) without errors and provide accurate results. \\
\hline
\textbf{Priority} & Low \\
\hline
\textbf{Traceability} & PR3, LR2, FR1 (image input), FR7 (data storage) \\
\hline
\end{tabular}
\caption{Functional Requirement FR15}
\end{table}

\begin{table}[h!]
\centering
\rowcolors{2}{white}{white}
\begin{tabular}{|p{3.5cm}|p{11.5cm}|}
\hline
\rowcolor{gray!30}
\textbf{FR16} & The system shall support regular updates to the AI model to incorporate new data and improve accuracy over time. \\
\hline
\textbf{Rationale} & Continuous learning enhances the model's performance and keeps it up-to-date with the latest medical knowledge. \\
\hline
\textbf{Verification} & Ensure that the system can accept and integrate updated models without disrupting service. \\
\hline
\textbf{Priority} & Medium \\
\hline
\textbf{Traceability} & MS1, MS2, FR3 (analysis), FR7 (data storage) \\
\hline
\end{tabular}
\caption{Functional Requirement FR16}
\end{table}  
\clearpage                                                                                                                                                    

\subsection{Non-Functional Requirements}

\subsubsection{Look and Feel Requirements}

\begin{table}[h!]
\centering
\rowcolors{2}{white}{white}
\begin{tabular}{|p{3.5cm}|p{11.5cm}|}
\hline
\rowcolor{gray!30}
\textbf{LF1} & The system shall have a user interface consistent with standard medical imaging software, using familiar layouts and terminology. \\
\hline
\textbf{Rationale} & Radiologists and healthcare professionals are accustomed to specific interface designs; maintaining consistency reduces the learning curve. \\
\hline
\textbf{Fit Criterion} & At least 90\% of radiologists surveyed find the interface intuitive and comparable to existing medical imaging software. \\
\hline
\textbf{Traceability} & FR1, FR5, FR6 \\
\hline
\end{tabular}
\caption{Non-functional Requirement LF1}
\end{table}

\begin{table}[h!]
\centering
\rowcolors{2}{white}{white}
\begin{tabular}{|p{3.5cm}|p{11.5cm}|}
\hline
\rowcolor{gray!30}
\textbf{LF2} & The system shall use color schemes and fonts that minimize eye strain during prolonged use. \\
\hline
\textbf{Rationale} & Medical professionals often work long hours; a comfortable visual interface helps prevent fatigue. \\
\hline
\textbf{Fit Criterion} & The interface adheres to ergonomic standards (e.g., ISO 9241-3) for visual displays. \\
\hline
\textbf{Traceability} & FR1, FR5 \\
\hline
\end{tabular}
\caption{Non-functional Requirement LF2}
\end{table}
\clearpage
\subsubsection{Usability and Humanity Requirements}

\begin{table}[h!]
\centering
\rowcolors{2}{white}{white}
\begin{tabular}{|p{3.5cm}|p{11.5cm}|}
\hline
\rowcolor{gray!30}
\textbf{UH1} & At least 90\% of healthcare professionals shall be able to perform common tasks (e.g., uploading images, viewing analysis results) without assistance after a 30-minute training session. \\
\hline
\textbf{Rationale} & Busy professionals need to use the system efficiently without extensive training. \\
\hline
\textbf{Fit Criterion} & Usability testing shows that users can complete key tasks unassisted after initial training. \\
\hline
\textbf{Traceability} & FR1, FR2, FR5, FR6 \\
\hline
\end{tabular}
\caption{Non-functional Requirement UH1}
\end{table}

\begin{table}[h!]
\centering
\rowcolors{2}{white}{white}
\begin{tabular}{|p{3.5cm}|p{11.5cm}|}
\hline
\rowcolor{gray!30}
\textbf{UH2} & The system shall provide context-sensitive help and tooltips for all interface elements to assist users in understanding functionalities. \\
\hline
\textbf{Rationale} & Immediate assistance reduces errors and enhances the user experience. \\
\hline
\textbf{Fit Criterion} & All interactive elements display helpful tooltips when hovered over, and a help section is accessible from all screens. \\
\hline
\textbf{Traceability} & FR2, FR6, FR11 \\
\hline
\end{tabular}
\caption{Non-functional Requirement UH2}
\end{table}
\clearpage
\subsubsection{Performance Requirements}

\begin{table}[h!]
\centering
\rowcolors{2}{white}{white}
\begin{tabular}{|p{3.5cm}|p{11.5cm}|}
\hline
\rowcolor{gray!30}
\textbf{PR1} & The system shall process and analyze a standard chest X-ray image within 30 seconds. \\
\hline
\textbf{Rationale} & Quick analysis is critical for timely diagnosis and treatment decisions in clinical settings. \\
\hline
\textbf{Fit Criterion} & Performance tests demonstrate that 95\% of standard images are processed within the 30-second threshold. \\
\hline
\textbf{Traceability} & FR3, FR4, FR5, FR8 \\
\hline
\end{tabular}
\caption{Non-functional Requirement PR1}
\end{table}

\begin{table}[h!]
\centering
\rowcolors{2}{white}{white}
\begin{tabular}{|p{3.5cm}|p{11.5cm}|}
\hline
\rowcolor{gray!30}
\textbf{PR2} & The system shall have an uptime of at least 99.9\%, ensuring high availability for users. \\
\hline
\textbf{Rationale} & Continuous access is essential in healthcare environments to prevent delays in patient care. \\
\hline
\textbf{Fit Criterion} & System logs indicate an uptime of 99.9\% over a 6-month monitoring period. \\
\hline
\textbf{Traceability} & FR5, FR7 \\
\hline
\end{tabular}
\caption{Non-functional Requirement PR2}
\end{table}

\begin{table}[h!]
\centering
\rowcolors{2}{white}{white}
\begin{tabular}{|p{3.5cm}|p{11.5cm}|}
\hline
\rowcolor{gray!30}
\textbf{PR3} & The system shall support concurrent processing of up to 20 images without significant performance degradation. \\
\hline
\textbf{Rationale} & Radiologists may need to process multiple images simultaneously, especially in busy clinics or hospitals. \\
\hline
\textbf{Fit Criterion} & System performance tests confirm that processing time per image does not exceed 45 seconds when 20 images are processed concurrently. \\
\hline
\textbf{Traceability} & FR5, FR7, FR15. \\
\hline
\end{tabular}
\caption{Non-functional Requirement PR3}
\end{table}
\clearpage
\subsubsection{Operational and Environmental Requirements}

\begin{table}[h!]
\centering
\rowcolors{2}{white}{white}
\begin{tabular}{|p{3.5cm}|p{11.5cm}|}
\hline
\rowcolor{gray!30}
\textbf{OE1} & The system shall integrate with existing hospital Picture Archiving and Communication Systems (PACS) using standard protocols like DICOM. \\
\hline
\textbf{Rationale} & Seamless integration with PACS improves workflow efficiency and reduces manual data handling. \\
\hline
\textbf{Fit Criterion} & The system successfully exchanges data with at least three major PACS solutions in a test environment. \\
\hline
\textbf{Traceability} & FR7, FR10, FR14 \\
\hline
\end{tabular}
\caption{Non-functional Requirement OE1}
\end{table}

\begin{table}[h!]
\centering
\rowcolors{2}{white}{white}
\begin{tabular}{|p{3.5cm}|p{11.5cm}|}
\hline
\rowcolor{gray!30}
\textbf{OE2} & The system shall operate effectively in network environments with latency up to 200ms and packet loss up to 1\%. \\
\hline
\textbf{Rationale} & Hospital networks may have variable conditions; the system must remain functional under less-than-ideal circumstances. \\
\hline
\textbf{Fit Criterion} & System testing under simulated network conditions confirms acceptable performance and responsiveness. \\
\hline
\textbf{Traceability} & FR4, FR6 \\
\hline
\end{tabular}
\caption{Non-functional Requirement OE2}
\end{table}
\clearpage
\subsubsection{Security and Privacy Requirements}

\begin{table}[h!]
\centering
\rowcolors{2}{white}{white}
\begin{tabular}{|p{3.5cm}|p{11.5cm}|}
\hline
\rowcolor{gray!30}
\textbf{SR1} & The system shall encrypt all patient data, including images and reports, both in transit and at rest, using AES-256 encryption. \\
\hline
\textbf{Rationale} & Protecting sensitive patient information is critical to comply with legal requirements and maintain trust. \\
\hline
\textbf{Fit Criterion} & Security audits confirm that all data storage and transmissions use AES-256 encryption. \\
\hline
\textbf{Traceability} & FR7, FR12,FR3 \\
\hline
\end{tabular}
\caption{Non-functional Requirement SR1}
\end{table}

\begin{table}[h!]
\centering
\rowcolors{2}{white}{white}
\begin{tabular}{|p{3.5cm}|p{11.5cm}|}
\hline
\rowcolor{gray!30}
\textbf{SR2} & The system shall implement role-based access control, ensuring users can only access functionalities and data appropriate to their roles (e.g., radiologist, radiographer, administrator). \\
\hline
\textbf{Rationale} & Restricting access based on roles prevents unauthorized data access and potential data breaches. \\
\hline
\textbf{Fit Criterion} & Access control tests verify that users cannot access data or functions outside their permissions. \\
\hline
\textbf{Traceability} & FR13, FR12 \\
\hline
\end{tabular}
\caption{Non-functional Requirement SR2}
\end{table}
\clearpage
\subsubsection{Maintainability and Support Requirements}

\begin{table}[h!]
\centering
\rowcolors{2}{white}{white}
\begin{tabular}{|p{3.5cm}|p{11.5cm}|}
\hline
\rowcolor{gray!30}
\textbf{MS1} & The system shall be developed using modular architecture with well-documented code to facilitate future maintenance and upgrades. \\
\hline
\textbf{Rationale} & A modular design simplifies debugging, testing, and integration of new features, reducing maintenance costs. \\
\hline
\textbf{Fit Criterion} & Code reviews confirm adherence to coding standards and modular design principles; documentation covers all modules. \\
\hline
\textbf{Traceability} & FR2, FR10, FR16 \\
\hline
\end{tabular}
\caption{Non-functional Requirement MS1}
\end{table}

\begin{table}[h!]
\centering
\rowcolors{2}{white}{white}
\begin{tabular}{|p{3.5cm}|p{11.5cm}|}
\hline
\rowcolor{gray!30}
\textbf{MS2} & The system shall include automated testing suites covering at least 80\% of the codebase to ensure reliability during updates. \\
\hline
\textbf{Rationale} & High test coverage helps detect issues early and maintains system stability over time. \\
\hline
\textbf{Fit Criterion} & Test coverage reports indicate that automated tests cover 80\% or more of the code. \\
\hline
\textbf{Traceability} & FR7, FR14, FR16\\
\hline
\end{tabular}
\caption{Non-functional Requirement MS2}
\end{table}

\subsubsection{Cultural Requirements}

\begin{table}[h!]
\centering
\rowcolors{2}{white}{white}
\begin{tabular}{|p{3.5cm}|p{11.5cm}|}
\hline
\rowcolor{gray!30}
\textbf{CR1} & The system shall support both English and French languages for the user interface and reports. \\
\hline
\textbf{Rationale} & Supporting Canada's official languages ensures accessibility for all users and meets legal requirements in certain provinces. \\
\hline
\textbf{Fit Criterion} & Users can select English or French as the system language, with all interface elements and reports displayed accordingly. \\
\hline
\textbf{Traceability} & FR6, FR12, FR13 \\
\hline
\end{tabular}
\caption{Non-functional Requirement CR1}
\end{table}
\clearpage
\subsubsection{Legal Requirements}

\begin{table}[h!]
\centering
\rowcolors{2}{white}{white}
\begin{tabular}{|p{3.5cm}|p{11.5cm}|}
\hline
\rowcolor{gray!30}
\textbf{LR1} & The system shall comply with all applicable healthcare data protection laws, including HIPAA in the United States and PIPEDA in Canada. \\
\hline
\textbf{Rationale} & Legal compliance is mandatory to protect patient rights and avoid legal repercussions. \\
\hline
\textbf{Fit Criterion} & Compliance audits verify adherence to relevant data protection laws and regulations. \\
\hline
\textbf{Traceability} & FR12, FR13, FR7 \\
\hline
\end{tabular}
\caption{Non-functional Requirement LR1}
\end{table}

\begin{table}[h!]
\centering
\rowcolors{2}{white}{white}
\begin{tabular}{|p{3.5cm}|p{11.5cm}|}
\hline
\rowcolor{gray!30}
\textbf{LR2} & The system shall adhere to the ISO 13485 standard for medical device software development. \\
\hline
\textbf{Rationale} & Following recognized standards ensures quality and safety in medical software. \\
\hline
\textbf{Fit Criterion} & Certification or compliance reports demonstrate adherence to ISO 13485. \\
\hline
\textbf{Traceability} & FR7, FR10, FR15 \\
\hline
\end{tabular}
\caption{Non-functional Requirement LR2}
\end{table}
\clearpage
\subsubsection{Health and Safety Requirements}

\begin{table}[h!]
\centering
\rowcolors{2}{white}{white}
\begin{tabular}{|p{3.5cm}|p{11.5cm}|}
\hline
\rowcolor{gray!30}
\textbf{HS1} & The system shall ensure that all AI-generated diagnoses are reviewed and confirmed by a qualified radiologist before being used in patient care decisions. \\
\hline
\textbf{Rationale} & To prevent misdiagnoses and ensure patient safety by involving human oversight. \\
\hline
\textbf{Fit Criterion} & System workflow requires radiologist validation before finalizing reports; logs confirm this process. \\
\hline
\textbf{Traceability} & FR3, FR6, FR4 \\
\hline
\end{tabular}
\caption{Non-functional Requirement HS1}
\end{table}

\begin{table}[h!]
\centering
\rowcolors{2}{white}{white}
\begin{tabular}{|p{3.5cm}|p{11.5cm}|}
\hline
\rowcolor{gray!30}
\textbf{HS2} & The system shall provide clear disclaimers indicating that AI analysis is a diagnostic aid and not a replacement for professional medical judgment. \\
\hline
\textbf{Rationale} & Users must understand the limitations of AI to prevent overreliance and potential errors. \\
\hline
\textbf{Fit Criterion} & Disclaimers are prominently displayed on analysis results and require user acknowledgment upon first use. \\
\hline
\textbf{Traceability} &FR12, FR6, FR11\\
\hline
\end{tabular}
\caption{Non-functional Requirement HS2}
\end{table}

\newpage
\section{Phase in Plan}    
\subsection{High Priority Functional Requirements}
\begin{table}[H]
  \label{TblHighPriorityFRs}
  \begin{tabular}{p{0.15\textwidth}|p{0.85\textwidth}}
  \toprule
  \textbf{Req ID} & \textbf{Completion Date and Rationale} \\
  \midrule
  FR1 & \textbf{Completion Date:} November 13, 2023\\
      & \textbf{Rationale:} FR1 is essential because accepting chest X-ray images is the foundation of our system. We need this functionality first to enable all other features to build upon it. \\
  \midrule
  FR3 & \textbf{Completion Date:} November 13, 2023\\
      & \textbf{Rationale:} FR3 is critical as it provides the core capability of analyzing X-rays for disease detection. Implementing this early allows us to validate the system's primary purpose and proceed with dependent features. \\
  \midrule
  FR4 & \textbf{Completion Date:} November 13, 2023\\
      & \textbf{Rationale:} FR4 is important for monitoring patient progress. By completing it alongside FR3, we ensure clinicians can assess changes over time from the initial release. \\
  \midrule
  FR6 & \textbf{Completion Date:} November 13, 2023\\
      & \textbf{Rationale:} FR6 allows us to present analysis results in a usable format for clinicians. Having this ready by November 13 is vital for user acceptance and practical use of the system. \\
  \midrule
  FR11 & \textbf{Completion Date:} November 13, 2023\\
       & \textbf{Rationale:} FR11 improves user trust by displaying confidence levels. An intuitive UI is essential for user adoption;therefore, we aim to have it completed by November 13. \\
  \midrule
  FR13 & \textbf{Completion Date:} November 13, 2023\\
       & \textbf{Rationale:} FR13 is necessary for security and compliance, ensuring proper access control from the start. Implementing it early helps us protect sensitive data effectively. \\
  \midrule
  FR14 & \textbf{Completion Date:} November 13, 2023\\
       & \textbf{Rationale:} FR14 safeguards the integrity of original X-ray images. Completing this by November 13 is important for legal considerations. \\
  \bottomrule
  \end{tabular}
  \end{table}

  \subsection{Medium Priority Functional Requirements}
\begin{table}[H]
  \label{TblMediumPriorityFRs}
  \begin{tabular}{p{0.15\textwidth}|p{0.85\textwidth}}
  \toprule
  \textbf{Req ID} & \textbf{Completion Date and Rationale} \\
  \midrule
  FR2 & \textbf{Completion Date:} February 5, 2024\\
      & \textbf{Rationale:} FR2 adds the ability to input additional patient symptoms, enhancing analysis accuracy. We scheduled it after the core functionalities to focus on essential features first. \\
  \midrule
  FR5 & \textbf{Completion Date:} February 5, 2024\\
      & \textbf{Rationale:} FR5 provides visual aids, improving usability. Implementing it after the main analysis features allows us to refine the user experience based on initial feedback. \\
  \midrule
  FR7 & \textbf{Completion Date:} February 5, 2024\\
      & \textbf{Rationale:} FR7 involves secure data storage for future reference. We plan to develop this after the initial deployment to ensure data handling aligns with user needs and compliance requirements. \\
  \midrule
  FR16 & \textbf{Completion Date:} February 5, 2024\\
       & \textbf{Rationale:} FR16 enables regular updates to the AI model, allowing continuous improvement. Scheduling it for February gives us time to stabilize the initial model and plan for updates without disrupting service. \\
  \bottomrule
  \end{tabular}
  \caption{High and Medium Priority Functional Requirements}
  \end{table}
  \subsection{Low Priority Functional Requirements}
\begin{table}[H]
  \label{TblLowPriorityFRs}
  \begin{tabular}{p{0.15\textwidth}|p{0.85\textwidth}}
  \toprule
  \textbf{Req ID} & \textbf{Completion Date and Rationale} \\
  \midrule
  FR8 & \textbf{Completion Date:} March 18, 2024\\
      & \textbf{Rationale:} FR8 offers alerts for significant condition changes. It's useful but not essential for initial deployment, so we scheduled it after higher-priority tasks. \\
  \midrule
  FR10 & \textbf{Completion Date:} March 18, 2024\\
       & \textbf{Rationale:} FR10 involves integrating with EHR systems, which adds complexity. We plan to address this later to focus on core functionalities first. \\
  \midrule
  FR12 & \textbf{Completion Date:} March 18, 2024\\
       & \textbf{Rationale:} FR12 allows patient self-upload of images. Given legal considerations and our initial focus on clinicians, we scheduled it for later development. \\
  \midrule
  FR15 & \textbf{Completion Date:} March 18, 2024\\
       & \textbf{Rationale:} FR15 expands support to other imaging modalities. We aim to perfect chest X-ray analysis before adding CT and MRI support, so we scheduled this for the last phase. \\
  \bottomrule
  \end{tabular}
  \caption{Low Priority Functional Requirements}
  \end{table}
  
\subsection{High Priority Non-Functional Requirements}
\begin{table}[H]
  \label{TblHighPriorityNFRs}
  \begin{tabular}{p{0.15\textwidth}|p{0.85\textwidth}}
  \toprule
  \textbf{Req ID} & \textbf{Completion Date and Rationale} \\
  \midrule
  PR1 & \textbf{Completion Date:} November 13, 2023\\
      & \textbf{Rationale:} PR1 ensures the system processes images quickly, which is crucial for usability in clinical settings. \\
  \midrule
  HS1 & \textbf{Completion Date:} November 13, 2023\\
      & \textbf{Rationale:} HS1 mandates that AI diagnoses are reviewed by a radiologist, ensuring patient safety. This oversight is necessary from day one. \\
  \midrule
  LF1 & \textbf{Completion Date:} November 13, 2023\\
      & \textbf{Rationale:} LF1 requires an interface consistent with standard medical software, aiding user adoption. We want users to feel comfortable using the system immediately. \\
  \midrule
  PR2 & \textbf{Completion Date:} November 13, 2023\\
      & \textbf{Rationale:} PR2 aims for high system uptime. Reliability is critical, so we must ensure the system is stable from the initial launch. \\
  \midrule
  SR2 & \textbf{Completion Date:} November 13, 2023\\
      & \textbf{Rationale:} SR2 implements role-based access control, necessary for security and compliance. We need to control data access from the beginning. \\
  \midrule
  MS1 & \textbf{Completion Date:} November 13, 2023\\
      & \textbf{Rationale:} MS1 involves modular architecture and documentation, facilitating future maintenance. \\
  \midrule
  LR1 & \textbf{Completion Date:} November 13, 2023\\
      & \textbf{Rationale:} LR1 ensures compliance with data protection laws, which is legally required. We must adhere to regulations from the outset to avoid legal issues. \\
  \midrule
  LR2 & \textbf{Completion Date:} November 13, 2023\\
      & \textbf{Rationale:} LR2 requires adherence to ISO 13485, ensuring quality and safety. Meeting industry standards builds trust and facilitates future certifications. \\
  \midrule
  HS2 & \textbf{Completion Date:} November 13, 2023\\
      & \textbf{Rationale:} HS2 mandates clear disclaimers about AI limitations. Providing this information from the start is important for user awareness and legal protection. \\
  \bottomrule
  \end{tabular}
  \end{table}
  \subsection{Medium and Low Priority Non-Functional Requirements}
\begin{table}[H]
  \label{TblMediumLowPriorityNFRs}
  \begin{tabular}{p{0.15\textwidth}|p{0.85\textwidth}}
  \toprule
  \textbf{Req ID} & \textbf{Completion Date and Rationale} \\
  \midrule
  LF2 & \textbf{Completion Date:} February 5, 2024\\
      & \textbf{Rationale:} LF2 improves the user interface to minimize eye strain. While important for user comfort, it can be refined after the main features are operational. \\
  \midrule
  UH2 & \textbf{Completion Date:} February 5, 2024\\
      & \textbf{Rationale:} UH2 adds context-sensitive help and tooltips. Enhancing usability is beneficial, but we plan to implement it after gathering initial user feedback. \\
  \midrule
  PR3 & \textbf{Completion Date:} February 5, 2024\\
      & \textbf{Rationale:} PR3 enhances performance by supporting concurrent image processing. Scheduling this for later allows us to optimize the system after core functionalities are stable. \\
  \midrule
  MS2 & \textbf{Completion Date:} February 5, 2024\\
      & \textbf{Rationale:} MS2 involves automated testing for reliability. Implementing it after initial development helps us maintain code quality as the project grows. \\
  \midrule
  OE1 & \textbf{Completion Date:} March 18, 2024\\
       & \textbf{Rationale:} OE1 integrates with hospital PACS systems. Given the complexity, we plan to tackle this once the system's core features are solidified. \\
  \midrule
  OE2 & \textbf{Completion Date:} March 18, 2024\\
       & \textbf{Rationale:} OE2 ensures operation under variable network conditions. This enhancement is scheduled for later, focusing first on standard network environments. \\
  \midrule
  CR1 & \textbf{Completion Date:} March 18, 2024\\
       & \textbf{Rationale:} CR1 adds multilingual support. We aim to include additional languages after ensuring the system functions well in the primary language. \\
  \bottomrule
  \end{tabular}
  \caption{Medium and Low Priority Non-Functional Requirements}
  \end{table}
  
  
\clearpage
\section{Likelihood of Changes for Functional Requirements}

\subsection{Very Unlikely to Change}

These requirements are core to our system's functionality and are essential for meeting the project's objectives. We do not anticipate any changes to these requirements.

\begin{enumerate}
    \item \textbf{FR1}: Accept chest X-ray images from authorized users.
    \begin{itemize}[label=-]
        \item \textbf{Rationale}: This is the fundamental functionality upon which all other features depend.
        \item \textbf{Possible Changes}: N/A
    \end{itemize}

    \item \textbf{FR3}: Analyze chest X-rays to detect diseases with $\geq85\%$ accuracy.
    \begin{itemize}[label=-]
        \item \textbf{Rationale}: Central to the system's purpose; critical for providing value to users.
        \item \textbf{Possible Changes}: N/A
    \end{itemize}

    \item \textbf{FR6}: Produce structured, human-readable reports summarizing findings.
    \begin{itemize}[label=-]
        \item \textbf{Rationale}: Essential for clinicians to interpret results effectively.
        \item \textbf{Possible Changes}: N/A
    \end{itemize}

    \item \textbf{FR13}: Support multiple user roles with appropriate access levels.
    \begin{itemize}[label=-]
        \item \textbf{Rationale}: Crucial for security and compliance with regulations.
        \item \textbf{Possible Changes}: N/A
    \end{itemize}
\end{enumerate}

\subsection{Unlikely to Change}

These requirements are important but may undergo minor adjustments based on user feedback or technical considerations.

\begin{enumerate}[resume]
    \item \textbf{FR2}: Enable users to input additional patient symptoms.
    \begin{itemize}[label=-]
        \item \textbf{Rationale}: Enhances analysis accuracy but not critical for initial operation.
        \item \textbf{Possible Changes}: Modify symptom input methods or adjust symptom lists.
    \end{itemize}

    \item \textbf{FR4}: Indicate if a patient's condition has changed between scans.
    \begin{itemize}[label=-]
        \item \textbf{Rationale}: Important for monitoring progress; criteria may need refinement.
        \item \textbf{Possible Changes}: Adjust methods for assessing condition changes.
    \end{itemize}

    \item \textbf{FR5}: Generate visual aids by highlighting affected areas.
    \begin{itemize}[label=-]
        \item \textbf{Rationale}: Improves usability; visualization techniques may evolve.
        \item \textbf{Possible Changes}: Update highlighting methods based on user feedback.
    \end{itemize}

    \item \textbf{FR7}: Store patient data securely for future reference.
    \begin{itemize}[label=-]
        \item \textbf{Rationale}: Important for data continuity; storage solutions might change.
        \item \textbf{Possible Changes}: Enhance security measures or adopt new technologies.
    \end{itemize}

    \item \textbf{FR9}: Allow healthcare professionals to adjust treatment plans.
    \begin{itemize}[label=-]
        \item \textbf{Rationale}: Adds value for clinical decision-making; integration methods may adjust.
        \item \textbf{Possible Changes}: Refine how treatment adjustments are linked to analysis.
    \end{itemize}

    \item \textbf{FR11}: Display confidence levels through an intuitive UI.
    \begin{itemize}[label=-]
        \item \textbf{Rationale}: Helps assess result reliability; presentation might be refined.
        \item \textbf{Possible Changes}: Modify the display format of confidence levels.
    \end{itemize}

    \item \textbf{FR14}: Create a new copy of the X-ray before AI analysis.
    \begin{itemize}[label=-]
        \item \textbf{Rationale}: Protects data integrity; implementation methods might improve.
        \item \textbf{Possible Changes}: Optimize data handling or copying procedures.
    \end{itemize}

    \item \textbf{FR16}: Support regular updates to the AI model.
    \begin{itemize}[label=-]
        \item \textbf{Rationale}: Important for continuous improvement; update processes may adjust.
        \item \textbf{Possible Changes}: Alter the frequency or method of updates.
    \end{itemize}
\end{enumerate}

\subsection{Likely to Change}

These requirements may change due to technical challenges, resource constraints, or evolving project scope.

\begin{enumerate}[resume]
    \item \textbf{FR8}: Provide alerts for significant changes between scans.
    \begin{itemize}[label=-]
        \item \textbf{Rationale}: Useful but not essential; may be deferred due to time constraints.
        \item \textbf{Possible Changes}: Simplify the alert system or postpone implementation.
    \end{itemize}

    \item \textbf{FR10}: Integrate with EHR systems to import/export data.
    \begin{itemize}[label=-]
        \item \textbf{Rationale}: Adds complexity; might be challenging within our timeline.
        \item \textbf{Possible Changes}: Limit integration scope or schedule for future development.
    \end{itemize}

    \item \textbf{FR12}: Allow patients to upload images for self-diagnosis with disclaimers.
    \begin{itemize}[label=-]
        \item \textbf{Rationale}: Raises legal and ethical concerns; may need alteration.
        \item \textbf{Possible Changes}: Restrict feature access or require additional consent.
    \end{itemize}

    \item \textbf{FR15}: Support additional imaging modalities like CT scans and MRIs.
    \begin{itemize}[label=-]
        \item \textbf{Rationale}: May exceed current project scope; could be deferred.
        \item \textbf{Possible Changes}: Focus on chest X-rays initially; schedule other modalities for later phases.
    \end{itemize}
\end{enumerate}

\section{Likelihood of Changes for Non-Functional Requirements}

\subsection{Very Unlikely to Change}

These requirements are essential for legal compliance, security, and core system performance.

\begin{enumerate}
    \item \textbf{SR1}: Encrypt all patient data using AES-256 encryption.
    \begin{itemize}[label=-]
        \item \textbf{Rationale}: Mandatory for data security and compliance.
        \item \textbf{Possible Changes}: N/A
    \end{itemize}

    \item \textbf{SR2}: Implement role-based access control.
    \begin{itemize}[label=-]
        \item \textbf{Rationale}: Critical for security and regulatory compliance.
        \item \textbf{Possible Changes}: N/A
    \end{itemize}

    \item \textbf{LR1}: Comply with healthcare data protection laws (HIPAA, PIPEDA).
    \begin{itemize}[label=-]
        \item \textbf{Rationale}: Legal compliance is mandatory.
        \item \textbf{Possible Changes}: N/A
    \end{itemize}
\end{enumerate}

\subsection{Unlikely to Change}

These requirements are important for system usability, maintainability, and reliability but may undergo minor adjustments.

\begin{enumerate}[resume]
    \item \textbf{PR1}: Process and analyze images within 30 seconds.
    \begin{itemize}[label=-]
        \item \textbf{Rationale}: Critical for usability; performance targets might adjust slightly.
        \item \textbf{Possible Changes}: Optimize algorithms or extend processing time marginally.
    \end{itemize}

    \item \textbf{HS1}: Ensure AI diagnoses are reviewed by a radiologist.
    \begin{itemize}[label=-]
        \item \textbf{Rationale}: Important for patient safety; review processes may streamline.
        \item \textbf{Possible Changes}: Adjust workflow integration based on feedback.
    \end{itemize}

    \item \textbf{LF1}: Have a user interface consistent with standard medical software.
    \begin{itemize}[label=-]
        \item \textbf{Rationale}: Aids user adoption; UI elements might be updated.
        \item \textbf{Possible Changes}: Refine interface design for improved usability.
    \end{itemize}

    \item \textbf{UH1}: Users perform tasks without assistance after 30-minute training.
    \begin{itemize}[label=-]
        \item \textbf{Rationale}: Key for efficiency; training materials may be adjusted.
        \item \textbf{Possible Changes}: Modify training duration or content.
    \end{itemize}

    \item \textbf{PR2}: Achieve system uptime of at least 99.9\%.
    \begin{itemize}[label=-]
        \item \textbf{Rationale}: Essential for reliability; targets may adjust slightly.
        \item \textbf{Possible Changes}: Revise uptime goals if necessary.
    \end{itemize}

    \item \textbf{MS1}: Develop using modular architecture with well-documented code.
    \begin{itemize}[label=-]
        \item \textbf{Rationale}: Important for maintenance; practices may evolve.
        \item \textbf{Possible Changes}: Update documentation standards or modularization approaches.
    \end{itemize}

    \item \textbf{MS2}: Include automated testing covering at least 80\% of the codebase.
    \begin{itemize}[label=-]
        \item \textbf{Rationale}: Crucial for reliability; coverage targets may adjust.
        \item \textbf{Possible Changes}: Modify test coverage goals based on feasibility.
    \end{itemize}

    \item \textbf{LR2}: Adhere to ISO 13485 standard.
    \begin{itemize}[label=-]
        \item \textbf{Rationale}: Important for quality; compliance may adjust based on feasibility.
        \item \textbf{Possible Changes}: Plan for partial compliance initially, aiming for full adherence over time.
    \end{itemize}

    \item \textbf{HS2}: Provide disclaimers that AI is a diagnostic aid.
    \begin{itemize}[label=-]
        \item \textbf{Rationale}: Essential for legal and ethical reasons.
        \item \textbf{Possible Changes}: Refine wording of disclaimers as needed.
    \end{itemize}
\end{enumerate}

\subsection{Likely to Change}

These requirements may be adjusted due to resource constraints, technical challenges, or evolving user needs.

\begin{enumerate}[resume]
    \item \textbf{LF2}: Use color schemes and fonts that minimize eye strain.
    \begin{itemize}[label=-]
        \item \textbf{Rationale}: Aesthetic elements are subjective; may change with user feedback.
        \item \textbf{Possible Changes}: Redesign UI for improved comfort.
    \end{itemize}

    \item \textbf{UH2}: Provide context-sensitive help and tooltips.
    \begin{itemize}[label=-]
        \item \textbf{Rationale}: Resource constraints might affect implementation scope.
        \item \textbf{Possible Changes}: Adjust help features or prioritize key areas.
    \end{itemize}

    \item \textbf{PR3}: Support processing of up to 20 images concurrently.
    \begin{itemize}[label=-]
        \item \textbf{Rationale}: Performance under load may be challenging.
        \item \textbf{Possible Changes}: Reduce number of concurrent images or optimize performance.
    \end{itemize}

    \item \textbf{OE1}: Integrate with hospital PACS using DICOM.
    \begin{itemize}[label=-]
        \item \textbf{Rationale}: Integration complexity may limit scope.
        \item \textbf{Possible Changes}: Support specific PACS systems or delay integration.
    \end{itemize}

    \item \textbf{OE2}: Operate effectively in networks with latency up to 200ms.
    \begin{itemize}[label=-]
        \item \textbf{Rationale}: Network optimization may be challenging within our timeline.
        \item \textbf{Possible Changes}: Adjust network condition requirements or focus on standard environments first.
    \end{itemize}

    \item \textbf{CR1}: Support both English and French languages.
    \begin{itemize}[label=-]
        \item \textbf{Rationale}: Multilingual support adds complexity; may be postponed.
        \item \textbf{Possible Changes}: Limit initial support to English; add languages later.
    \end{itemize}
\end{enumerate}



\afterpage{
\begin{landscape}
\section{Traceability Matrix}
\begin{table}[h!]
\centering
\scriptsize
\setlength{\tabcolsep}{3pt} % Adjust the column separation to make it tighter
\renewcommand{\arraystretch}{1.15} % Adjust row height for a tighter fit
\begin{tabular}{|p{1cm}|p{0.9cm}|p{0.9cm}|p{0.9cm}|p{0.9cm}|p{0.9cm}|p{0.9cm}|p{0.9cm}|p{0.9cm}|p{0.9cm}|p{0.9cm}|p{0.9cm}|p{0.9cm}|p{0.9cm}|p{0.9cm}|p{0.9cm}|p{0.9cm}|}
\hline
    & \textbf{FR1} & \textbf{FR2} & \textbf{FR3} & \textbf{FR4} & \textbf{FR5} & \textbf{FR6} & \textbf{FR7} & \textbf{FR8} & \textbf{FR9} & \textbf{FR10} & \textbf{FR11} & \textbf{FR12} & \textbf{FR13} & \textbf{FR14} & \textbf{FR15} & \textbf{FR16} \\ 
\hline
\textbf{FR1} & X & X & & & & & X & & & & & & & & & \\ 
\hline
\textbf{FR2} & & X & & & & & & & & & & & & & & X \\ 
\hline
\textbf{FR3} & X & & X & X & X & X & & & & & & & & & & X \\ 
\hline
\textbf{FR4} & & & X & X & X & & & X & & & & & & & & \\ 
\hline
\textbf{FR5} & & & & X & X & X & & & & & & & & & X & \\ 
\hline
\textbf{FR6} & X & & X & X & X & X & & & X & & & & & & & \\ 
\hline
\textbf{FR7} & X & & & & & X & X & & & X & & & X & & & \\ 
\hline
\textbf{FR8} & & & & X & & X & & X & & & & & & & & \\ 
\hline
\textbf{FR9} & & & & & & X & & & X & & & & & & & \\ 
\hline
\textbf{FR10} & & & & & & & X & & & X & & & X & & & X \\ 
\hline
\textbf{FR11} & & & & & & & & & & & X & & & & & \\ 
\hline
\textbf{FR12} & X & & & & & & & & & & X & X & X & & & \\ 
\hline
\textbf{FR13} & & & & & & & X & & & X & & X & X & & & \\ 
\hline
\textbf{FR14} & X & & & & & & & & & & & & & X & & \\ 
\hline
\textbf{FR15} & X & & & & & & & & & & & & & & X & \\ 
\hline
\textbf{FR16} & & & X & & & & & & & X & & & & & X & X \\ 
\hline
\textbf{LF1} & X & & X & X & X & & & & & & & & & & & \\ 
\hline
\textbf{LF2} & X & & & & & & & & & & & & & & & \\ 
\hline
\textbf{UH1} & X & X & & & & & & & & & & & & & & \\ 
\hline
\textbf{UH2} & & X & & & X & X & & & & & X & & & & & \\ 
\hline
\textbf{PR1} & & & X & X & X & & & & & & & & & & & \\ 
\hline
\textbf{PR2} & & & & & X & & X & & & & & & & & & \\ 
\hline
\textbf{PR3} & & & & & X & & X & & & & & & & & & \\ 
\hline
\textbf{OE1} & & & & & & & X & & & & & & & X & & \\ 
\hline
\textbf{OE2} & & & & X & & X & & & & & & & & & & \\ 
\hline
\textbf{SR1} & & & & & & & X & & & & & X & & & & \\ 
\hline
\textbf{SR2} & & & & & & & & & & & & X & X & & & \\ 
\hline
\textbf{MS1} & & & & & & & & & & X & & & & & & X \\ 
\hline
\textbf{MS2} & & & & & & & X & & & & & & & X & & X \\ 
\hline
\textbf{CR1} & & & & & & & & & & & & & & X & & \\ 
\hline
\textbf{LR1} & & & & & & & & & & & & X & X & & & \\ 
\hline
\textbf{LR2} & & & & & & & X & & & X & & & & X & X & \\ 
\hline
\textbf{HS1} & & & X & X & & & & & & & & & & & & \\ 
\hline
\textbf{HS2} & & & & & & & & & & & & X & & & & \\ 
\hline
\end{tabular}
\caption{Traceability Matrix Showing the Dependencies Between Functional and Non-Functional Requirements}
\label{Table:Traceability}
\end{table}
\end{landscape}
}
\section{Values of Auxiliary Constants}

\begin{itemize}
    \item \textbf{Uptime:} 99.99\% – Ensures near-continuous availability of the AI system, minimizing downtime for uninterrupted analysis and diagnostics.
    \item \textbf{Training Photos:} 471.12 GB (zipped) – A substantial dataset of chest X-rays used to train the AI model, contributing to its robustness and accuracy.
    \item \textbf{Accuracy Rate:} 90\% – The model achieves a high accuracy rate in classifying and diagnosing chest X-rays, balancing precision with speed for practical clinical use.
    \item \textbf{Response Time:} Less than 1 second per image – The system is optimized to provide immediate feedback on X-ray results, significantly reducing wait times.
\end{itemize}

\section{Impact Analysis}
This project is primarily designed for rapid recognition of chest X-rays in the field of thoracic medicine through artificial intelligence. Its main significance lies in the ability to quickly display preliminary diagnostic results, thereby improving recognition efficiency and reducing patient wait times. Additionally, the system can be used for big data classification, gradually improving the accuracy of analyses, reducing misdiagnoses, and easing the workload of doctors. If this project can be deployed on a large scale, it will enhance productivity in the medical field, lower healthcare costs, and speed up medical services. Moreover, by utilizing cloud technology, it can help people living in remote areas to obtain basic assessments from X-rays without needing a doctor. We can further analyse the society factors into these following sections: Economic, Health, Safety, Cultural etc. \\

According to updated data from the Canadian government in 2023, the average annual salary for doctors is CAD 233,726.[Pat Link1] Each chest examination requires approximately 15 minutes, while the average waiting time for each chest X-ray is between 1 to 2 days [Pat Link2]. If this technology can reduce the average analysis time to immediate results, it would decrease the analysis cost for each X-ray by about CAD 30. In Canada, there are approximately 20 million X-ray examinations conducted annually [Pat Link 3], with chest X-rays accounting for around 6 to 8 million of those. Based on these calculations, this could save between CAD 160 million and CAD 240 million. \\

At the same time, automated data classification can be utilized in emergency situations. For example, during the COVID-19 pandemic, the healthcare system was severely overloaded, and there was a widespread shortage of doctors and nurses [Pat link 4]. If this technology can be employed to assist in classifying X-rays during the COVID-19 period, it would significantly increase efficiency and alleviate the burden on the healthcare system, while also helping more patients. This would enable the multifaceted benefits that come with technological advancements to be realized. \\

When evaluating the safety implications of using AI for disease diagnosis, two primary perspectives emerge. On one hand, AI can significantly speed up the diagnostic process, enabling patients to receive treatment more quickly, which is generally expected to have a positive impact. On the other hand, the accuracy of AI-generated diagnoses can be uncertain, as these models typically provide probabilities rather than definitive conclusions. This uncertainty can lead to potential risks, including ethical concerns and legal disputes, especially if a misdiagnosis occurs as every new technology might bring. [Pat link 5] \\

From a cultural perspective, the deployment of AI for medical diagnostics, such as analyzing chest X-rays, can lead to changes in the way healthcare is perceived and delivered[Pat link 6]. In some cultures, there may be a strong preference for face-to-face consultations with doctors, where personal interaction and trust in human judgment are valued highly. The introduction of AI could challenge these traditions, as patients might need to adjust to relying on technology for initial diagnoses. However, in regions where access to medical professionals is limited, such as remote or underserved areas, AI-based diagnostics could be seen as a valuable and empowering tool, providing a sense of autonomy and timely care [Pat link 6].

\newpage

\bibliographystyle {plainnat}
\bibliography {../../refs/References}

\newpage{}
\section*{Appendix --- Reflection}

\wss{Not required for CAS 741}

The information in this section will be used to evaluate the team members on the
graduate attribute of Lifelong Learning.  


\begin{enumerate}
  \item What went well while writing this deliverable? 
  \item What pain points did you experience during this deliverable, and how did
  you resolve them?
  \item How many of your requirements were inspired by speaking to your
  client(s) or their proxies (e.g. your peers, stakeholders, potential users)?
  \item Which of the courses you have taken, or are currently taking, will help
  your team to be successful with your capstone project.
  \item What knowledge and skills will the team collectively need to acquire to
  successfully complete this capstone project?  Examples of possible knowledge
  to acquire include domain specific knowledge from the domain of your
  application, or software engineering knowledge, mechatronics knowledge or
  computer science knowledge.  Skills may be related to technology, or writing,
  or presentation, or team management, etc.  You should look to identify at
  least one item for each team member.
  \item For each of the knowledge areas and skills identified in the previous
  question, what are at least two approaches to acquiring the knowledge or
  mastering the skill?  Of the identified approaches, which will each team
  member pursue, and why did they make this choice?
\end{enumerate}

\end{document}