% THIS DOCUMENT IS TAILORED TO REQUIREMENTS FOR SCIENTIFIC COMPUTING.  IT SHOULDN'T
% BE USED FOR NON-SCIENTIFIC COMPUTING PROJECTS
\documentclass[12pt]{article}

\usepackage{amsmath, mathtools}
\usepackage{amsfonts}
\usepackage{amssymb}
\usepackage{graphicx}
\usepackage{colortbl}
\usepackage{xr}
\usepackage{hyperref}
\usepackage{longtable}
\usepackage{xfrac}
\usepackage{tabularx}
\usepackage{float}
\usepackage{siunitx}
\usepackage{booktabs}
\usepackage{caption}
\usepackage{pdflscape}
\usepackage{afterpage}
\usepackage[table,xcdraw]{xcolor}
\usepackage{geometry}
\usepackage{array}
\usepackage{booktabs}
\usepackage{longtable}
\usepackage{enumitem}
\usepackage{graphicx}
\usepackage[numbers]{natbib}
\graphicspath{ {../assets/} }


%\usepackage{refcheck}

\hypersetup{
    bookmarks=true,         % show bookmarks bar?
      colorlinks=true,       % false: boxed links; true: colored links
    linkcolor=red,          % color of internal links (change box color with linkbordercolor)
    citecolor=green,        % color of links to bibliography
    filecolor=magenta,      % color of file links
    urlcolor=cyan           % color of external links
}

\input{../packages/Comments.tex}
%% Common Parts

\newcommand{\progname}{CXR} % PUT YOUR PROGRAM NAME HERE
\newcommand{\authname}{Team 27, Neuralyzers
\\ Ayman Akhras 
\\ Nathan Luong
\\ Patrick Zhou
\\ Kelly Deng
\\ Reza Jodeiri} % AUTHOR NAMES                  

\usepackage{hyperref}
    \hypersetup{colorlinks=true, linkcolor=blue, citecolor=blue, filecolor=blue,
                urlcolor=blue, unicode=false}
    \urlstyle{same}
                                

\input{../packages/Reflection.tex}

% For easy change of table widths
\newcommand{\colZwidth}{1.0\textwidth}
\newcommand{\colAwidth}{0.13\textwidth}
\newcommand{\colBwidth}{0.82\textwidth}
\newcommand{\colCwidth}{0.1\textwidth}
\newcommand{\colDwidth}{0.05\textwidth}
\newcommand{\colEwidth}{0.8\textwidth}
\newcommand{\colFwidth}{0.17\textwidth}
\newcommand{\colGwidth}{0.5\textwidth}
\newcommand{\colHwidth}{0.28\textwidth}

% Used so that cross-references have a meaningful prefix
\newcounter{defnum} %Definition Number
\newcommand{\dthedefnum}{GD\thedefnum}
\newcommand{\dref}[1]{GD\ref{#1}}
\newcounter{datadefnum} %Datadefinition Number
\newcommand{\ddthedatadefnum}{DD\thedatadefnum}
\newcommand{\ddref}[1]{DD\ref{#1}}
\newcounter{theorynum} %Theory Number
\newcommand{\tthetheorynum}{TM\thetheorynum}
\newcommand{\tref}[1]{TM\ref{#1}}
\newcounter{tablenum} %Table Number
\newcommand{\tbthetablenum}{TB\thetablenum}
\newcommand{\tbref}[1]{TB\ref{#1}}
\newcounter{assumpnum} %Assumption Number
\newcommand{\atheassumpnum}{A\theassumpnum}
\newcommand{\aref}[1]{A\ref{#1}}

\newcounter{adrnum} %ADR Number
\newcommand{\atheadrnum}{ADR\theadrnum}

\newcounter{goalnum} %Goal Number
\newcommand{\gthegoalnum}{GS\thegoalnum}
\newcommand{\gsref}[1]{GS\ref{#1}}
\newcounter{instnum} %Instance Number
\newcommand{\itheinstnum}{IM\theinstnum}
\newcommand{\iref}[1]{IM\ref{#1}}
\newcounter{reqnum} %Requirement Number
\newcommand{\rthereqnum}{R\thereqnum}
\newcommand{\rref}[1]{R\ref{#1}}
\newcounter{nfrnum} %NFR Number
\newcommand{\rthenfrnum}{NFR\thenfrnum}
\newcommand{\nfrref}[1]{NFR\ref{#1}}
\newcounter{lcnum} %Likely change number
\newcommand{\lthelcnum}{LC\thelcnum}
\newcommand{\lcref}[1]{LC\ref{#1}}

\usepackage{fullpage}

\newcommand{\deftheory}[9][Not Applicable]
{
\newpage
\noindent \rule{\textwidth}{0.5mm}

\paragraph{RefName: } \textbf{#2} \phantomsection 
\label{#2}

\paragraph{Label:} #3

\noindent \rule{\textwidth}{0.5mm}

\paragraph{Equation:}

#4

\paragraph{Description:}

#5

\paragraph{Notes:}

#6

\paragraph{Source:}

#7

\paragraph{Ref.\ By:}

#8

\paragraph{Preconditions for \hyperref[#2]{#2}:}
\label{#2_precond}

#9

\paragraph{Derivation for \hyperref[#2]{#2}:}
\label{#2_deriv}

#1

\noindent \rule{\textwidth}{0.5mm}

}

\begin{document}

\title{Software Requirements Specification for \progname: subtitle describing software} 
\author{\authname}
\date{\today}
	
\maketitle

~\newpage
\pagenumbering{arabic}
\tableofcontents

~\newpage

\section*{Revision History}

\begin{tabularx}{\textwidth}{p{3cm}p{2cm}X}
  \toprule
  {\bf Date} & {\bf Version} & {\bf Notes}\\
  \midrule
  2024-10-01 & 1.0 & Added reference material, functional and non-functional requirements, and outlined the purpose of the project.\\
  2024-10-07 & 1.1 & Included details about stakeholders, system constraints, and an analysis of the potential impact of the project on various sectors.\\
  2024-10-11 & 1.2 & Developed formal mathematical specifications, designed a Finite State Machine (FSM) for the system, and provided a reflection on the project's challenges and future directions.\\
  2024-10-11 & 1.3 & Add personas, change system constraints, add more details to user characteristics and stakeholders\\
  \bottomrule
  \end{tabularx}


~\newpage

\section{Reference Material}

This section records information for easy reference and aims to reduce ambiguity in understanding key concepts used in the project.

\subsection{Table of Units}

Throughout this document, SI (Système International d'Unités) is employed as the unit system. In addition to basic units, several derived units are used as described below. For each unit, the symbol is given, followed by a description of the unit and the SI name.

\renewcommand{\arraystretch}{1.2}
\noindent \begin{tabular}{l l l} 
    \toprule		
    \textbf{Symbol} & \textbf{Unit} & \textbf{SI}\\
    \midrule 
    \si{\second} & Time & Second\\
    \si{GB} & Data & Gigabyte\\
    \si{MB} & Data & Megabyte\\
    \si{LOC} & Quantity & Lines of Code\\
    \bottomrule
\end{tabular}

\subsection{Definitions}
This subsection provides a list of terms that are used in the subsequent sections and their meanings, with the purpose of reducing ambiguity and making it easier to correctly understand the requirements:

\begin{itemize}
    
    
    \item[-] \textbf{Artificial Intelligence (AI) Model}: A program that analyzes datasets to identify patterns and make predictions. Used extensively in medical image analysis for automating diagnostics. 
    
    \item[-] \textbf{Convolutional Neural Network (CNN)}: A deep learning algorithm that processes images by assigning weights and biases, allowing it to identify patterns and features in medical images such as chest X-rays.
    
    \item[-] \textbf{DICOM (Digital Imaging and Communications in Medicine)}: The international standard for medical images, defining formats for image exchange that ensure clinical quality.
    
    \item[-] \textbf{Containerized Application}: A portable version of an application that can be run on a container run-time, such as Docker.
  
    \item[-] \textbf{Machine Learning (ML)}: A subset of AI focusing on using data and algorithms to mimic human learning, improving accuracy over time.
    \item[-] \textbf{Picture Archiving and Communication System (PACS)}: A system for acquiring, storing, transmitting, and displaying medical images digitally, providing a filmless clinical environment.
    \item[-] \textbf{PHI}: Personal Health Information - Private and confidential data that must be protected under the HIPPA act.
    \item[-] \textbf{HIPPA}: Health Insurance Portability and Accountability Act, a set of standards protecting sensitive health information from disclosure without patient's consent.
    \item[-] \textbf{AWS - Amazon Web Services}: A public cloud provider, offering all HIPAA-compliance cloud services that helps Neuralanalyzer host, manage, and scale our application.
    \item[-] \textbf{AWS ECS}: AWS Elastic Cloud Service: - An AWS managed service for managing and maintaining application containers at run-time.
    \item[-] \textbf{AWS ECR}: AWS Elastic Container Registry - An AWS managed service for storing and managing container images.
    \item[-] \textbf{AWS Fargate}: An AWS managed service for running containerized applications.
    \item[-] \textbf{AWS Cognito}: An AWS managed service for authentication logic, handling user and password management.
    \item[-] \textbf{React}: A web front-end framework, written in Javascript.
    \item[-] \textbf{Flask}: An HTTP-based server framework, written in Python.
    \item[-] \textbf{Finite State Machine (FSM)}: A computation model that simulates sequential logic using state transitions, applied in processes like user authentication and backend workflows.
    
    \item[-] \textbf{ROC Curve (Receiver Operating Characteristic Curve)}: A graph that shows the performance of a classification model by plotting the true positive rate against the false positive rate at various threshold levels.
    
    \item[-] \textbf{Service-Level Agreement (SLA)}: Defines the guaranteed uptime of the system, such as ensuring the availability of the AI service for 99.99\% of operational hours.
    
    \item[-] \textbf{Software as a Medical Device (SaMD)}: Software classified as a medical device under regulatory frameworks, such as those defined by the Food and Drugs Act.
    
    \item[-] \textbf{TorchXRAYVision}: An open-source library for classifying diseases based on chest X-ray images, offering pre-trained models to accelerate the development process.
    
    \item[-] \textbf{X-ray}: A form of high-energy electromagnetic radiation used in medical imaging to produce images of the inside of the body, enabling the diagnosis of conditions through radiographic film or digital detectors.
    
    \item[-] \textbf{MIT License}: An open-source software license that allows for the free use, modification, and distribution of software.
    
    \item[-] \textbf{Training Data}: Refers to the dataset of labeled chest X-ray images used to train the AI model. In this project, the dataset size is approximately 471.12 GB.
    
\end{itemize}

\subsection{Abbreviations and Acronyms}
\renewcommand{\arraystretch}{1.3}
\noindent \begin{tabular}{l l} 
  \toprule		
  \textbf{Symbol} & \textbf{Description}\\
  \midrule 
  SRS & Software Requirements Specification\\
  AI & Artificial Intelligence\\
  CNN & Convolutional Neural Network\\
  DICOM & Digital Imaging and Communications in Medicine\\
  IVDDs & In Vitro Diagnostic Devices\\
  ML & Machine Learning\\
  PACS & Picture Archiving and Communication System\\
  SaMD & Software as a Medical Device\\
  ROC & Receiver Operating Characteristic Curve\\
  SLA & Service-Level Agreement\\
  FR & Functional Requirement\\
  NFR & Non-Functional Requirement\\
  FSM & Finite State Machine\\
  CXR & Chest X-Ray Project\\
  POC & Proof of Concept\\
  TM & Theoretical Model\\
  AWS & Amazon Web Services\\
  ECS & Elastic Container Service\\
  ECR & Elastic Container Registry\\
  \bottomrule
\end{tabular}
\newpage
\subsection{Values of Auxiliary Constants}

\renewcommand{\arraystretch}{1.3}
\noindent \begin{tabular}{|p{3.5cm}|p{9.5cm}|}
    \hline
    \rowcolor{lightgray} \textbf{Parameter} & \textbf{Description} \\
    \hline
    Uptime & 99.99\% – Ensures near-continuous availability of the AI system, minimizing downtime for uninterrupted analysis and diagnostics. \\
    \hline
    Training Photos & 471.12 GB (zipped) – A substantial dataset of chest X-rays used to train the AI model, contributing to its robustness and accuracy. \\
    \hline
    Accuracy Rate & 90\% – The model achieves a high accuracy rate in classifying and diagnosing chest X-rays, balancing precision with speed for practical clinical use. \\
    \hline
    Response Time & Less than 1 second per image – The system is optimized to provide immediate feedback on X-ray results, significantly reducing wait times. \\
    \hline
    Batch Size & 32 images per batch – Defines the number of images processed together during each training iteration. \\
    \hline
    Learning Rate & 0.001 – The step size during gradient descent, impacting model training speed. \\
    \hline
    Number of Epochs & 50 – The number of times the model processes the entire training dataset. \\
    \hline
    Input Image Size & 224x224 pixels – Suitable for CNN architectures, balancing computational efficiency and image detail. \\
    \hline
    Threshold for ROC Analysis & 0.85 – The probability cut-off for positive classification, balancing sensitivity and specificity. \\
    \hline
    Precision & 92\% – Reflects the percentage of true positive cases among all cases predicted positive. \\
    \hline
    Recall & 88\% – Reflects the percentage of true positive cases among all actual positive cases. \\
    \hline
\end{tabular}


\newpage

\section{Introduction}

Chest X-rays are among the most commonly used diagnostic imaging techniques in clinical practice for evaluating respiratory and cardiac conditions. Despite their widespread use, interpreting chest X-ray images can be complex, time-consuming, and requires a high level of expertise. Misinterpretations can result in incorrect diagnoses and inappropriate treatment plans, leading to increased healthcare costs and potential harm to patients.

\noindent Healthcare professionals face significant challenges in accurately and efficiently diagnosing abnormalities from chest X-ray images. This highlights the urgent need for a reliable, efficient, and user-friendly AI/ML tool that can assist in interpreting chest X-ray images, reducing the burden on radiologists and improving diagnostic accuracy.

\noindent This section outlines the purpose of this Software Requirements Specification (SRS) document, the scope of the requirements, the characteristics of intended readers, stakeholders and the overall impact of the project.

\subsection{Purpose of Document}

The purpose of this SRS document is to provide a comprehensive foundation that guides the design, development, and implementation of the software system. The primary objectives of this document include:

\begin{itemize}
    \item[-] Establishing a shared understanding of the software requirements between stakeholders, developers, and other involved parties.
    \item[-] Defining the scope of the software project, including the functionalities that will be included and those that will be excluded.
    \item[-] Serving as a detailed guide for developers, outlining system constraints, properties of a correct solution, and both functional and non-functional requirements.
    \item[-] Facilitating effective communication among stakeholders and providing a framework to manage changes throughout the software development lifecycle.
\end{itemize}

\subsection{Scope of Requirements}

The scope of this project focuses on the analysis of chest X-ray images using AI/ML techniques to assist in the detection of diseases and abnormalities. The system is designed to accept chest X-ray images as input and provide diagnostic insights, including the identification of specific conditions or abnormalities. The key aspects of the project's scope include:

\begin{itemize}
    \item[-] Training Data: The system requires a diverse set of labeled training data, encompassing various disease categories and areas of abnormality. The datasets should represent a wide range of patient demographics, including age, gender, and geographic diversity, and should include examples of rare diseases.
    \item[-] Input Requirements: Input X-ray images must be high-resolution and captured using standard, certified imaging equipment to ensure consistency in analysis.
    \item[-] Specialization: The system is specifically designed for chest X-ray analysis and is not intended for interpreting other types of medical imaging.
    \item[-] Assistance, Not Automation: The system is intended to assist healthcare professionals by highlighting potential issues and supporting decision-making. It is not meant to replace radiologists or fully automate the diagnostic process.
    \item[-] Limitations: The system may have limitations in detecting certain rare or highly complex diseases that fall outside its training data. It does not integrate patient history, lab results, or other diagnostic tests that could be crucial for a comprehensive diagnosis. Additionally, the system is not designed to determine the underlying causes of detected conditions or predict long-term outcomes.
\end{itemize}

\noindent This scope ensures that the system remains focused on its core functionality while recognizing its limitations, thus providing realistic expectations for stakeholders and users.

\subsection{Characteristics of Intended Reader} \label{sec_IntendedReader}

The reader of this SRS document should possess an undergraduate level of software development skills, as it details the system constraints, functional requirements, and non-functional requirements. The reader should be familiar with software development theories, as the SRS includes sections on theoretical and instance models. Additionally, a basic understanding of project management is helpful for interpreting the development plan and the traceability matrix. Ideally, the reader should have a fundamental understanding of Artificial Intelligence, though this is not a requirement, as the SRS focuses primarily on system requirements rather than technical details. To comprehend input data types and other data constraints, the reader should have at least one year of programming experience.\\
\newpage
\subsection{Stakeholders}
\subsubsection{The Clients} 
This refers to Dr. Mehdi Moradi and his research team. Dr. Moradi, an Associate Professor in the Department of Computing and Software at McMaster University, has significant experience in the field of medical imaging and AI-based diagnostic tools. He is also an Associate Member in Medical Imaging, with a research focus on integrating advanced machine learning techniques into clinical applications. His work includes developing AI algorithms that can analyze medical images with high accuracy, which aligns closely with the goals of this project. His research on the Imagenome dataset has been instrumental in advancing the understanding of AI-based image interpretation, especially in radiology.
\begin{enumerate}
\item \textbf{Role:} Dr. Moradi and his team will serve as the main decision-makers for the scope, features, and overall system design of the AI solution. They will provide critical feedback on the system's development, ensuring that it meets clinical standards and aligns with real-world needs. Their expertise in interpreting complex imaging data will be crucial in refining the AI model's ability to accurately detect abnormalities in chest X-rays. Moreover, their guidance will help in addressing challenges such as data variability, imbalanced datasets, and ethical considerations like bias and fairness in AI predictions. Dr. Moradi’s experience with the clinical deployment of AI solutions will be vital for navigating regulatory hurdles, such as compliance with healthcare regulations, and ensuring the system's compatibility with existing medical imaging workflows.
\end{enumerate}
\subsubsection{The Customers:}
This refers to healthcare providers, patients, and medical facilities.
\newline
Healthcare providers include physicians, radiologists, doctors, etc. They are the primary users of the AI system, as the software will help them in interpreting chest X-rays more efficiently by detecting abnormalities and analyzing potential diseases. The use of this software as part of the diagnostic process will help improve efficiency and accuracy.

\begin{enumerate}
\item \textbf{Role:} They will use the AI system to aid in clinical diagnosis and decision-making. They will upload the X-Ray and interpret the result produced by the AI system, making changes or adjustments if necessary. They will read chest X-rays faster and more accurately with the aid of AI, reducing workload and helping to make decisions on priorities. They can also help point out if the AI system makes any mistakes, which helps it improve accuracy and cover rare diseases over time.
\end{enumerate}
Patients include people who have certain lung or cardiac diseases or those who think they might have one. They benefit from faster, more accurate diagnoses with improved healthcare accessibility. They also can get early detection of life-threatening diseases.

\begin{enumerate}
\item \textbf{Role:} Upload X-ray images and receive results from the AI system. Patients provide the chest X-rays that the AI system analyzes, which are critical for the AI to perform its diagnostic function. Patient data (X-rays) might be used to improve the AI system, so they are required to give consent for their X-ray images to be used in AI development.
\end{enumerate}
Medical facilities include hospitals and clinics. They can integrate the software into their hospital systems, which can help them to improve efficiency, reduce diagnostic errors, and manage costs.

\begin{enumerate}
\item \textbf{Role:} They will integrate the AI system into their existing infrastructure (radiology departments, diagnostic tools, etc.). They are also responsible for validating the AI system's performance in their environment. They will monitor how well the AI is diagnosing chest abnormalities and ensure that the results align with clinical expectations.
\end{enumerate}

\subsubsection{Other Stakeholders:}
This includes the technical support team and technicians who work for hospitals. They are responsible for ensuring seamless connectivity with existing systems, maintaining workflow, and ensuring data security and privacy.

\begin{enumerate}
\item \textbf{Role:} They help configure the AI system based on the needs of the medical facility. They ensure compatibility of the AI system with the hospital’s workflow and make necessary adjustments to optimize performance.
\end{enumerate}
\subsection{Personas}
\subsubsection{Physicians / Radiologists Persona:}

Dr. Emily is a senior radiologist with over 10 years of professional experience in medical imaging. She specializes in diagnosing complex lung / cardiac conditions through chest X-rays and other medical tests. Her typical workday is fast-paced, as she needs to review numerous images while collaborating with colleagues to ensure diagnoses are sent out on time. With her professional skills, Dr. Carter can identify ambiguous patterns in X-rays but the increasing workload and complexity of cases can sometimes make it challenging to maintain high-quality standards.\\
\newline
Dr.Emily is excited about how AI can help with diagnosis, especially with flagging and annotating potential abnormalities and providing help on decision-making, which she hopes to increase her overall working efficiency. However, Dr.Emily values transparency in X-ray results so she wants the AI to offer clear explanation instead of being a "black box" that makes decisions for her. She expects the AI system to integrate smoothly into her existing workflow with an intuitive user interface and not too much technology knowledge required. 

\subsubsection{Patient Persona:}
James is a construction site worker and a father of two who recently took a chest X-ray examination after experiencing consistent shortness of breath. As it takes around one week to get the result, he is anxious that it might be a serious condition and would like to know the results as soon as possible. He’s not familiar with medical imaging technology and doesn't know how would AI help with diagnosis. But he really wants to get an accurate and timely result regarding his health condition.\\
\newline
James hopes that AI can help catch potential abnormalities that might be missed by human errors. Privacy and security are also important to James, and he wants to be sure that his personal medical data is handled confidentially and safely. He doesn't want his personal data shared with unauthorized organizations. 

\subsubsection{Developer Persona:}
Alex is an AI developer with proficiency in pyTorch and Python and has some fundamental knowledge in machine learning. Alex is currently working with Dr. Steve on building and optimizing AI models for medical imaging, focusing specifically on chest X-ray analysis.\\
\newline
Alex is motivated to push the boundaries of AI in healthcare and he hope his contribution can make an impact in people's lives. Alex is primarily responsible for designing, training, and fine-tuning the AI models that support chest X-ray diagnosis. He also works closely with physicians and radiologists to get their feedback and improve the algorithms and interfaces. The main challenge for Alex is to ensure that the AI model produce high-accuracy results and explainable outputs at the same time for transparency.

\subsubsection{Researcher Persona:}
Dr. Steve is a postdoctoral researcher specializing in artificial intelligence and medical diagnostics. He focuses on advancing AI technologies that improve diagnostic accuracy and efficiency in radiology, particularly for chest X-rays.\\
\newline
His day-to-day work involves working on researches on AI applications in healthcare, collaborating with other academic institutions / medical facilities  to validate AI models in real-world scenarios. He hopes to contribute to the AI community in a meaningful way. However, he finds that challenging to ensure the transparency of AI model when communicating the results to users. Despite that, his continuous collaboration with physicians and other AI developers helps to ensure that his research has practical improvements for improving efficiency and accuracy in diagnosing chest-related diseases.

\subsection{Impact Analysis}

The primary benefit of this project lies in the ability to deliver preliminary diagnostic results quickly, thereby enhancing diagnostic efficiency and significantly reducing patient wait times. Moreover, the system’s capability for large-scale data classification can gradually improve the precision of analyses, reduce misdiagnoses, and alleviate the workload on medical professionals. When scaled effectively, this project has the potential to increase productivity within the medical sector, reduce healthcare costs, and accelerate service delivery. Additionally, by leveraging cloud-based deployment, this solution can extend diagnostic support to remote regions, providing preliminary assessments without the immediate need for in-person medical consultations.

\subsubsection{Economic Impact}
According to 2023 data from the Canadian government, the average annual salary for doctors is CAD 233,726 \cite{2}. With each chest X-ray analysis typically taking around 15 minutes and a wait time of 1 to 2 days for results \cite{1}, this AI tool has the potential to reduce analysis time to near-instantaneous levels, cutting costs by approximately CAD 30 per examination. Given that Canada conducts around 20 million X-ray examinations annually, with chest X-rays comprising about 6 to 8 million of these \cite{3}, the potential cost savings range from CAD 160 million to CAD 240 million annually. This economic benefit not only supports more efficient use of healthcare resources but also enables reinvestment into other critical areas of patient care.

\subsubsection{Health and Safety Impact}
From a health and safety perspective, the automated nature of this system can play a critical role in emergency situations. During the COVID-19 pandemic, the healthcare infrastructure faced unprecedented pressure, with a widespread shortage of medical professionals \cite{4}. In such scenarios, AI-assisted tools like this could be deployed to manage and classify X-ray data, significantly improving processing speed and relieving the burden on healthcare providers. This allows for more patients to receive timely care, which is crucial during high-demand periods.
\newline
However, deploying AI for disease diagnosis involves considerations of accuracy and reliability. AI models, while efficient, produce probabilistic outcomes rather than absolute determinations. This inherent uncertainty can introduce risks, including ethical dilemmas and potential legal challenges, particularly if a misdiagnosis occurs \cite{5}. Thus, while the technology can accelerate diagnosis, integrating human oversight remains essential to mitigate these risks and ensure patient safety.

\subsubsection{Cultural Impact}
The introduction of AI-based diagnostic systems, such as this project, can influence the cultural landscape of healthcare delivery. In some cultures, there is a strong preference for direct consultations with healthcare professionals, where personal interaction and trust in human expertise are preferred. The adoption of AI in these contexts may require a cultural shift as patients adapt to receiving initial assessments through automated systems. However, in regions with limited access to healthcare professionals, such as remote or underserved areas, AI-based diagnostics can be perceived as an empowering tool. It offers a level of autonomy and timely access to healthcare information that was previously unavailable \cite{6}. This dual impact reflects the need to balance technological advancements with cultural sensitivity.

\subsubsection{Overall Societal Impact}
This project, when implemented at scale, has the potential to transform the delivery of healthcare. By reducing analysis time and increasing access to diagnostic support, it addresses critical issues such as the strain on healthcare systems and accessibility of medical services in remote areas. The use of cloud technology further ensures that the system can be deployed across a range of settings, from urban hospitals to rural clinics, providing consistent support regardless of location. As society increasingly embraces digital solutions in healthcare, this project stands to play a significant role in reshaping diagnostic practices, making them faster, more efficient, and more accessible.

\section{General System Description}

This section provides general information about the system.  It identifies the
interfaces between the system and its environment, describes the user
characteristics and lists the system constraints. \\

\subsection{System Context}
 
The below system context diagram shows that the system will take in an input(chest X-ray image) from the user after they login. The user upload the chest X-ray image to the system and then system will annotate and provide the interpretation of the X-ray image back to the user.



\begin{figure}[h!]
  \centering
  \includegraphics[width=1\textwidth]{SystemContext.png}
  \caption{Neuralanalyzer's System Context Diagram}
  \label{fig:overall-infra}
\end{figure}



\begin{itemize}
\item User Responsibilities:
\begin{itemize}
\item {Need to ensure that the chest X-ray image is of high resolution. }
\item {Need to ensure that the chest X-ray image is a chest X-ray image, not other types of medical images.}
\item {Need to ensure that the chest X-ray image taken by standard machines. Images with poor lighting, noises or does not follow standard protocols will not be accepted. }
\end{itemize}
\item Program Responsibilities:
\begin{itemize}
\item {Detect abnormalities / diseases in the chest X-ray images.}
\item {Classify abnormalities / diseases into specific categories.}
\item {Quantify size / degree / volume of abnormalities.}
\item {Produce report with possible diagnosis and the above information.}
\end{itemize}
\end{itemize}
The software is typically used in medical and clinical contexts to support healthcare professionals and physicians to speed up X-ray readings and improve accuracy in diagnosis. It is a safety critical application as failure could lead to delay in treatment, misdiagnosis and potentially severe outcomes for the patient.
\subsection{Behavior Overview}
\begin{figure}[H] 
  \centering
  \includegraphics[width=1\linewidth]{UseCase.png}
  \caption{Use Case Diagram of AI for chest X-ray read}
  \label{fig:use-case-diagram}
\end{figure}
\vspace{-5em} 
\newpage
\subsubsection{Normal Operation}
As observed in the use case diagram, the system begins with user input from either patients or radiologists, who upload chest X-ray images to be analyzed. These images can be stored locally or in the cloud for easy access. Once uploaded, the system processes the images using machine learning algorithms to detect abnormalities, annotate findings, and generate diagnostic reports. These reports include detailed analysis results and suggested diagnostic advice, which radiologists can review, edit, and compare with previous records if needed.The system also enables radiologists to refine patient treatment plans based on the AI-generated insights, ensuring a smooth and accurate workflow from image upload to delivering meaningful medical recommendations.
\subsubsection{Undesired Event Handling}
If an undesired event occurs during the operation of the system, it will transition into an error state. In this state, no further data processing will occur to prevent the possibility of using corrupted or inaccurate information, ensuring the integrity of the user's input. A clear error message will be provided to inform the user about the issue, helping them understand the cause and potentially prevent similar occurrences in future operations. To resume normal functionality, the user will need to restart the process and re-upload their data, ensuring that the undesired event does not impact the final results.

\subsection{User Characteristics} \label{SecUserCharacteristics}
Physicians / Radiologists: 
\begin{enumerate}
 \item They expect the AI system to help them in improving diagnostic accuracy, reducing daily workload, and identify abnormalities that are hard to notice.
 \item They want the system to be reliable and precise.
 \item They need an intuitive and simple interface that integrates smoothly with their current workflow without advanced technical knowledge to use the AI system.
 \item They need clear explanations of AI outputs, such as why certain abnormalities are marked. This would help them incorporate AI into their decision-making with more confident.
 \item They need transparency in how the AI arrives at its conclusions.
 \end{enumerate}
 Patients:
 \begin{enumerate}
 \item They expect their data to be handled securely and confidentially by the AI system.
 \item They would like to reduce the chance of missed diagnoses and get faster treatments.
 \item May not completely trust in the results of the AI system. They may rely on professionals (doctors and radiologists) to explain how the AI system arrives to the conclusion.
 \end{enumerate}
Developers:
 \begin{enumerate}
 \item They has deep technical expertise as they work on creating and optimizing the AI system.
 \item They need access to feedback from radiologists and physicians, as well as patient data and case studies. 
 \item They has the ability to update the system on a regular basis, adding new features and incorporating new data based on other user's feedback.
 \end{enumerate}
Hospitals and Clinics:
 \begin{enumerate}
 \item May not interact with the AI system directly, but are interested in the performance (number of images processed, diagnostic accuracy rates, and the time saved etc). 
 \item They focus on the operational efficiency of the AI system that improve overall workflow efficiency.
 \item They focus on if the AI system follows regulatory and clinical standards. 
 \end{enumerate}
\subsection{System Constraints}

\subsubsection{Government Regulations}
The system must comply with healthcare regulations and obtain approvals from bodies such as the FDA, adhering to PHIPA and other certifications before deployment. Compliance is critical, as non-compliance could result in legal consequences, fines, or restrictions on operation in certain regions.

\subsubsection{Data Usage Consent}
The system must respect patients' consent for data usage, especially under regulations like PIPEDA in Canada. In some regions, even anonymized data may require explicit patient approval, which can limit available training datasets. Additionally, even anonymized data may fall under these regulations due to the risk of re-identification using advanced techniques. Non-compliance could lead to severe penalties, including fines, lawsuits, or suspension of the service.

\subsubsection{Cultural Sensitivity}
Different regions and cultures may have varying levels of acceptance toward AI in healthcare, which can impact the system's deployment. Some countries may require human oversight for every AI-generated decision. Understanding these differences is crucial for identifying target markets and tailoring the rollout strategy to areas more receptive to AI-driven healthcare solutions.

\subsubsection{Budget Limitations in Hospitals}
The application may face constraints due to limited budgets in some healthcare organizations, requiring it to be affordable and sustainable. This limitation could restrict access to advanced hardware or impact the system's speed and performance. It is vital to ensure the application is lightweight and can leverage available resources from the hosting hospital to ensure functionality.

\subsubsection{Legal Liability}
In cases of misdiagnosis, the application's legal responsibility may vary by region, necessitating protections or disclaimers. Obtaining patient consent and providing clear disclaimers are essential steps to mitigate risks of fines or legal action against the company.

\subsubsection{Compatibility with Older Equipment}
The system may need to integrate with older X-ray machines, work with limited computational resources, or function with restricted internet bandwidth in certain hospitals. These constraints are often non-negotiable due to existing infrastructure, and the system must be designed to function effectively under these conditions. Ensuring compatibility is critical, as the system relies on input from X-ray images to operate effectively.


\section{Specific System Description}
This section first presents the problem description, which gives a high-level
view of the problem to be solved.  This is followed by the solution characteristics
specification, which presents the assumptions, theories, definitions and finally
the instance models.

\subsection{Problem Description}
Disease prediction via medical imaging continues to be one of the most important aspect of health-care. However, the health-care industry is still slowly catching up with current technology, making disease prediction and prevention processes labour-intensive and inefficient. With a prolong history in operation, countless images has been captured from CT Scanners and X-Ray machines, for medical and health-care research, but yet to be applied in hospital settings between physicians and patients.
\newline
Our team, Neuralanalyzer, saw an opportunity to transform traditional health-care processes, by applying Machine Learning, and modern software engineering techniques; aiming to provide an optimized, unified, and HIPAA-compliant platform to predict diseases from chest X-Ray scans.
\newpage
\subsubsection{Goal Statements}
\noindent Given the inputs and the constraints of the system, the goal statements are:

\begin{itemize}

\item[GS\refstepcounter{goalnum}\thegoalnum \label{G_prediction_accuracy}:] \textbf{High Prediction Accuracy: }Machine Learning Model used by Neuralanalyzer must have a high prediction accuracy for Chest X-Ray images. We are aiming for a minimum of 75\% accuracy.
\item[GS\refstepcounter{goalnum}\thegoalnum \label{G_ease_of_ML_intro}:] \textbf{Simple to introduce more ML models: }New classifier models must be simple and intuitive to be integrated with the system, with short time-to-market. This goal will require a containerized ML-plugin code template (in Python) and guidelines documentations for new models to follow.
\item[GS\refstepcounter{goalnum}\thegoalnum \label{G_fault-tolerance}:] \textbf{High Fault-Tolerance: }Neuralanalyzer's system (front-end, back-end, ML-plugins) must handle run-time errors as clever as possible, to prevents run-time crashes, and unwanted downtime. Comprehensive unit and integration testings must be present, to reduce errors and bugs within the system.
\item[GS\refstepcounter{goalnum}\thegoalnum \label{G_good_UI}:] \textbf{Intuitive UI/UX: }Neuralanalyzer's front-end must be user-friendly and straight-forward to be used. Front-end's core features such as login, account registration, and image-prediction should require as few clicks as possible.
\item[GS\refstepcounter{goalnum}\thegoalnum \label{G_dev_work_automation}:] \textbf{High automation of Dev-workflow: }Repetitive, and crucial development workflows must be implemented within Github Actions. Example including: Unit \& Integration Testing, Linting Check, Docker Image build, Deployment on AWS.
\item[GS\refstepcounter{goalnum}\thegoalnum \label{G_avaialblity}:] \textbf{High Availability: }Neuralanalyzer application must be highly available, with very minimal downtime. Our team aiming to have our targeted SLA as 99\%.
\item[GS\refstepcounter{goalnum}\thegoalnum \label{G_data_security}:] \textbf{Secure Data Transfer \& Hosting: }Image-related operations, such as storing, transferring via front-end, ML-training must be authenticated, authorized, and encrypted. This requires an introduction of a authentication service, which our team will be using AWS Cognito.

\end{itemize}
\newpage
\subsection{Solution Characteristics Specification}
\subsubsection{Overall Solution Architecture}
\begin{figure}[h!]
  \centering
  \includegraphics[width=1\textwidth]{app-runtime.png}
  \caption{Neuralanalyzer's Run-Time AWS Infrastructure}
  \label{fig:overall-runtime-infra}
\end{figure}

\begin{itemize}
    \item \textbf{API Gateway}: Act as an ingress service, responsible for routing the requests to the Layer 7 Load Balancer. All requests forwarded to the load-balancer must be authenticated via \textbf{AWS Cognito}.
    \item \textbf{AWS Cognito}: Act as an authorization service, responsible for logins, registers, and reset passwords.
    \item \textbf{Layer 7 Load Balancer}: Responsible for load balance Layer 7 traffic (HTTP in this case) on the the back-end service.
    \item \textbf{Back-end Application}: Responsible for routing the requests to the correct ML services.
    \item \textbf{ML Services}: Responsible for returning a prediction based on an image passed-in by the back-end service.
    \item \textbf{Open-source ML Service}: Act as a back-up if the other services failed, responsible for returning a prediction based on an image passed-in by the back-end service.
    \item \textbf{AWS Cloudwatch}: Act as a run-time monitoring service, responsible for storing logs, and alert the developers if any defined errors/ metrics has occurred.
\end{itemize}

\subsubsection{Overall Continuous Development and Delivery Pipeline}

\begin{figure}[h!]
  \centering
  \includegraphics[width=1\textwidth]{app-build-pipeline.png}
  \caption{Neuralanalyzer's CI/CD Pipeline}
  \label{fig:overall-ci-cd-infra}
\end{figure}

\begin{itemize}
    \item \textbf{Github Actions}: Triggered every time the code is merge onto the main-branch of our Github repository.
    \item \textbf{AWS ECR}: Storing the versions of Neuralanalyzer application Docker Images. A new docker image will be tag as latest, after new source code is ready on Github.
    \item \textbf{AWS ECS}: Orchestrate the run-time of the application containers.
    \item \textbf{AWS Fargate}: The live containers that contains application and business logic of Neuralanalyzer.
\end{itemize}


\subsubsection{Architectural Decision Records}
\begin{itemize}
\item ADR\refstepcounter{adrnum}\theadrnum \label{ADR_AWS_for_prod}: Usage of Amazon Web Services.
    \begin{itemize}
        \item \textbf{Rationale}: Our team has chosen to go with the popular AWS platform to host Neuralanalyzer.
        \item \textbf{Benefits}:
        \begin{itemize}
            \item AWS is HIPPA compliance, making our application hosted on AWS HIPAA-compliance by default.
            \item AWS is a mature platform with large catalog of services, and support from both the AWS team, and the community around it.
            \item Team Members are familiar with AWS, making the development velocity straght-forward, and speedy.
        \end{itemize}
        \item \textbf{Drawbacks}:
        \begin{itemize}
            \item AWS can get expensive if not being utilized correctly.
            \item Some functionalities (such as Authentication on AWS Cognito) are AWS-native, which encourage vendor-locked, and will require developer effort to move off AWS later on. 
        \end{itemize}
    \end{itemize}
    
    \item ADR\refstepcounter{adrnum}\theadrnum \label{ADR_splitting_AI_from_backend}: Splitting up the ML-Models from the core Neuralanalyzer back-end service.
    \begin{itemize}
        \item \textbf{Rationale}: Since the Machine Learning Models only serve the purpose of classifying, we refactored the models out as it's own services, allowing the back-end to handle authentication, and logic coordination.
        \item \textbf{Benefits}:
        \begin{itemize}
            \item Allowing separate development and deployment of AI-Models, and other application logic (authentication, databases, etc.)
            \item Enabling the ability to connect multiple models, via plugins, onto the system, providing Neuralanalyzer with more flexibility.
        \end{itemize}
        \item \textbf{Drawbacks}:
        \begin{itemize}
            \item Increase complexity in managing source code, and at run-time. For example: runtime security, availability, service discovery, etc.
        \end{itemize}
    \end{itemize}
    
    \item ADR\refstepcounter{adrnum}\theadrnum \label{ADR_train_on_cas}: Training Neuralanalyzer ML Model on CAS Department's servers.
    \begin{itemize}
        \item \textbf{Rationale}: Training ML Models can be costly, through a quick calculation, training our in-house ML Model on AWS (with the lowest settings) for 10 hours can cost minimum of 150 dollars. For research and development purposes, we decided to kick-start the training process on CAS's Department servers.
        \item \textbf{Benefits}:
        \begin{itemize}
            \item Massively speeds up research and development since department's servers are already set-up with student's MacID.
            \item Reducing cost for AI-Model training.
        \end{itemize}
        \item \textbf{Drawbacks}:
        \begin{itemize}
            \item This separate training process makes this particular development workflow hard and time-consuming to automate.
        \end{itemize}
    \end{itemize}
    
    \item ADR\refstepcounter{adrnum}\theadrnum \label{ADR_torchxray_as_backup}: Including TorchXRayVision as a back-up model.
    
    \begin{itemize}
        \item \textbf{Rationale}: To negating the risk of our in-house ML Models not working as expected, using a stable open-source project to back up the core-functionality of the application can drastically improve the fault-tolerance of the system.
        \item \textbf{Benefits}:
        \begin{itemize}
            \item Ability to quickly create a working and stable MVP, which contains the front-end, back-end, and the ML model.
            \item Ability to apply transfer-learning on to this model to fine-tune its accuracy.
            \item Ability to change the source code of this model to fit's our team's need.
        \end{itemize}
        \item \textbf{Drawbacks}:
        \begin{itemize}
            \item Un-ability to re-train the model.
            \item Out-of-the-box model has low accuraccy score.
        \end{itemize}
    \end{itemize}
\end{itemize}

\subsubsection{Input Data Constraints} \label{sec_DataConstraints} 
From our math formulation of the machine learning plugin, the input data will be in a form of an image binary, of size 224x224 pixels. The image being passed in into the ML model will be automatically transformed into gray-scale. Example below:
\begin{center}
    \includegraphics[scale=0.3]{chest-x-ray.png}
\end{center}

\subsubsection{Properties of a Correct Solution} \label{sec_CorrectSolution}
In the context of our machine learning model, the absolute correct solution can not be calculated at run-time. However, A solution is considered good enough when during in the training and validation process, the error rate of the system has been reduced to an agreeable number before bringing it to production.
\newpage
\subsection{Assumptions and Relevant Facts}
\subsubsection{Legal and Regulatory Facts} \label{sec_legal}
This project must comply with critical legal and regulatory standards to ensure the secure handling of medical data and the development of software classified as a medical device. In the United States, the Health Insurance Portability and Accountability Act (HIPAA) mandates strict protections for handling Personal Health Information (PHI), with non-compliance potentially resulting in fines of up to \$1.5 million per violation \cite{10}. Similarly, in Canada, the Personal Information Protection and Electronic Documents Act (PIPEDA) governs the handling of personal data in commercial activities, with possible penalties reaching up to CAD 100,000 for non-compliance \cite{2}. For cloud-based storage solutions like those provided by Amazon Web Services (AWS), adherence to their shared responsibility model is necessary, where AWS ensures the security of the cloud infrastructure while users manage the security of their data within the cloud \cite{7}. Additionally, the project must adhere to ISO 13485 standards for medical device software, which ensure quality management and regulatory compliance throughout the development and testing phases \cite{8}. Using the Digital Imaging and Communications in Medicine (DICOM) standard also ensures compatibility with Picture Archiving and Communication Systems (PACS), facilitating standardized exchange and storage of medical imaging data \cite{9}.
\subsubsection{Assumptions} \label{sec_assumpt}
\begin{itemize}

\item[A\refstepcounter{assumpnum}\theassumpnum \label{A_meaningfulLabel}:] ML Models validation process will have to be \textbf{publicly published and accepted by all stakeholders} after being developed. The reason is that the accuracy of the ML model is vital for the safety of the patient, as well as the success of the project. Therefore, this validation method needs to be transparent to all stakeholders and users.

\item[A\refstepcounter{assumpnum}\theassumpnum \label{A_meaningfulLabel}:] AWS will share the responsibility with Neuralanalyzer to protect the customer PHI in a law-compliant manner, as mentioned in this \href{https://docs.aws.amazon.com/whitepapers/latest/aws-risk-and-compliance/shared-responsibility-model.html}{shared-responsibility blog post}.

\item[A\refstepcounter{assumpnum}\theassumpnum \label{A_meaningfulLabel}:] McMaster Department's servers will be available for our group members to perform model training and development. Under another assumption that our group members, as users of department resources, must follow department guidelines.

\item[A\refstepcounter{assumpnum}\theassumpnum \label{A_meaningfulLabel}:] It is assumed that the training data set used for model development will include diverse and representative samples to ensure the model’s robustness and generalization across different populations.
\end{itemize}

  \clearpage
\section{Requirements}
\subsection{Formal Mathematical Specification}

\subsubsection{Formal Definitions}
\begin{itemize}
  \item Let \( X = \{x_1, x_2, \ldots, x_n\} \) be the set of chest X-ray images, where \( x_i \) represents each individual image.
  \item Let \( Y = \{y_1, y_2, \ldots, y_n\} \) be the set of labels, where \( y_i \in \{0, 1\} \). Here, \( y_i = 1 \) indicates the presence of a disease, and \( y_i = 0 \) indicates its absence.
  \item Let \( M \) be the machine learning model used for disease detection.
  \item Let \( \theta \) represent the parameters of model \( M \), which are learned during training.
  \item Let \( P(x_i) \) represent the probability that an image \( x_i \) is classified as containing a disease by the model \( M \).
\end{itemize}

\subsubsection{Formal Expressions}
\begin{enumerate}
  \item \textbf{Training the Model:}
    \begin{align*}
      \theta^* &= \text{arg min}_\theta \sum_{i=1}^n (M(x_i; \theta) - y_i)^2 \\
      &\text{where } M(x_i; \theta) \text{ is the predicted value for } x_i.
    \end{align*}
    The model parameters \( \theta \) are adjusted during training to minimize the difference between the predicted values and the actual labels.

  \item \textbf{Prediction:}
    \begin{align*}
      \hat{y}_i &= \begin{cases}
        1 & \text{if } P(x_i) \geq 0.85 \\
        0 & \text{if } P(x_i) < 0.85
      \end{cases}
    \end{align*}
    The model predicts \( \hat{y}_i = 1 \) (disease present) if the probability \( P(x_i) \) is greater than or equal to 0.85, and \( \hat{y}_i = 0 \) (disease absent) otherwise.

  \item \textbf{Evaluation:}
    \begin{align*}
      \text{Accuracy} &= \frac{\sum_{i=1}^n \mathbb{I}(\hat{y}_i = y_i)}{n} \\
      &\text{where } \mathbb{I}(\hat{y}_i = y_i) \text{ equals 1 if the prediction is correct, otherwise 0.}
    \end{align*}
    Accuracy measures the proportion of images for which the model's predictions match the actual labels.

  \item \textbf{Loss Function:}
    \begin{align*}
      L(\theta) &= \sum_{i=1}^n (M(x_i; \theta) - y_i)^2
    \end{align*}
    The loss function measures how far the model's predictions are from the actual labels. The model aims to minimize this function during training.
\end{enumerate}

\subsubsection{Model Constraints}
\begin{itemize}
  \item The model \( M \) should achieve an accuracy of at least 85\% on the validation dataset.
  \item Each input image \( x_i \) should be preprocessed to a fixed size (e.g., 224x224 pixels) to ensure compatibility with the model.
  \item The model should be trained on a diverse dataset that includes images from different demographics and disease categories.
\end{itemize}
\subsection{Finite State Machine Diagram}
% addd the state.png file under finite state diagram subsection
\begin{figure}[H]
    \centering
    \includegraphics[width=1\linewidth]{statem.png}
    \caption{Finite State Machine Diagram}
    \label{fig:enter-label}
\end{figure}

\newpage
\subsection{Functional Requirements}
\begin{table}[h!]
\centering
\rowcolors{2}{white}{white} % All rows except the first will be white
\begin{tabular}{|p{3.5cm}|p{11.5cm}|}
\hline
\rowcolor{gray!30} % Light gray for the first row (header)
\textbf{FR1} & The system shall accept chest X-ray images as input from authorized users, including healthcare professionals and patients. \\
\hline
\textbf{Rationale} & To perform analysis, the system requires chest X-ray images to be uploaded by users in supported formats (DICOM, JPEG, PNG). \\
\hline
\textbf{Verification} & Test the upload functionality with different image formats and ensure the system can process the images without errors. \\
\hline
\textbf{Priority} & High \\
\hline
\textbf{Traceability} & LF1, LF2, UH1, FR2 (for input handling), FR7 (data storage) \\
\hline
\end{tabular}
\caption{Functional Requirement FR1}
\centering
\rowcolors{2}{white}{white}
\begin{tabular}{|p{3.5cm}|p{11.5cm}|}
\hline
\rowcolor{gray!30}
\textbf{FR2} & The system shall enable users to input additional patient symptoms, such as cough, chest pain, or fever. \\
\hline
\textbf{Rationale} & Including patient symptoms helps provide more context for the AI model to enhance disease detection accuracy. Our goal is to make the AI more useful by giving it more information to work with. \\
\hline
\textbf{Verification} & Verify that users can enter patient symptoms and that they are correctly linked to the associated X-ray images. \\
\hline
\textbf{Priority} & Medium \\
\hline
\textbf{Traceability} & UH1, UH2, MS1, FR1 (input dependency), FR6 (reporting of symptoms) \\
\hline
\end{tabular}
\caption{Functional Requirement FR2}
\end{table}
\begin{table}[h!]
\centering
\rowcolors{2}{white}{white}
\begin{tabular}{|p{3.5cm}|p{11.5cm}|}
\hline
\rowcolor{gray!30}
\textbf{FR3} & The system shall analyze chest X-ray images to detect the presence or absence of specific diseases with an accuracy of 85\% or higher. \\
\hline
\textbf{Rationale} & One of the core functionalities of our system is its ability to detect diseases like pneumonia from X-rays. This ensures the AI model is effective in clinical settings. \\
\hline
\textbf{Verification} & Evaluate the accuracy of the disease detection model against a test dataset and confirm it meets or exceeds the 85\% accuracy threshold. \\
\hline
\textbf{Priority} & High \\
\hline
\textbf{Traceability} & PR1, HS1, SR1, FR1 (input of images), FR6 (reporting) \\
\hline
\end{tabular}
\caption{Functional Requirement FR3}
\end{table}
\begin{table}[h!]
\centering
\rowcolors{2}{white}{white}
\begin{tabular}{|p{3.5cm}|p{11.5cm}|}
\hline
\rowcolor{gray!30}
\textbf{FR4} & The system shall indicate whether a patient's condition has improved, worsened, or remained stable between scans. \\
\hline
\textbf{Rationale} & This feature helps clinicians monitor disease progression or regression, which is vital for evaluating the effectiveness of treatments over time.\\
\hline
\textbf{Verification} & Verify that the system provides a status of the patient's condition after comparing multiple scans. \\
\hline
\textbf{Priority} & High \\
\hline
\textbf{Traceability} & PR1, OE2, HS1, FR3 (analysis results), FR8 (alerts) \\
\hline
\end{tabular}
\caption{Functional Requirement FR4}
\end{table}
\begin{table}[h!]
\centering
\rowcolors{2}{white}{white}
\begin{tabular}{|p{3.5cm}|p{11.5cm}|}
\hline
\rowcolor{gray!30}
\textbf{FR5} & The system shall generate visual aids by highlighting affected areas on the chest X-ray images. \\
\hline
\textbf{Rationale} & By highlighting affected regions, our system makes it easier for clinicians to identify the areas of concern in a chest X-ray. This improves their ability to act quickly. \\
\hline
\textbf{Verification} & Ensure that the system overlays the appropriate markers on the X-ray images to highlight abnormal areas. \\
\hline
\textbf{Priority} & Medium\\
\hline
\textbf{Traceability} & PR1, PR3, LF1, FR3 (analysis results), FR6 (report visualization) \\
\hline
\end{tabular}
\caption{Functional Requirement FR5}
\end{table}
\begin{table}[h!]
\centering
\rowcolors{2}{white}{white}
\begin{tabular}{|p{3.5cm}|p{11.5cm}|}
\hline
\rowcolor{gray!30}
\textbf{FR6} & The system shall produce a structured, human-readable report summarizing key findings, disease detection results, and progression status. \\
\hline
\textbf{Rationale} & A comprehensive report makes it easier for clinicians to interpret the analysis results, especially in fast-paced environments where time is critical. \\
\hline
\textbf{Verification} & Verify that the system generates a report containing all necessary details in an organized format. \\
\hline
\textbf{Priority} & High \\
\hline
\textbf{Traceability} &  LF1, UH2, HS1, FR3 (analysis), FR4 (condition status). \\
\hline
\end{tabular}
\caption{Functional Requirement FR6}
\end{table}
\begin{table}[h!]
\centering
\rowcolors{2}{white}{white}
\begin{tabular}{|p{3.5cm}|p{11.5cm}|}
\hline
\rowcolor{gray!30}
\textbf{FR7} & The system shall store patient data, including images and reports, in a secure database for future reference. \\
\hline
\textbf{Rationale} & Keeping patient records secure and accessible for future consultations or research is important for patient care continuity. \\
\hline
\textbf{Verification} & Verify that the system correctly stores and retrieves patient data from the database. \\
\hline
\textbf{Priority} & Medium \\
\hline
\textbf{Traceability} & PR2, PR3, SR1, LR2, FR1 (data input), FR6 (reporting) \\
\hline
\end{tabular}
\caption{Functional Requirement FR7}
\end{table}

\begin{table}[h!]
\centering
\rowcolors{2}{white}{white}
\begin{tabular}{|p{3.5cm}|p{11.5cm}|}
\hline
\rowcolor{gray!30}
\textbf{FR8} & The system shall provide alerts to clinicians if significant changes in a patient's condition are detected between scans. \\
\hline
\textbf{Rationale} & Automated alerts can save time and ensure that urgent cases are handled quickly, improving patient outcomes. \\
\hline
\textbf{Verification} & Test that alerts are generated when the system detects a significant change in disease severity or condition. \\
\hline
\textbf{Priority} & Low \\
\hline
\textbf{Traceability} & PR1, SR2, FR4 (condition changes), FR6 (reporting) \\
\hline
\end{tabular}
\caption{Functional Requirement FR8}
\end{table}
\begin{table}[h!]
  \centering
  \rowcolors{2}{white}{white}
  \begin{tabular}{|p{3.5cm}|p{11.5cm}|}
  \hline
  \rowcolor{gray!30}
  \textbf{FR9} & The system shall allow healthcare professionals to adjust treatment plans based on the X-ray analysis results. \\
  \hline
  \textbf{Rationale} & Allowing clinicians to modify treatment plans based on X-ray results helps to tailor patient care more effectively. This ensures that the system supports decision-making in real-time clinical settings. \\
  \hline
  \textbf{Verification} & Verify that healthcare professionals can modify treatment plans directly in the system, and that these changes are linked to the corresponding X-ray analysis results. \\
  \hline
  \textbf{Priority} & High \\
  \hline
  \textbf{Traceability} & MS1, SR2, FR6 (reporting adjustments), FR7 (data updates) \\
  \hline
  \end{tabular}
  \caption{Functional Requirement FR9}
  \end{table}

\begin{table}[h!]
\centering
\rowcolors{2}{white}{white}
\begin{tabular}{|p{3.5cm}|p{11.5cm}|}
\hline
\rowcolor{gray!30}
\textbf{FR10} & The system shall allow integration with electronic health record (EHR) systems to import and export patient data. \\
\hline
\textbf{Rationale} & Integrating with existing EHR systems improves workflow efficiency and reduces the need for manual data entry, making the system more useful in a clinical setting. \\
\hline
\textbf{Verification} & Test the system’s ability to connect with standard EHR systems and exchange data using appropriate protocols. \\
\hline
\textbf{Priority} & Low \\
\hline
\textbf{Traceability} & OE1, MS1, LR2, FR7 (data storage), FR9 (treatment adjustment) \\
\hline
\end{tabular}
\caption{Functional Requirement FR10}
\end{table}
\begin{table}[h!]
\centering
\rowcolors{2}{white}{white}
\begin{tabular}{|p{3.5cm}|p{11.5cm}|}
\hline
\rowcolor{gray!30}
\textbf{FR11} & The system shall display confidence levels for disease detection and progression analysis results through an intuitive user interface that requires minimal training to operate. \\
\hline
\textbf{Rationale} & Providing confidence levels helps clinicians assess the reliability of the results. \\
\hline
\textbf{Verification} & Check that the system displays a confidence score alongside each analysis result. \\
\hline
\textbf{Priority} & High \\
\hline
\textbf{Traceability} & UH2, PR1, FR3 (analysis results), FR5 (visual aids) \\
\hline
\end{tabular}
\caption{Functional Requirement FR11}
\end{table}

\begin{table}[h!]
\centering
\rowcolors{2}{white}{white}
\begin{tabular}{|p{3.5cm}|p{11.5cm}|}
\hline
\rowcolor{gray!30}
\textbf{FR12} & The system shall allow patients to upload their own chest X-ray images for self-diagnosis purposes, subject to appropriate disclaimers. \\
\hline
\textbf{Rationale} & Allowing patients to upload their own images increases engagement in their healthcare journey, though it must be made clear that professional interpretation is still necessary. \\
\hline
\textbf{Verification} & Ensure that patients can upload images and that the system displays appropriate disclaimers before analysis. \\
\hline
\textbf{Priority} & Low \\
\hline
\textbf{Traceability} & SR1, SR2, HS2, FR1 (image upload), FR13 (role management) \\
\hline
\end{tabular}
\caption{Functional Requirement FR12}
\end{table}

\begin{table}[h!]
\centering
\rowcolors{2}{white}{white}
\begin{tabular}{|p{3.5cm}|p{11.5cm}|}
\hline
\rowcolor{gray!30}
\textbf{FR13} & The system shall support multiple user roles with appropriate access levels (e.g., physician, radiologist, patient). \\
\hline
\textbf{Rationale} & Different users need access to different features and data, so assigning roles ensures that each user can only perform actions they are authorized to. \\
\hline
\textbf{Verification} & Verify that users are assigned roles and that their access is restricted to the appropriate features based on their role. \\
\hline
\textbf{Priority} & High \\
\hline
\textbf{Traceability} &  SR2, LR1, FR7 (data storage), FR12 (patient data access) \\
\hline
\end{tabular}
\caption{Functional Requirement FR13}
\end{table}
\begin{table}[h!]
\centering
\rowcolors{2}{white}{white}
\begin{tabular}{|p{3.5cm}|p{11.5cm}|}
\hline
\rowcolor{gray!30}
\textbf{FR14} & The system shall create a new copy of a patient’s X-ray before running the AI model for analysis. \\
\hline
\textbf{Rationale} & Creating a new copy ensures that the original X-ray remains unchanged for future reference, while the AI model works on the new version for tagging and analysis. \\
\hline
\textbf{Verification} & Verify that the system creates a new X-ray copy, preserving the original file while the AI model runs on the duplicate. \\
\hline
\textbf{Priority} & High \\
\hline
\textbf{Traceability} & MS2, OE1, FR1 (image upload), FR3 (analysis) \\
\hline
\end{tabular}
\caption{Functional Requirement FR14}
\end{table}
\begin{table}[h!]
\centering
\rowcolors{2}{white}{white}
\begin{tabular}{|p{3.5cm}|p{11.5cm}|}
\hline
\rowcolor{gray!30}
\textbf{FR15} & The system shall support additional medical imaging modalities, such as CT scans and MRIs, for comprehensive analysis. \\
\hline
\textbf{Rationale} & Supporting multiple imaging types expands the system's utility in various clinical scenarios. \\
\hline
\textbf{Verification} & Test the system's ability to accept and analyze different image types (e.g., CT, MRI) without errors and provide accurate results. \\
\hline
\textbf{Priority} & Low \\
\hline
\textbf{Traceability} & PR3, LR2, FR1 (image input), FR7 (data storage) \\
\hline
\end{tabular}
\caption{Functional Requirement FR15}
\end{table}

\begin{table}[h!]
\centering
\rowcolors{2}{white}{white}
\begin{tabular}{|p{3.5cm}|p{11.5cm}|}
\hline
\rowcolor{gray!30}
\textbf{FR16} & The system shall support regular updates to the AI model to incorporate new data and improve accuracy over time. \\
\hline
\textbf{Rationale} & Continuous learning enhances the model's performance and keeps it up-to-date with the latest medical knowledge. \\
\hline
\textbf{Verification} & Ensure that the system can accept and integrate updated models without disrupting service. \\
\hline
\textbf{Priority} & Medium \\
\hline
\textbf{Traceability} & MS1, MS2, FR3 (analysis), FR7 (data storage) \\
\hline
\end{tabular}
\caption{Functional Requirement FR16}
\end{table}  
\clearpage                                                                                                                                                    

\subsection{Non-Functional Requirements}

\subsubsection{Look and Feel Requirements}

\begin{table}[h!]
\centering
\rowcolors{2}{white}{white}
\begin{tabular}{|p{3.5cm}|p{11.5cm}|}
\hline
\rowcolor{gray!30}
\textbf{LF1} & The system shall have a user interface consistent with standard medical imaging software, using familiar layouts and terminology. \\
\hline
\textbf{Rationale} & Radiologists and healthcare professionals are accustomed to specific interface designs; maintaining consistency reduces the learning curve. \\
\hline
\textbf{Fit Criterion} & At least 90\% of radiologists surveyed find the interface intuitive and comparable to existing medical imaging software. \\
\hline
\textbf{Traceability} & FR1, FR5, FR6 \\
\hline
\end{tabular}
\caption{Non-functional Requirement LF1}
\end{table}

\begin{table}[h!]
\centering
\rowcolors{2}{white}{white}
\begin{tabular}{|p{3.5cm}|p{11.5cm}|}
\hline
\rowcolor{gray!30}
\textbf{LF2} & The system shall use color schemes and fonts that minimize eye strain during prolonged use. \\
\hline
\textbf{Rationale} & Medical professionals often work long hours; a comfortable visual interface helps prevent fatigue. \\
\hline
\textbf{Fit Criterion} & The interface adheres to ergonomic standards (e.g., ISO 9241-3) for visual displays. \\
\hline
\textbf{Traceability} & FR1, FR5 \\
\hline
\end{tabular}
\caption{Non-functional Requirement LF2}
\end{table}
\newpage
\subsubsection{Usability and Humanity Requirements}

\begin{table}[h!]
\centering
\rowcolors{2}{white}{white}
\begin{tabular}{|p{3.5cm}|p{11.5cm}|}
\hline
\rowcolor{gray!30}
\textbf{UH1} & At least 90\% of healthcare professionals shall be able to perform common tasks (e.g., uploading images, viewing analysis results). \\
\hline
\textbf{Rationale} & Busy professionals need to use the system efficiently without extensive training. \\
\hline
\textbf{Fit Criterion} & Usability testing shows that users can complete key tasks unassisted after initial training. \\
\hline
\textbf{Traceability} & FR1, FR2, FR5, FR6 \\
\hline
\end{tabular}
\caption{Non-functional Requirement UH1}
\end{table}

\begin{table}[h!]
\centering
\rowcolors{2}{white}{white}
\begin{tabular}{|p{3.5cm}|p{11.5cm}|}
\hline
\rowcolor{gray!30}
\textbf{UH2} & The system shall provide context-sensitive help and tooltips for all interface elements to assist users in understanding functionalities. \\
\hline
\textbf{Rationale} & Immediate assistance reduces errors and enhances the user experience. \\
\hline
\textbf{Fit Criterion} & All interactive elements display helpful tooltips when hovered over, and a help section is accessible from all screens. \\
\hline
\textbf{Traceability} & FR2, FR6, FR11 \\
\hline
\end{tabular}
\caption{Non-functional Requirement UH2}
\end{table}
\newpage
\subsubsection{Performance Requirements}

\begin{table}[h!]
\centering
\rowcolors{2}{white}{white}
\begin{tabular}{|p{3.5cm}|p{11.5cm}|}
\hline
\rowcolor{gray!30}
\textbf{PR1} & The system shall process and analyze a standard chest X-ray image within 30 seconds. \\
\hline
\textbf{Rationale} & Quick analysis is critical for timely diagnosis and treatment decisions in clinical settings. \\
\hline
\textbf{Fit Criterion} & Performance tests demonstrate that 95\% of standard images are processed within the 30-second threshold. \\
\hline
\textbf{Traceability} & FR3, FR4, FR5, FR8 \\
\hline
\end{tabular}
\caption{Non-functional Requirement PR1}
\end{table}

\begin{table}[h!]
\centering
\rowcolors{2}{white}{white}
\begin{tabular}{|p{3.5cm}|p{11.5cm}|}
\hline
\rowcolor{gray!30}
\textbf{PR2} & The system shall maintain consistent availability, ensuring minimal disruptions to user access and experience. \\
\hline
\textbf{Rationale} & Continuous access is essential in healthcare environments to prevent delays in patient care. \\
\hline
\textbf{Fit Criterion} & System logs indicate an uptime of 99.9\% over a 6-month monitoring period. \\
\hline
\textbf{Traceability} & FR5, FR7 \\
\hline
\end{tabular}
\caption{Non-functional Requirement PR2}
\end{table}

\begin{table}[h!]
\centering
\rowcolors{2}{white}{white}
\begin{tabular}{|p{3.5cm}|p{11.5cm}|}
\hline
\rowcolor{gray!30}
\textbf{PR3} & The system shall support concurrent processing of up to 20 images without significant performance degradation. \\
\hline
\textbf{Rationale} & Radiologists may need to process multiple images simultaneously, especially in busy clinics or hospitals. \\
\hline
\textbf{Fit Criterion} & System performance tests confirm that processing time per image does not exceed 45 seconds when 20 images are processed concurrently. \\
\hline
\textbf{Traceability} & FR5, FR7, FR15. \\
\hline
\end{tabular}
\caption{Non-functional Requirement PR3}
\end{table}
\newpage
\subsubsection{Operational and Environmental Requirements}

\begin{table}[h!]
\centering
\rowcolors{2}{white}{white}
\begin{tabular}{|p{3.5cm}|p{11.5cm}|}
\hline
\rowcolor{gray!30}
\textbf{OE1} & The system shall integrate with existing hospital Picture Archiving and Communication Systems (PACS) using standard protocols like DICOM. \\
\hline
\textbf{Rationale} & Seamless integration with PACS improves workflow efficiency and reduces manual data handling. \\
\hline
\textbf{Fit Criterion} & The system successfully exchanges data with at least three major PACS solutions in a test environment. \\
\hline
\textbf{Traceability} & FR7, FR10, FR14 \\
\hline
\end{tabular}
\caption{Non-functional Requirement OE1}
\end{table}

\begin{table}[h!]
\centering
\rowcolors{2}{white}{white}
\begin{tabular}{|p{3.5cm}|p{11.5cm}|}
\hline
\rowcolor{gray!30}
\textbf{OE2} & The system shall operate effectively in network environments with latency up to 200ms and packet loss up to 1\%. \\
\hline
\textbf{Rationale} & Hospital networks may have variable conditions; the system must remain functional under less-than-ideal circumstances. \\
\hline
\textbf{Fit Criterion} & System testing under simulated network conditions confirms acceptable performance and responsiveness. \\
\hline
\textbf{Traceability} & FR4, FR6 \\
\hline
\end{tabular}
\caption{Non-functional Requirement OE2}
\end{table}
\subsubsection{Security and Privacy Requirements}

\begin{table}[h!]
\centering
\rowcolors{2}{white}{white}
\begin{tabular}{|p{3.5cm}|p{11.5cm}|}
\hline
\rowcolor{gray!30}
\textbf{SR1} & The system shall encrypt all patient data, including images and reports, both in transit and at rest, using AES-256 encryption. \\
\hline
\textbf{Rationale} & Protecting sensitive patient information is critical to comply with legal requirements and maintain trust. \\
\hline
\textbf{Fit Criterion} & Security audits confirm that all data storage and transmissions use AES-256 encryption. \\
\hline
\textbf{Traceability} & FR7, FR12,FR3 \\
\hline
\end{tabular}
\caption{Non-functional Requirement SR1}
\end{table}

\begin{table}[h!]
\centering
\rowcolors{2}{white}{white}
\begin{tabular}{|p{3.5cm}|p{11.5cm}|}
\hline
\rowcolor{gray!30}
\textbf{SR2} & The system shall implement role-based access control, ensuring users can only access functionalities and data appropriate to their roles (e.g., radiologist, radiographer, administrator). \\
\hline
\textbf{Rationale} & Restricting access based on roles prevents unauthorized data access and potential data breaches. \\
\hline
\textbf{Fit Criterion} & Access control tests verify that users cannot access data or functions outside their permissions. \\
\hline
\textbf{Traceability} & FR13, FR12 \\
\hline
\end{tabular}
\caption{Non-functional Requirement SR2}
\end{table}
\newpage
\subsubsection{Maintainability and Support Requirements}

\begin{table}[h!]
\centering
\rowcolors{2}{white}{white}
\begin{tabular}{|p{3.5cm}|p{11.5cm}|}
\hline
\rowcolor{gray!30}
\textbf{MS1} & The system shall be developed using modular architecture with well-documented code to facilitate future maintenance and upgrades. \\
\hline
\textbf{Rationale} & A modular design simplifies debugging, testing, and integration of new features, reducing maintenance costs. \\
\hline
\textbf{Fit Criterion} & Code reviews confirm adherence to coding standards and modular design principles; documentation covers all modules. \\
\hline
\textbf{Traceability} & FR2, FR10, FR16 \\
\hline
\end{tabular}
\caption{Non-functional Requirement MS1}
\end{table}

\begin{table}[h!]
\centering
\rowcolors{2}{white}{white}
\begin{tabular}{|p{3.5cm}|p{11.5cm}|}
\hline
\rowcolor{gray!30}
\textbf{MS2} & The system shall include automated testing suites covering at least 80\% of the codebase to ensure reliability during updates. \\
\hline
\textbf{Rationale} & High test coverage helps detect issues early and maintains system stability over time. \\
\hline
\textbf{Fit Criterion} & Test coverage reports indicate that automated tests cover 80\% or more of the code. \\
\hline
\textbf{Traceability} & FR7, FR14, FR16\\
\hline
\end{tabular}
\caption{Non-functional Requirement MS2}
\end{table}
\newpage
\subsubsection{Cultural Requirements}

\begin{table}[h!]
\centering
\rowcolors{2}{white}{white}
\begin{tabular}{|p{3.5cm}|p{11.5cm}|}
\hline
\rowcolor{gray!30}
\textbf{CR1} & The system shall support both English and French languages for the user interface and reports. \\
\hline
\textbf{Rationale} & Supporting Canada's official languages ensures accessibility for all users and meets legal requirements in certain provinces. \\
\hline
\textbf{Fit Criterion} & Users can select English or French as the system language, with all interface elements and reports displayed accordingly. \\
\hline
\textbf{Traceability} & FR6, FR12, FR13 \\
\hline
\end{tabular}
\caption{Non-functional Requirement CR1}
\end{table}
\subsubsection{Legal Requirements}

\begin{table}[h!]
\centering
\rowcolors{2}{white}{white}
\begin{tabular}{|p{3.5cm}|p{11.5cm}|}
\hline
\rowcolor{gray!30}
\textbf{LR1} & The system shall comply with all applicable healthcare data protection laws, including HIPAA in the United States and PIPEDA in Canada. \\
\hline
\textbf{Rationale} & Legal compliance is mandatory to protect patient rights and avoid legal repercussions. \\
\hline
\textbf{Fit Criterion} & Compliance audits verify adherence to relevant data protection laws and regulations. \\
\hline
\textbf{Traceability} & FR12, FR13, FR7 \\
\hline
\end{tabular}
\caption{Non-functional Requirement LR1}
\end{table}

\begin{table}[h!]
\centering
\rowcolors{2}{white}{white}
\begin{tabular}{|p{3.5cm}|p{11.5cm}|}
\hline
\rowcolor{gray!30}
\textbf{LR2} & The system shall adhere to the ISO 13485 standard for medical device software development. \\
\hline
\textbf{Rationale} & Following recognized standards ensures quality and safety in medical software. \\
\hline
\textbf{Fit Criterion} & Certification or compliance reports demonstrate adherence to ISO 13485. \\
\hline
\textbf{Traceability} & FR7, FR10, FR15 \\
\hline
\end{tabular}
\caption{Non-functional Requirement LR2}
\end{table}
\newpage
\subsubsection{Health and Safety Requirements}

\begin{table}[h!]
\centering
\rowcolors{2}{white}{white}
\begin{tabular}{|p{3.5cm}|p{11.5cm}|}
\hline
\rowcolor{gray!30}
\textbf{HS1} & The system shall ensure that all AI-generated diagnoses are reviewed and confirmed by a qualified radiologist before being used in patient care decisions. \\
\hline
\textbf{Rationale} & To prevent misdiagnoses and ensure patient safety by involving human oversight. \\
\hline
\textbf{Fit Criterion} & System workflow requires radiologist validation before finalizing reports; logs confirm this process. \\
\hline
\textbf{Traceability} & FR3, FR6, FR4 \\
\hline
\end{tabular}
\caption{Non-functional Requirement HS1}
\end{table}

\begin{table}[h!]
\centering
\rowcolors{2}{white}{white}
\begin{tabular}{|p{3.5cm}|p{11.5cm}|}
\hline
\rowcolor{gray!30}
\textbf{HS2} & The system shall provide clear disclaimers indicating that AI analysis is a diagnostic aid and not a replacement for professional medical judgment. \\
\hline
\textbf{Rationale} & Users must understand the limitations of AI to prevent overreliance and potential errors. \\
\hline
\textbf{Fit Criterion} & Disclaimers are prominently displayed on analysis results and require user acknowledgment upon first use. \\
\hline
\textbf{Traceability} &FR12, FR6, FR11\\
\hline
\end{tabular}
\caption{Non-functional Requirement HS2}
\end{table}

\newpage
\section{Phase in Plan}    
\subsection{High Priority Functional Requirements}
\begin{table}[H]
  \label{TblHighPriorityFRs}
  \begin{tabular}{p{0.15\textwidth}|p{0.85\textwidth}}
  \toprule
  \textbf{Req ID} & \textbf{Completion Date and Rationale} \\
  \midrule
  FR1 & \textbf{Completion Date:} October 17, 2023\\
      & \textbf{Rationale:} FR1 is essential because accepting chest X-ray images is the foundation of our system. We need this functionality first to enable all other features to build upon it. \\
  \midrule
  FR3 & \textbf{Completion Date:} October 20, 2023\\
      & \textbf{Rationale:} FR3 is critical as it provides the core capability of analyzing X-rays for disease detection. Implementing this early allows us to validate the system's primary purpose and proceed with dependent features. \\
  \midrule
  FR4 & \textbf{Completion Date:} October 25, 2023\\
      & \textbf{Rationale:} FR4 is important for monitoring patient progress. By completing it alongside FR3, we ensure clinicians can assess changes over time from the initial release. \\
  \midrule
  FR6 & \textbf{Completion Date:} November 13, 2023\\
      & \textbf{Rationale:} FR6 allows us to present analysis results in a usable format for clinicians. Having this ready by November 13 is vital for user acceptance and practical use of the system. \\
  \midrule
  FR11 & \textbf{Completion Date:} November 20, 2023\\
       & \textbf{Rationale:} FR11 improves user trust by displaying confidence levels. An intuitive UI is essential for user adoption. \\
  \midrule
  FR13 & \textbf{Completion Date:} November 25, 2023\\
       & \textbf{Rationale:} FR13 is necessary for security and compliance, ensuring proper access control from the start. Implementing it early helps us protect sensitive data effectively. \\
  \midrule
  FR14 & \textbf{Completion Date:} November 15, 2023\\
       & \textbf{Rationale:} FR14 safeguards the integrity of original X-ray images.\\
  \bottomrule
  \end{tabular}
  \end{table}

  \subsection{Medium Priority Functional Requirements}
\begin{table}[H]
  \label{TblMediumPriorityFRs}
  \begin{tabular}{p{0.15\textwidth}|p{0.85\textwidth}}
  \toprule
  \textbf{Req ID} & \textbf{Completion Date and Rationale} \\
  \midrule
  FR2 & \textbf{Completion Date:} December 3, 2024\\
      & \textbf{Rationale:} FR2 adds the ability to input additional patient symptoms, enhancing analysis accuracy. We scheduled it after the core functionalities to focus on essential features first. \\
  \midrule
  FR5 & \textbf{Completion Date:} December 5, 2024\\
      & \textbf{Rationale:} FR5 provides visual aids, improving usability. Implementing it after the main analysis features allows us to refine the user experience based on initial feedback. \\
  \midrule
  FR7 & \textbf{Completion Date:} Januaray 20, 2024\\
      & \textbf{Rationale:} FR7 involves secure data storage for future reference. We plan to develop this after the initial deployment to ensure data handling aligns with user needs and compliance requirements. \\
  \midrule
  FR16 & \textbf{Completion Date:} February 2, 2024\\
       & \textbf{Rationale:} FR16 enables regular updates to the AI model, allowing continuous improvement. Scheduling it for February gives us time to stabilize the initial model and plan for updates without disrupting service. \\
  \bottomrule
  \end{tabular}
  \caption{High and Medium Priority Functional Requirements}
  \end{table}
  \subsection{Low Priority Functional Requirements}
\begin{table}[H]
  \label{TblLowPriorityFRs}
  \begin{tabular}{p{0.15\textwidth}|p{0.85\textwidth}}
  \toprule
  \textbf{Req ID} & \textbf{Completion Date and Rationale} \\
  \midrule
  FR8 & \textbf{Completion Date:} February 10, 2024\\
      & \textbf{Rationale:} FR8 offers alerts for significant condition changes. It's useful but not essential for initial deployment, so we scheduled it after higher-priority tasks. \\
  \midrule
  FR10 & \textbf{Completion Date:} February 11, 2024\\
       & \textbf{Rationale:} FR10 involves integrating with EHR systems, which adds complexity. We plan to address this later to focus on core functionalities first. \\
  \midrule
  FR12 & \textbf{Completion Date:} March 2, 2024\\
       & \textbf{Rationale:} FR12 allows patient self-upload of images. Given legal considerations and our initial focus on clinicians, we scheduled it for later development. \\
  \midrule
  FR15 & \textbf{Completion Date:} March 10, 2024\\
       & \textbf{Rationale:} FR15 expands support to other imaging modalities. We aim to perfect chest X-ray analysis before adding CT and MRI support, so we scheduled this for the last phase. \\
  \bottomrule
  \end{tabular}
  \caption{Low Priority Functional Requirements}
  \end{table}
  
\subsection{High Priority Non-Functional Requirements}
\begin{table}[H]
  \label{TblHighPriorityNFRs}
  \begin{tabular}{p{0.15\textwidth}|p{0.85\textwidth}}
  \toprule
  \textbf{Req ID} & \textbf{Completion Date and Rationale} \\
  \midrule
  PR1 & \textbf{Completion Date:} October 17, 2023\\
      & \textbf{Rationale:} PR1 ensures the system processes images quickly, which is crucial for usability in clinical settings. \\
  \midrule
  HS1 & \textbf{Completion Date:} October 22, 2023\\
      & \textbf{Rationale:} HS1 mandates that AI diagnoses are reviewed by a radiologist, ensuring patient safety. This oversight is necessary from day one. \\
  \midrule
  LF1 & \textbf{Completion Date:} December 15, 2023\\
      & \textbf{Rationale:} LF1 requires an interface consistent with standard medical software, aiding user adoption. We want users to feel comfortable using the system immediately. \\
  \midrule
  PR2 & \textbf{Completion Date:} October 25, 2023\\
      & \textbf{Rationale:} PR2 aims for high system uptime. Reliability is critical, so we must ensure the system is stable from the initial launch. \\
  \midrule
  SR2 & \textbf{Completion Date:} December 4, 2023\\
      & \textbf{Rationale:} SR2 implements role-based access control, necessary for security and compliance. We need to control data access from the beginning. \\
  \midrule
  MS1 & \textbf{Completion Date:} December 8, 2023\\
      & \textbf{Rationale:} MS1 involves modular architecture and documentation, facilitating future maintenance. \\
  \midrule
  LR1 & \textbf{Completion Date:} Januaray 7, 2023\\
      & \textbf{Rationale:} LR1 ensures compliance with data protection laws, which is legally required. We must adhere to regulations from the outset to avoid legal issues. \\
  \midrule
  LR2 & \textbf{Completion Date:} January 20, 2023\\
      & \textbf{Rationale:} LR2 requires adherence to ISO 13485, ensuring quality and safety. Meeting industry standards builds trust and facilitates future certifications. \\
  \midrule
  HS2 & \textbf{Completion Date:} January 25, 2023\\
      & \textbf{Rationale:} HS2 mandates clear disclaimers about AI limitations. Providing this information from the start is important for user awareness and legal protection. \\
  \bottomrule
  \end{tabular}
  \end{table}
  \subsection{Medium and Low Priority Non-Functional Requirements}
\begin{table}[H]
  \label{TblMediumLowPriorityNFRs}
  \begin{tabular}{p{0.15\textwidth}|p{0.85\textwidth}}
  \toprule
  \textbf{Req ID} & \textbf{Completion Date and Rationale} \\
  \midrule
  LF2 & \textbf{Completion Date:} February 18, 2024\\
      & \textbf{Rationale:} LF2 improves the user interface to minimize eye strain. While important for user comfort, it can be refined after the main features are operational. \\
  \midrule
  UH2 & \textbf{Completion Date:} February 15, 2024\\
      & \textbf{Rationale:} UH2 adds context-sensitive help and tooltips. Enhancing usability is beneficial, but we plan to implement it after gathering initial user feedback. \\
  \midrule
  PR3 & \textbf{Completion Date:} February 10, 2024\\
      & \textbf{Rationale:} PR3 enhances performance by supporting concurrent image processing. Scheduling this for later allows us to optimize the system after core functionalities are stable. \\
  \midrule
  MS2 & \textbf{Completion Date:} February 2, 2024\\
      & \textbf{Rationale:} MS2 involves automated testing for reliability. Implementing it after initial development helps us maintain code quality as the project grows. \\
  \midrule
  OE1 & \textbf{Completion Date:} March 18, 2024\\
       & \textbf{Rationale:} OE1 integrates with hospital PACS systems. Given the complexity, we plan to tackle this once the system's core features are solidified. \\
  \midrule
  OE2 & \textbf{Completion Date:} March 14, 2024\\
       & \textbf{Rationale:} OE2 ensures operation under variable network conditions. This enhancement is scheduled for later, focusing first on standard network environments. \\
  \midrule
  CR1 & \textbf{Completion Date:} March 4, 2024\\
       & \textbf{Rationale:} CR1 adds multilingual support. We aim to include additional languages after ensuring the system functions well in the primary language. \\
  \bottomrule
  \end{tabular}
  \caption{Medium and Low Priority Non-Functional Requirements}
  \end{table}
  
  
\clearpage
\section{Likelihood of Changes for Requirements}

\subsection{Very Unlikely to Change}

These requirements are core to our system's functionality and essential for meeting the project's objectives. We do not anticipate any changes to these requirements.

\begin{enumerate}
    \item \textbf{FR1}: Accept chest X-ray images from authorized users.
    \begin{itemize}[label=-]
        \item \textbf{Rationale}: This is the fundamental functionality upon which all other features depend.
        \item \textbf{Possible Changes}: N/A
    \end{itemize}

    \item \textbf{FR3}: Analyze chest X-rays to detect diseases with $\geq85\%$ accuracy.
    \begin{itemize}[label=-]
        \item \textbf{Rationale}: Central to the system's purpose; critical for providing value to users.
        \item \textbf{Possible Changes}: N/A
    \end{itemize}

    \item \textbf{FR6}: Produce structured, human-readable reports summarizing findings.
    \begin{itemize}[label=-]
        \item \textbf{Rationale}: Essential for clinicians to interpret results effectively.
        \item \textbf{Possible Changes}: N/A
    \end{itemize}

    \item \textbf{FR13}: Support multiple user roles with appropriate access levels.
    \begin{itemize}[label=-]
        \item \textbf{Rationale}: Crucial for security and compliance with regulations.
        \item \textbf{Possible Changes}: N/A
    \end{itemize}

    \item \textbf{SR1}: Encrypt all patient data using AES-256 encryption.
    \begin{itemize}[label=-]
        \item \textbf{Rationale}: Mandatory for data security and compliance.
        \item \textbf{Possible Changes}: N/A
    \end{itemize}

    \item \textbf{SR2}: Implement role-based access control.
    \begin{itemize}[label=-]
        \item \textbf{Rationale}: Critical for security and regulatory compliance.
        \item \textbf{Possible Changes}: N/A
    \end{itemize}

    \item \textbf{LR1}: Comply with healthcare data protection laws (HIPAA, PIPEDA).
    \begin{itemize}[label=-]
        \item \textbf{Rationale}: Legal compliance is mandatory.
        \item \textbf{Possible Changes}: N/A
    \end{itemize}
\end{enumerate}

\subsection{Unlikely to Change}

These requirements are important but may undergo minor adjustments based on user feedback or technical considerations.

\begin{enumerate}[resume]
    \item \textbf{FR2}: Enable users to input additional patient symptoms.
    \begin{itemize}[label=-]
        \item \textbf{Rationale}: Enhances analysis accuracy but not critical for initial operation.
        \item \textbf{Possible Changes}: Modify symptom input methods or adjust symptom lists.
    \end{itemize}

    \item \textbf{FR4}: Indicate if a patient's condition has changed between scans.
    \begin{itemize}[label=-]
        \item \textbf{Rationale}: Important for monitoring progress; criteria may need refinement.
        \item \textbf{Possible Changes}: Adjust methods for assessing condition changes.
    \end{itemize}

    \item \textbf{FR5}: Generate visual aids by highlighting affected areas.
    \begin{itemize}[label=-]
        \item \textbf{Rationale}: Improves usability; visualization techniques may evolve.
        \item \textbf{Possible Changes}: Update highlighting methods based on user feedback.
    \end{itemize}

    \item \textbf{FR7}: Store patient data securely for future reference.
    \begin{itemize}[label=-]
        \item \textbf{Rationale}: Important for data continuity; storage solutions might change.
        \item \textbf{Possible Changes}: Enhance security measures or adopt new technologies.
    \end{itemize}

    \item \textbf{FR9}: Allow healthcare professionals to adjust treatment plans.
    \begin{itemize}[label=-]
        \item \textbf{Rationale}: Adds value for clinical decision-making; integration methods may adjust.
        \item \textbf{Possible Changes}: Refine how treatment adjustments are linked to analysis.
    \end{itemize}

    \item \textbf{FR11}: Display confidence levels through an intuitive UI.
    \begin{itemize}[label=-]
        \item \textbf{Rationale}: Helps assess result reliability; presentation might be refined.
        \item \textbf{Possible Changes}: Modify the display format of confidence levels.
    \end{itemize}

    \item \textbf{FR14}: Create a new copy of the X-ray before AI analysis.
    \begin{itemize}[label=-]
        \item \textbf{Rationale}: Protects data integrity; implementation methods might improve.
        \item \textbf{Possible Changes}: Optimize data handling or copying procedures.
    \end{itemize}

    \item \textbf{FR16}: Support regular updates to the AI model.
    \begin{itemize}[label=-]
        \item \textbf{Rationale}: Important for continuous improvement; update processes may adjust.
        \item \textbf{Possible Changes}: Alter the frequency or method of updates.
    \end{itemize}

    \item \textbf{PR1}: Process and analyze images within 30 seconds.
    \begin{itemize}[label=-]
        \item \textbf{Rationale}: Critical for usability; performance targets might adjust slightly.
        \item \textbf{Possible Changes}: Optimize algorithms or extend processing time marginally.
    \end{itemize}

    \item \textbf{HS1}: Ensure AI diagnoses are reviewed by a radiologist.
    \begin{itemize}[label=-]
        \item \textbf{Rationale}: Important for patient safety; review processes may streamline.
        \item \textbf{Possible Changes}: Adjust workflow integration based on feedback.
    \end{itemize}

    \item \textbf{LF1}: Have a user interface consistent with standard medical software.
    \begin{itemize}[label=-]
        \item \textbf{Rationale}: Aids user adoption; UI elements might be updated.
        \item \textbf{Possible Changes}: Refine interface design for improved usability.
    \end{itemize}

    \item \textbf{UH1}: Users perform tasks without assistance after 30-minute training.
    \begin{itemize}[label=-]
        \item \textbf{Rationale}: Key for efficiency; training materials may be adjusted.
        \item \textbf{Possible Changes}: Modify training duration or content.
    \end{itemize}

    \item \textbf{PR2}: Achieve system uptime of at least 99.9\%.
    \begin{itemize}[label=-]
        \item \textbf{Rationale}: Essential for reliability; targets may adjust slightly.
        \item \textbf{Possible Changes}: Revise uptime goals if necessary.
    \end{itemize}

    \item \textbf{MS1}: Develop using modular architecture with well-documented code.
    \begin{itemize}[label=-]
        \item \textbf{Rationale}: Important for maintenance; practices may evolve.
        \item \textbf{Possible Changes}: Update documentation standards or modularization approaches.
    \end{itemize}

    \item \textbf{MS2}: Include automated testing covering at least 80\% of the codebase.
    \begin{itemize}[label=-]
        \item \textbf{Rationale}: Crucial for reliability; coverage targets may adjust.
        \item \textbf{Possible Changes}: Modify test coverage goals based on feasibility.
    \end{itemize}

    \item \textbf{LR2}: Adhere to ISO 13485 standard.
    \begin{itemize}[label=-]
        \item \textbf{Rationale}: Important for quality; compliance may adjust based on feasibility.
        \item \textbf{Possible Changes}: Plan for partial compliance initially, aiming for full adherence over time.
    \end{itemize}

    \item \textbf{HS2}: Provide disclaimers that AI is a diagnostic aid.
    \begin{itemize}[label=-]
        \item \textbf{Rationale}: Essential for legal and ethical reasons.
        \item \textbf{Possible Changes}: Refine wording of disclaimers as needed.
    \end{itemize}
\end{enumerate}

\subsection{Likely to Change}

These requirements may be adjusted due to resource constraints, technical challenges, or evolving user needs.

\begin{enumerate}[resume]
    \item \textbf{FR8}: Provide alerts for significant changes between scans.
    \begin{itemize}[label=-]
        \item \textbf{Rationale}: Useful but not essential; may be deferred due to time constraints.
        \item \textbf{Possible Changes}: Simplify the alert system or postpone implementation.
    \end{itemize}

    \item \textbf{FR10}: Integrate with EHR systems to import/export data.
    \begin{itemize}[label=-]
        \item \textbf{Rationale}: Adds complexity; might be challenging within our timeline.
        \item \textbf{Possible Changes}: Limit integration scope or schedule for future development.
    \end{itemize}

    \item \textbf{FR12}: Allow patients to upload images for self-diagnosis with disclaimers.
    \begin{itemize}[label=-]
        \item \textbf{Rationale}: Raises legal and ethical concerns; may need alteration.
        \item \textbf{Possible Changes}: Restrict feature access or require additional consent.
    \end{itemize}

    \item \textbf{FR15}: Support additional imaging modalities like CT scans and MRIs.
    \begin{itemize}[label=-]
        \item \textbf{Rationale}: May exceed current project scope; could be deferred.
        \item \textbf{Possible Changes}: Focus on chest X-rays initially; schedule other modalities for later phases.
    \end{itemize}

    \item \textbf{LF2}: Use color schemes and fonts that minimize eye strain.
    \begin{itemize}[label=-]
        \item \textbf{Rationale}: Aesthetic elements are subjective; may change with user feedback.
        \item \textbf{Possible Changes}: Redesign UI for improved comfort.
    \end{itemize}

    \item \textbf{UH2}: Provide context-sensitive help and tooltips.
    \begin{itemize}[label=-]
        \item \textbf{Rationale}: Resource constraints might affect implementation scope.
        \item \textbf{Possible Changes}: Adjust help features or prioritize key areas.
    \end{itemize}

    \item \textbf{PR3}: Support processing of up to 20 images concurrently.
    \begin{itemize}[label=-]
        \item \textbf{Rationale}: Performance under load may be challenging.
        \item \textbf{Possible Changes}: Reduce number of concurrent images or optimize performance.
    \end{itemize}

    \item \textbf{OE1}: Integrate with hospital PACS using DICOM.
    \begin{itemize}[label=-]
        \item \textbf{Rationale}: Integration complexity may limit scope.
        \item \textbf{Possible Changes}: Support specific PACS systems or delay integration.
    \end{itemize}

    \item \textbf{OE2}: Operate effectively in networks with latency up to 200ms.
    \begin{itemize}[label=-]
        \item \textbf{Rationale}: Network optimization may be challenging within our timeline.
        \item \textbf{Possible Changes}: Adjust network condition requirements or focus on standard environments first.
    \end{itemize}

    \item \textbf{CR1}: Support both English and French languages.
    \begin{itemize}[label=-]
        \item \textbf{Rationale}: Multilingual support adds complexity; may be postponed.
        \item \textbf{Possible Changes}: Limit initial support to English; add languages later.
    \end{itemize}
\end{enumerate}


\newpage


\bibliographystyle {plainnat}
\bibliography {../../refs/References}
\afterpage{
\begin{landscape}
\section*{Appendix --- Traceability Matrix}
\begin{table}[h!]
\centering
\scriptsize
\setlength{\tabcolsep}{3pt} % Adjust the column separation to make it tighter
\renewcommand{\arraystretch}{1.15} % Adjust row height for a tighter fit
\begin{tabular}{|p{1cm}|p{0.9cm}|p{0.9cm}|p{0.9cm}|p{0.9cm}|p{0.9cm}|p{0.9cm}|p{0.9cm}|p{0.9cm}|p{0.9cm}|p{0.9cm}|p{0.9cm}|p{0.9cm}|p{0.9cm}|p{0.9cm}|p{0.9cm}|p{0.9cm}|}
\hline
    & \textbf{FR1} & \textbf{FR2} & \textbf{FR3} & \textbf{FR4} & \textbf{FR5} & \textbf{FR6} & \textbf{FR7} & \textbf{FR8} & \textbf{FR9} & \textbf{FR10} & \textbf{FR11} & \textbf{FR12} & \textbf{FR13} & \textbf{FR14} & \textbf{FR15} & \textbf{FR16} \\ 
\hline
\textbf{FR1} & X & X & & & & & X & & & & & & & & & \\ 
\hline
\textbf{FR2} & & X & & & & & & & & & & & & & & X \\ 
\hline
\textbf{FR3} & X & & X & X & X & X & & & & & & & & & & X \\ 
\hline
\textbf{FR4} & & & X & X & X & & & X & & & & & & & & \\ 
\hline
\textbf{FR5} & & & & X & X & X & & & & & & & & & X & \\ 
\hline
\textbf{FR6} & X & & X & X & X & X & & & X & & & & & & & \\ 
\hline
\textbf{FR7} & X & & & & & X & X & & & X & & & X & & & \\ 
\hline
\textbf{FR8} & & & & X & & X & & X & & & & & & & & \\ 
\hline
\textbf{FR9} & & & & & & X & & & X & & & & & & & \\ 
\hline
\textbf{FR10} & & & & & & & X & & & X & & & X & & & X \\ 
\hline
\textbf{FR11} & & & & & & & & & & & X & & & & & \\ 
\hline
\textbf{FR12} & X & & & & & & & & & & X & X & X & & & \\ 
\hline
\textbf{FR13} & & & & & & & X & & & X & & X & X & & & \\ 
\hline
\textbf{FR14} & X & & & & & & & & & & & & & X & & \\ 
\hline
\textbf{FR15} & X & & & & & & & & & & & & & & X & \\ 
\hline
\textbf{FR16} & & & X & & & & & & & X & & & & & X & X \\ 
\hline
\textbf{LF1} & X & & X & X & X & & & & & & & & & & & \\ 
\hline
\textbf{LF2} & X & & & & & & & & & & & & & & & \\ 
\hline
\textbf{UH1} & X & X & & & & & & & & & & & & & & \\ 
\hline
\textbf{UH2} & & X & & & X & X & & & & & X & & & & & \\ 
\hline
\textbf{PR1} & & & X & X & X & & & & & & & & & & & \\ 
\hline
\textbf{PR2} & & & & & X & & X & & & & & & & & & \\ 
\hline
\textbf{PR3} & & & & & X & & X & & & & & & & & & \\ 
\hline
\textbf{OE1} & & & & & & & X & & & & & & & X & & \\ 
\hline
\textbf{OE2} & & & & X & & X & & & & & & & & & & \\ 
\hline
\textbf{SR1} & & & & & & & X & & & & & X & & & & \\ 
\hline
\textbf{SR2} & & & & & & & & & & & & X & X & & & \\ 
\hline
\textbf{MS1} & & & & & & & & & & X & & & & & & X \\ 
\hline
\textbf{MS2} & & & & & & & X & & & & & & & X & & X \\ 
\hline
\textbf{CR1} & & & & & & & & & & & & & & X & & \\ 
\hline
\textbf{LR1} & & & & & & & & & & & & X & X & & & \\ 
\hline
\textbf{LR2} & & & & & & & X & & & X & & & & X & X & \\ 
\hline
\textbf{HS1} & & & X & X & & & & & & & & & & & & \\ 
\hline
\textbf{HS2} & & & & & & & & & & & & X & & & & \\ 
\hline
\end{tabular}
\caption{Traceability Matrix Showing the Dependencies Between Functional and Non-Functional Requirements}
\label{Table:Traceability}
\end{table}
\end{landscape}
}
\newpage{}
\section*{Appendix --- Reflection}

\begin{enumerate}
  \item What went well while writing this deliverable? \\\\
We felt the SRS document provided us strong growth and structures as the content was comprehensive. We were able to gain a better understanding of the necessary elements like the scope, context diagrams, and nonfunctional requirements such as performance and safety. The clear delineation between normal operations and undesired event handling gave us confidence in its thoroughness. Overall, we felt that the SRS document provided us with a solid foundation to work from, especially in terms of its clarity, organization, and attention to detail. It gave us a clear picture of the system's requirements and made it easier to move forward with design and implementation.


  \item What pain points did you experience during this deliverable, and how did
  you resolve them? \\

When writing the Software Requirements Specification document caused us few challenges that hindered our process. One common issue we faced was the ambiguity in the requirements this can create confusion for our team. We also felt that some things had incomplete requirements, as some aspects were overlooked or not thoroughly explored. Additionally, failing to gather input from all stakeholders, such as clients, end-users, and developers, can result in missing critical features or misunderstanding the true needs of the project. We resolved these pain points by talking with the teaching assistants of this course, whom of which helped us immensely.  


  \item How many of your requirements were inspired by speaking to your
  client(s) or their proxies (e.g. your peers, stakeholders, potential users)? \\
  
Many of our requirements came from us asking healthcare professionals as well as our supervisor. There was a big emphasises on the need for accuracy, reliability, and a user-friendly interface. These discussions helped shape key requirements, such as incorporating a robust AI model for detecting lung diseases, ensuring transparency in how the AI reaches its conclusions, and making the system accessible to users with varying levels of technical expertise.
      
  \item Which of the courses you have taken, or are currently taking, will help
  your team to be successful with your capstone project. \\

  By taking courses such as Machine Learning, Human-Computer Interfaces (HCI), and previous taken software development will help us substantially. The Machine Learning course has equipped us to gain a better understanding and via taking 4ML3 we can develop an accurate AI model capable of analyzing chest X-rays and identifying potential lung diseases, including knowledge of various algorithms. Meanwhile, as for HCI, the course is teaching us the importance of creating intuitive, user-friendly interfaces that make complex AI results accessible to medical professionals and patients alike. By applying user-centered design principles, we aim to ensure seamless interaction with our system and enhance user trust through transparent communication of AI decisions. All in all, our previous software development courses have further solidified our programming skills and understanding of software architecture, enabling us to effectively integrate these elements into a cohesive application.
  
  \item What knowledge and skills will the team collectively need to acquire to
  successfully complete this capstone project?  Examples of possible knowledge
  to acquire include domain specific knowledge from the domain of your
  application, or software engineering knowledge, mechatronics knowledge or
  computer science knowledge.  Skills may be related to technology, or writing,
  or presentation, or team management, etc.  You should look to identify at
  least one item for each team member.
  \item For each of the knowledge areas and skills identified in the previous
  question, what are at least two approaches to acquiring the knowledge or
  mastering the skill?  Of the identified approaches, which will each team
  member pursue, and why did they make this choice? \\\\

    The following Two quesitons have been answered below: \\\\

\textbf{Assigned Team Members: All} \\
All team members will be required to thoroughly understand the medical conditions and diseases related to chest X-rays that the AI model will be detecting. This knowledge will be acquired through reviewing medical literature, studying annotated chest X-ray datasets, and utilizing any available online medical resources or tutorials that explain chest X-ray interpretation and diagnosis. Every team member will be using all the discussed methods to gain expertise in this domain, as it is essential for everyone to understand the features of different lung conditions (such as pneumonia, tuberculosis, or lung cancer). The team will start by studying medical images and datasets, and then further enhance their knowledge by reading research papers and consulting with healthcare professionals to understand how chest X-rays reveal signs of disease.


\textbf{Assigned Team Members: Patrick, Reza, and Kelly} \\
Learning the fundamentals of PyTorch is essential for building AI models to analyze lung X-rays. Our primary focus will be gaining a better understanding of PyTorch's torch tensor data structure. Understanding automatic differentiation through PyTorch's autograd module will allow for efficient backpropagation during model training. Moreover, gaining knowledge on how to handle X-ray images, including resizing, normalization, and augmentation, will be crucial. This will involve using libraries like \texttt{torchvision} to apply image transformations and \texttt{DataLoader} to manage batching and loading data for training.


\textbf{Assigned Team Members: Ayman} \\
Data Visualization Analysis will require developers to understand users' expected data points, graphic representations, and what types of data need to be compared. Group members are required to have sufficient knowledge of Python data science libraries such as \texttt{Matplotlib}, \texttt{Seaborn}, \texttt{NumPy}, and \texttt{pandas} to create informative graphics and present accurate information to users efficiently. Ayman will be using online tutorials and documentation to gain expertise in data visualization and data analysis to find the most effective way to analyze the logs from the Game Engine and AI Agent.

\subsection*{13.4 Web Creation}

\textbf{Assigned Team Members: Nathan} \\
The Angular Frontend Development will require developers to design and implement a user-friendly interface for our web application. The team will focus on creating a responsive and efficient UI that meets users' needs by ensuring optimal performance and accessibility. The frontend should effectively interact with the backend API for data retrieval and submission, enabling smooth integration of user interactions with the underlying server logic.

This requires sufficient knowledge of Angular, TypeScript, HTML, and CSS. They must be proficient in Angular components, directives, services, routing, and form validation to build dynamic, modular, and maintainable applications. The team will utilize Angular Material or Bootstrap for UI components to create visually appealing and responsive designs.


\textbf{Assigned Team Members: All} \\
The main component of the AI model for chest X-ray analysis centers around comprehending the user’s expectations. Healthcare professionals, including radiologists, pulmonologists, and general practitioners, will be the main users of this application since they will need to use the tool for diagnosing lung diseases. Therefore, it is critical to gain an understanding of users and their information needs when designing the AI model and its interface, with the intention of providing relevant information that is precise and adequate.

Furthermore, our application would be helpful for end-users, including junior doctors in medical school, who can use the application as an educational tool or supplemental diagnostic aid. Determining which users will be the target audience and their expectations will inform design choices ranging from how complex the expected output is to how user-friendly the user-interface design is. Everyone is expected to partake in user research, where they will seek feedback from potential end-users and consult healthcare professionals to ensure the tool caters to the specific needs of users from different disciplines. The team will also consider designs that may be more attractive to patients, such as picture-based diagnostic aids or X-ray interpretation rather than a plethora of diagnostic radiology information. This will involve creating a proper compromise between providing detailed diagnostic and analytical information.

  
\end{enumerate}

\section*{Appendix --- Refinements to the Original SRS Template}
The following changes were made to tailor the SRS template to our project requirements:
\begin{itemize}
  \item Impact Analysis (Section 2.5): This section was added to discuss the overall impact of the project on the stakeholders and society.
  \item Behavior Overview (Section 3.2): This section was added to provide a high-level overview of the system's behavior under various conditions, showcasing it through a use case diagram.
  \item Solution Characteristic Specification (Section 4.2): Major changes were made to this section to describe the system's specifications and architecture in more detail.
  \item Formal Mathematical Specification (Section 5.1): This was added as a preliminary to the system's requirements to provide a formal mathematical specification of the system.
  \item Phase in Plan (Section 6): An outline was added to show the timeline of the project, providing a clear understanding of the project's development stages.
\end{itemize}


\end{document}