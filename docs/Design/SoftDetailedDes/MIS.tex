\documentclass[12pt, titlepage]{article}

\usepackage{amsmath, mathtools}

\usepackage[round]{natbib}
\usepackage{amsfonts}
\usepackage{amssymb}
\usepackage{graphicx}
\usepackage{colortbl}
\usepackage{xr}
\usepackage{hyperref}
\usepackage{longtable}
\usepackage{xfrac}
\usepackage{tabularx}
\usepackage{float}
\usepackage{siunitx}
\usepackage{booktabs}
\usepackage{multirow}
\usepackage[section]{placeins}
\usepackage{caption}
\usepackage{fullpage}

\hypersetup{
bookmarks=true,     % show bookmarks bar?
colorlinks=true,       % false: boxed links; true: colored links
linkcolor=red,          % color of internal links (change box color with linkbordercolor)
citecolor=blue,      % color of links to bibliography
filecolor=magenta,  % color of file links
urlcolor=cyan          % color of external links
}

\usepackage{array}

\externaldocument{../../SRS/SRS}

%% Comments

\usepackage{color}

\newif\ifcomments\commentstrue %displays comments
%\newif\ifcomments\commentsfalse %so that comments do not display

\ifcomments
\newcommand{\authornote}[3]{\textcolor{#1}{[#3 ---#2]}}
\newcommand{\todo}[1]{\textcolor{red}{[TODO: #1]}}
\else
\newcommand{\authornote}[3]{}
\newcommand{\todo}[1]{}
\fi

\newcommand{\wss}[1]{\authornote{blue}{SS}{#1}} 
\newcommand{\plt}[1]{\authornote{magenta}{TPLT}{#1}} %For explanation of the template
\newcommand{\an}[1]{\authornote{cyan}{Author}{#1}}

%% Common Parts

\newcommand{\progname}{ProgName} % PUT YOUR PROGRAM NAME HERE
\newcommand{\authname}{Team \#, Team Name
\\ Student 1 name
\\ Student 2 name
\\ Student 3 name
\\ Student 4 name} % AUTHOR NAMES                  

\usepackage{hyperref}
    \hypersetup{colorlinks=true, linkcolor=blue, citecolor=blue, filecolor=blue,
                urlcolor=blue, unicode=false}
    \urlstyle{same}
                                

The purpose of reflection questions is to give you a chance to assess your own
learning and that of your group as a whole, and to find ways to improve in the
future. Reflection is an important part of the learning process.  Reflection is
also an essential component of a successful software development process.  

Reflections are most interesting and useful when they're honest, even if the
stories they tell are imperfect. You will be marked based on your depth of
thought and analysis, and not based on the content of the reflections
themselves. Thus, for full marks we encourage you to answer openly and honestly
and to avoid simply writing ``what you think the evaluator wants to hear.''

Please answer the following questions.  Some questions can be answered on the
team level, but where appropriate, each team member should write their own
response:


\begin{document}

\title{Module Interface Specification for \progname{}}

\author{\authname}

\date{\today}

\maketitle

\pagenumbering{roman}

\section{Revision History}

\begin{tabularx}{\textwidth}{p{3cm}p{2cm}X}
\toprule {\bf Date} & {\bf Version} & {\bf Notes}\\
\midrule
Date 1 & 1.0 & Notes\\
Date 2 & 1.1 & Notes\\
\bottomrule
\end{tabularx}

~\newpage

\section{Symbols, Abbreviations and Acronyms}

See SRS Documentation at \wss{give url}

\wss{Also add any additional symbols, abbreviations or acronyms}

\newpage

\tableofcontents

\newpage

\pagenumbering{arabic}

\section{Introduction}

The following document details the Module Interface Specifications for the
\projname application. Complementary documents include the Module Guide.

The full documentation and implementation can be found at \href{https://github.com/RezaJodeiri/CXR-Capstone}{CXR-Capstone.git}.

\section{Notation}

\wss{You should describe your notation.  You can use what is below as
  a starting point.}

The structure of the MIS for modules comes from \citet{HoffmanAndStrooper1995},
with the addition that template modules have been adapted from
\cite{GhezziEtAl2003}.  The mathematical notation comes from Chapter 3 of
\citet{HoffmanAndStrooper1995}.  For instance, the symbol := is used for a
multiple assignment statement and conditional rules follow the form $(c_1
\Rightarrow r_1 | c_2 \Rightarrow r_2 | ... | c_n \Rightarrow r_n )$.

The following table summarizes the primitive data types used by \progname. 

\begin{center}
\renewcommand{\arraystretch}{1.2}
\noindent 
\begin{tabular}{l l p{7.5cm}} 
\toprule 
\textbf{Data Type} & \textbf{Notation} & \textbf{Description}\\ 
\midrule
character & char & a single symbol or digit\\
integer & $\mathbb{Z}$ & a number without a fractional component in (-$\infty$, $\infty$) \\
natural number & $\mathbb{N}$ & a number without a fractional component in [1, $\infty$) \\
real & $\mathbb{R}$ & any number in (-$\infty$, $\infty$)\\
\bottomrule
\end{tabular} 
\end{center}

\noindent
The specification of \progname \ uses some derived data types: sequences, strings, and
tuples. Sequences are lists filled with elements of the same data type. Strings
are sequences of characters. Tuples contain a list of values, potentially of
different types. In addition, \progname \ uses functions, which
are defined by the data types of their inputs and outputs. Local functions are
described by giving their type signature followed by their specification.

\subsection{Data Types from Libraries}

The following table summarizes the data types provided by external libraries and
used by \projname.

\begin{table}[H]
  \centering
  \renewcommand{\arraystretch}{1.2}
  \noindent
  \begin{tabular}{l l p{7.5cm}}
    \toprule
    \textbf{Data Type}          & \textbf{Notation}           & \textbf{Description}                                                                                                                                                                                                                                            \\
    \midrule                                                                                                                                                                                    \\
    DICOM      & DICOM      & A single media track within a media stream.                                                                                                                                                                                                               \\
    JSON                  & JSON                  & JavaScript Object Notation, it is a text-based open standard data interchange setup and only provides a data encoding specification.                                                                                                                      \\
    PNG                  & PNG                  & Portable Network Graphics, a raster-graphics file format that supports lossless data compression                                                                                                                      \\
    JPEG/JPG                & JPEG/JPG                  & Joint Photographic Experts Group, a commonly used method of lossy compression for digital images, particularly for those images produced by digital photography                                                                                                                      \\
    ASCII                & ASCII                  & American Standard Code for Information Interchange, a character encoding standard used to represent text in computers and electronic devices.                                                                   \\
    \bottomrule
  \end{tabular}
  \caption{Data types from libraries}
\end{table}

\section{Module Decomposition}

The following table is taken directly from the Module Guide document for this project.

\begin{table}[h!]
\centering
\begin{tabular}{p{0.3\textwidth} p{0.6\textwidth}}
\toprule
\textbf{Level 1} & \textbf{Level 2}\\
\midrule

{Hardware-Hiding} & ~ \\
\midrule

\multirow{7}{0.3\textwidth}{Behaviour-Hiding} & Input Parameters\\
& Output Format\\
& Output Verification\\
& Temperature ODEs\\
& Energy Equations\\ 
& Control Module\\
& Specification Parameters Module\\
\midrule

\multirow{3}{0.3\textwidth}{Software Decision} & {Sequence Data Structure}\\
& ODE Solver\\
& Plotting\\
\bottomrule

\end{tabular}
\caption{Module Hierarchy}
\label{TblMH}
\end{table}

\newpage
~\newpage

\section{MIS of Web Application Server Module} 

\subsection{Module}
Web Application Server   
\subsection{Uses}
M2: HTTP Server Module \\ 
M5: Doctor Profile View Module \\ 
M6: Patient List View Module \\ 
\subsection{Syntax}
\subsubsection{Exported Constants}
NA
\subsubsection{Exported Access Programs}

\begin{center}
\begin{tabular}{p{2.5cm} p{3.5cm} p{3.5cm} p{3cm}}
\hline
\textbf{Name} & \textbf{In} & \textbf{Out} & \textbf{Exceptions} \\
\hline
handleRequest & HTTP request & HTTP response & InvalidRequestException \\
\hline
\end{tabular}
\end{center}
% /********************* M1 *********************/
\subsection{Semantics}
\subsubsection{State Variables}
\begin{itemize}
  \item \textbf{sessionData}: Stores current session information.
  \item \textbf{activeUsers}: Keeps track of currently active users.
\end{itemize}
\subsubsection{Environment Variables}
\begin{itemize}
  \item \textbf{serverPort}: The port on which the server listens for incoming connections.
  \item \textbf{hostAddress}: The server's host address.
\end{itemize}
\subsubsection{Assumptions}
\begin{itemize}
  \item Assumes the HTTP Server Module (M2) is properly configured and running.
  \item Assumes valid HTTP requests are received.
\end{itemize}
\subsubsection{Access Routine Semantics}

handleRequest(request)
\begin{itemize}
\item transition:  Processes the incoming HTTP request and routes it to the appropriate view module.
\item output: Returns the HTTP response based on the request.
\end{itemize}
\subsubsection{Local Functions}
\begin{itemize}
  \item \textbf{parseRequest(request)}: Parses the incoming HTTP request to extract necessary information.
  \item \textbf{generateResponse(data)}: Constructs an HTTP response based on the processed data.
  \item \textbf{authenticateUser(credentials)}: Verifies the user's credentials before processing the request.
\end{itemize}
\newpage

\newpage
% /********************* M2 *********************/
\section{HTTP Server Module}

\subsection{Other Modules the Current Module Uses}
\begin{itemize}
    \item M1: Web Application Server Module
    \item M3: Disease Prediction Server Module
    \item M4: Disease Progression Tracking Server Module
    \item M5: Doctor Profile View Module
    \item M6: Patient List View Module
    \item Other view modules (M7-M10) for displaying data.
\end{itemize}

\subsection{State Variables}
\begin{itemize}
    \item \textbf{requestQueue}: Holds incoming HTTP requests until they are processed.
    \item \textbf{responseQueue}: Holds outgoing HTTP responses that need to be sent back to clients.
\end{itemize}

\subsection{Exported Constants and Access Programs}
\subsubsection{Exported Access Programs}
\begin{tabular}{|l|l|l|l|}
    \hline
    \textbf{Name} & \textbf{In} & \textbf{Out} & \textbf{Exceptions} \\
    \hline
    \texttt{startServer} & None & None & \texttt{ServerStartException} \\
    \texttt{stopServer} & None & None & \texttt{ServerStopException} \\
    \texttt{processRequest} & \texttt{HTTP request} & \texttt{HTTP response} & \texttt{InvalidRequestException} \\
    \hline
\end{tabular}

\subsubsection{Exported Constants}
\begin{itemize}
    \item \textbf{SERVER\_PORT}: 8080
    \item \textbf{MAX\_CONNECTIONS}: 100
\end{itemize}

\subsection{Environment Variables}
\begin{itemize}
    \item \textbf{serverPort}: The port on which the server listens for incoming HTTP connections.
    \item \textbf{maxConnections}: Maximum number of simultaneous connections the server can handle.
\end{itemize}

\subsection{Assumptions}
\begin{itemize}
    \item Assumes the Web Application Server Module (M1) is properly configured and running.
    \item Assumes incoming HTTP requests are formatted correctly.
\end{itemize}

\subsection{Access Routine Semantics}
\subsubsection{startServer()}
\begin{itemize}
    \item \textbf{Transition}: Starts the HTTP server, initializes necessary resources, and begins listening for incoming requests.
    \item \textbf{Output}: No output, but may throw a \texttt{ServerStartException} if the server cannot be started.
\end{itemize}

\subsubsection{stopServer()}
\begin{itemize}
    \item \textbf{Transition}: Stops the HTTP server, gracefully shuts down connections.
    \item \textbf{Output}: No output, but may throw a \texttt{ServerStopException} if the server cannot be stopped.
\end{itemize}

\subsubsection{processRequest(request)}
\begin{itemize}
    \item \textbf{Transition}: Takes an incoming HTTP request and processes it, routing it to the appropriate server module or view module.
    \item \textbf{Output}: Returns an HTTP response based on the processed request.
\end{itemize}

\subsection{Local Functions}
\begin{itemize}
    \item \textbf{parseRequest()}: Parses the incoming HTTP request to extract necessary information such as headers and parameters.
    \item \textbf{generateResponse()}: Constructs an HTTP response based on the processed data from the request.
    \item \textbf{handleError()}: Handles errors that arise during request processing and generates appropriate error responses.
\end{itemize}

% /********************* M3 *********************/
\section{Disease Prediction Server Module}

\subsection{Other Modules the Current Module Uses}
\begin{itemize}
    \item M1: Web Application Server Module
    \item M2: HTTP Server Module
    \item M4: Disease Progression Tracking Server Module
    \item M5: Doctor Profile View Module
    \item M6: Patient List View Module
    \item M7: Patient Diseases Progression View Module
\end{itemize}

\subsection{State Variables}
\begin{itemize}
    \item \textbf{model}: The pre-trained model from \texttt{torchxrayvision} used for predicting lung diseases from X-ray images.
    \item \textbf{modelAccuracy}: Tracks the accuracy of the current model after training and validation.
    \item \textbf{predictionThreshold}: A constant threshold to determine the classification outcome (e.g., disease presence).
    \item \textbf{patientImageData}: Holds the chest X-ray image data used for prediction.
\end{itemize}

\subsection{Exported Constants and Access Programs}
\subsubsection{Exported Access Programs}
\begin{tabular}{|l|l|l|l|}
    \hline
    \textbf{Name} & \textbf{In} & \textbf{Out} & \textbf{Exceptions} \\
    \hline
    \texttt{loadModel} & None & Loaded model & \texttt{ModelLoadException} \\
    \texttt{predictDisease} & X-ray image data & Disease prediction & \texttt{InvalidImageException} \\
    \hline
\end{tabular}

\subsubsection{Exported Constants}
\begin{itemize}
    \item \textbf{PREDICTION\_THRESHOLD}: 0.75 (threshold for classification of disease presence)
    \item \textbf{MODEL\_PATH}: Path to the pre-trained model (e.g., \texttt{./models/chest\_xray\_model.pth})
    \item \textbf{MAX\_PREDICTIONS}: 1000 (maximum number of predictions to handle concurrently)
\end{itemize}

\subsection{Environment Variables}
\begin{itemize}
    \item \textbf{modelPath}: The path where the \texttt{torchxrayvision} pre-trained model is saved or loaded from.
    \item \textbf{predictionEndpoint}: The endpoint for making predictions using chest X-ray images.
\end{itemize}

\subsection{Assumptions}
\begin{itemize}
    \item Assumes the pre-trained \texttt{torchxrayvision} model is available and compatible with the data provided.
    \item Assumes valid X-ray image data is available for predictions.
    \item Assumes the Web Application Server Module (M1) and HTTP Server Module (M2) are properly configured and running.
\end{itemize}

\subsection{Access Routine Semantics}
\subsubsection{loadModel()}
\begin{itemize}
    \item \textbf{Transition}: Loads the pre-trained disease prediction model from the specified path after image is uploaded using \texttt{torchxrayvision}.
\end{itemize}

\subsubsection{predictDisease(patientImageData)}
\begin{itemize}
    \item \textbf{Transition}: Uses the loaded model to make predictions based on the provided X-ray image data.
    \item \textbf{Output}: Returns the disease prediction (e.g., probability of a disease being present) or throws an \texttt{InvalidImageException} if the image is invalid.
\end{itemize}

\subsection{Local Functions}
\begin{itemize}
    \item \textbf{loadModel()}: Loads the pre-trained model from disk or cloud storage using \texttt{torchxrayvision}'s functionality.
    \item \textbf{evaluateModel()}: Evaluates the model’s performance with a test dataset to calculate metrics like accuracy and sensitivity.
    \item \textbf{preprocessImage()}: Preprocesses incoming X-ray image data to fit the model's input requirements (e.g., resizing, normalization).
    \item \textbf{postprocessPrediction()}: Processes the raw output from the model (e.g., probabilities) into a human-readable format (e.g., disease labels).
\end{itemize}

% /********************* M4 *********************/
\section{Disease Progression Tracking Server Module}

\subsection{Other Modules the Current Module Uses}
\begin{itemize}
\item M1: Web Application Server Module
\item M2: HTTP Server Module
\item M3: Disease Prediction Server Module
\item M5: Doctor Profile View Module
\item M6: Patient List View Module
\item M7: Patient Overview Module
\item M8: Patient Diseases Progression View Module
\end{itemize}

\subsection{State Variables}
\begin{itemize}
    \item \textbf{progressionData}: Stores historical data of disease progression for each patient.
    \item \textbf{timeStamps}: Records the dates and times when progression data is captured.
    \item \textbf{patientHistory}: Maintains a detailed history of each patient's disease states over time.
\end{itemize}

\subsection{Exported Constants and Access Programs}
\subsubsection{Exported Access Programs}
\begin{tabular}{|l|l|l|l|}
    \hline
    \textbf{Name} & \textbf{In} & \textbf{Out} & \textbf{Exceptions} \\
    \hline
    \texttt{trackProgression} & Patient ID, data & Confirmation of tracking & \texttt{DataNotFoundException} \\
    \hline
    \texttt{getProgressionHistory} & Patient ID & Progression history & \texttt{DataNotFoundException} \\
    \hline
\end{tabular}

\subsubsection{Exported Constants}
\begin{itemize}
\item \textbf{DATA\_RETENTION\_PERIOD}: 5 years (duration for storing disease progression data)
\item \textbf{TRACKING\_INTERVAL}: 30 days (standard interval for recording progression data)
\item \textbf{MAX\_HISTORY\_ENTRIES}: 10000 (maximum number of progression entries per patient)
\end{itemize}

\subsection{Environment Variables}
\begin{itemize}
\item \textbf{dataStoragePath}: The path where progression tracking data is stored.
\item \textbf{updateInterval}: The time interval for automatically updating progression data.
\end{itemize}

\subsection{Assumptions}
\begin{itemize}
\item Assumes valid and accurate disease prediction data is available from M3.
\item Assumes patients' data is regularly updated.
\item Assumes the Web Application Server Module (M1) and HTTP Server Module (M2) are properly configured and running.
\end{itemize}

\subsection{Access Routine Semantics}
\subsubsection{trackProgression(patientID, data)}

\begin{itemize}
    \item \textbf{Transition}: Stores new disease progression data for the given patient.
    \item \textbf{Output}: Returns confirmation of data storage or throws a \texttt{DataNotFoundException} if the patient data is not found.
\end{itemize}

\subsubsection{getProgressionHistory(patientID)}

\begin{itemize}
    \item \textbf{Transition}: Retrieves historical progression data for the specified patient.
    \item \textbf{Output}: Returns the progression history or throws a \texttt{DataNotFoundException} if no history is found.
\end{itemize}


\subsection{Local Functions}
\begin{itemize}
\item \textbf{updateProgressionData()}: Updates the disease progression data at regular intervals based on new predictions or patient information.
\item \textbf{analyzeProgressionTrends()}: Analyzes progression data to identify trends or anomalies in disease progression.
\item \textbf{archiveOldData()}: Moves data older than the retention period to an archive for long-term storage.
\end{itemize}
% /********************* M5 *********************/
\section{Doctor Profile View Module}

\subsection{Other Modules the Current Module Uses}
\begin{itemize}
    \item fill this 
\end{itemize}

\subsection{State Variables}
\begin{itemize}
    \item \textbf{Title}: fill this 
\end{itemize}

\subsection{Exported Constants and Access Programs}
\subsubsection{Exported Access Programs}
\begin{tabular}{|l|l|l|l|}
    \hline
    \textbf{Name} & \textbf{In} & \textbf{Out} & \textbf{Exceptions} \\
    \hline 
    \texttt{fill 1} & fill 2 & fill 3 & \texttt{fill 4} \\
    \hline
    \texttt{fill 1} & fill 2 & fill 3 & \texttt{fill 4} \\
    \hline
\end{tabular}

\subsubsection{Exported Constants}
\begin{itemize}
\item \textbf{Title}: file this 
\end{itemize}

\subsection{Environment Variables}
\begin{itemize}
    \item fill this
\end{itemize}

\subsection{Assumptions}
\begin{itemize}
    \item fill this
\end{itemize}

\subsection{Access Routine Semantics}
\subsubsection{trackProgression(patientID, data)}

\begin{itemize}
    \item fill this
\end{itemize}

\subsubsection{}

\begin{itemize}
    \item fill this
\end{itemize}


\subsection{Local Functions}
\begin{itemize}
    \item fill this
\end{itemize}
% /********************* M6 *********************/
\section{Patient List View Module}

\subsection{Other Modules the Current Module Uses}
\begin{itemize}
    \item M1: Web Application Server Module
    \item M2: HTTP Server Module
    \item M11: Patient Medical Record View Module
    \item M15: Patient Medical Record Update Module
\end{itemize}

\subsection{State Variables}
\begin{itemize}
\item \textbf{patientList}: Stores a list of patients assigned to a doctor, including their basic information such as name, age, and medical condition.
\item \textbf{searchFilters}: Stores the current filters applied to the patient list for sorting and searching purposes.
\end{itemize}

\subsection{Exported Constants and Access Programs}
\subsubsection{Exported Access Programs}
\begin{tabular}{|l|l|l|l|}
    \hline
    \textbf{Name} & \textbf{In} & \textbf{Out} & \textbf{Exceptions} \\
    \hline 
    \texttt{viewPatientList} & Doctor ID, filters & List of patient details & \texttt{PatientListNotFoundException} \\
    \hline
    \texttt{applyFilter} & Filter parameters & Filtered patient list & \texttt{InvalidFilterException} \\
    \hline
    \texttt{sortList} & Sorting criteria & Sorted patient list & None \\
    \hline
\end{tabular}

\subsubsection{Exported Constants}
\begin{itemize}
\item \textbf{DEFAULT\_SORT\_ORDER}: "alphabetical" (default order in which patients are listed)
\end{itemize}

\subsection{Environment Variables}
\begin{itemize}
    \item fill this
\end{itemize}

\subsection{Assumptions}
\begin{itemize}
    \item fill this
\end{itemize}

\subsection{Access Routine Semantics}
\subsubsection{trackProgression(patientID, data)}

\begin{itemize}
    \item fill this
\end{itemize}

\subsubsection{}

\begin{itemize}
    \item fill this
\end{itemize}


\subsection{Local Functions}
\begin{itemize}
    \item fill this
\end{itemize}
% /********************* M7 *********************/
\section{Module NAME HERE!!!}

\subsection{Other Modules the Current Module Uses}
\begin{itemize}
    \item fill this 
\end{itemize}

\subsection{State Variables}
\begin{itemize}
    \item \textbf{Title}: fill this 
\end{itemize}

\subsection{Exported Constants and Access Programs}
\subsubsection{Exported Access Programs}
\begin{tabular}{|l|l|l|l|}
    \hline
    \textbf{Name} & \textbf{In} & \textbf{Out} & \textbf{Exceptions} \\
    \hline 
    \texttt{fill 1} & fill 2 & fill 3 & \texttt{fill 4} \\
    \hline
    \texttt{fill 1} & fill 2 & fill 3 & \texttt{fill 4} \\
    \hline
\end{tabular}

\subsubsection{Exported Constants}
\begin{itemize}
\item \textbf{Title}: file this 
\end{itemize}

\subsection{Environment Variables}
\begin{itemize}
    \item fill this
\end{itemize}

\subsection{Assumptions}
\begin{itemize}
    \item fill this
\end{itemize}

\subsection{Access Routine Semantics}
\subsubsection{trackProgression(patientID, data)}

\begin{itemize}
    \item fill this
\end{itemize}

\subsubsection{}

\begin{itemize}
    \item fill this
\end{itemize}


\subsection{Local Functions}
\begin{itemize}
    \item fill this
\end{itemize}
% /********************* M8 *********************/
\section{Module NAME HERE!!!}

\subsection{Other Modules the Current Module Uses}
\begin{itemize}
    \item fill this 
\end{itemize}

\subsection{State Variables}
\begin{itemize}
    \item \textbf{Title}: fill this 
\end{itemize}

\subsection{Exported Constants and Access Programs}
\subsubsection{Exported Access Programs}
\begin{tabular}{|l|l|l|l|}
    \hline
    \textbf{Name} & \textbf{In} & \textbf{Out} & \textbf{Exceptions} \\
    \hline 
    \texttt{fill 1} & fill 2 & fill 3 & \texttt{fill 4} \\
    \hline
    \texttt{fill 1} & fill 2 & fill 3 & \texttt{fill 4} \\
    \hline
\end{tabular}

\subsubsection{Exported Constants}
\begin{itemize}
\item \textbf{Title}: file this 
\end{itemize}

\subsection{Environment Variables}
\begin{itemize}
    \item fill this
\end{itemize}

\subsection{Assumptions}
\begin{itemize}
    \item fill this
\end{itemize}

\subsection{Access Routine Semantics}
\subsubsection{trackProgression(patientID, data)}

\begin{itemize}
    \item fill this
\end{itemize}

\subsubsection{}

\begin{itemize}
    \item fill this
\end{itemize}


\subsection{Local Functions}
\begin{itemize}
    \item fill this
\end{itemize}
% /********************* M9 *********************/
\section{Module NAME HERE!!!}

\subsection{Other Modules the Current Module Uses}
\begin{itemize}
    \item fill this 
\end{itemize}

\subsection{State Variables}
\begin{itemize}
    \item \textbf{Title}: fill this 
\end{itemize}

\subsection{Exported Constants and Access Programs}
\subsubsection{Exported Access Programs}
\begin{tabular}{|l|l|l|l|}
    \hline
    \textbf{Name} & \textbf{In} & \textbf{Out} & \textbf{Exceptions} \\
    \hline 
    \texttt{fill 1} & fill 2 & fill 3 & \texttt{fill 4} \\
    \hline
    \texttt{fill 1} & fill 2 & fill 3 & \texttt{fill 4} \\
    \hline
\end{tabular}

\subsubsection{Exported Constants}
\begin{itemize}
\item \textbf{Title}: file this 
\end{itemize}

\subsection{Environment Variables}
\begin{itemize}
    \item fill this
\end{itemize}

\subsection{Assumptions}
\begin{itemize}
    \item fill this
\end{itemize}

\subsection{Access Routine Semantics}
\subsubsection{trackProgression(patientID, data)}

\begin{itemize}
    \item fill this
\end{itemize}

\subsubsection{}

\begin{itemize}
    \item fill this
\end{itemize}


\subsection{Local Functions}
\begin{itemize}
    \item fill this
\end{itemize}
% /********************* M10 *********************/
\section{Patient Medical Record View Module}

\subsection{Other Modules the Current Module Uses}
\begin{itemize}
    \item fill this 
\end{itemize}

\subsection{State Variables}
\begin{itemize}
    \item \textbf{Title}: fill this 
\end{itemize}

\subsection{Exported Constants and Access Programs}
\subsubsection{Exported Access Programs}
\begin{tabular}{|l|l|l|l|}
    \hline
    \textbf{Name} & \textbf{In} & \textbf{Out} & \textbf{Exceptions} \\
    \hline 
    \texttt{fill 1} & fill 2 & fill 3 & \texttt{fill 4} \\
    \hline
    \texttt{fill 1} & fill 2 & fill 3 & \texttt{fill 4} \\
    \hline
\end{tabular}

\subsubsection{Exported Constants}
\begin{itemize}
\item \textbf{Title}: file this 
\end{itemize}

\subsection{Environment Variables}
\begin{itemize}
    \item fill this
\end{itemize}

\subsection{Assumptions}
\begin{itemize}
    \item fill this
\end{itemize}

\subsection{Access Routine Semantics}
\subsubsection{trackProgression(patientID, data)}

\begin{itemize}
    \item fill this
\end{itemize}

\subsubsection{}

\begin{itemize}
    \item fill this
\end{itemize}


\subsection{Local Functions}
\begin{itemize}
    \item fill this
\end{itemize}
% /********************* M11 *********************/
\section{Module NAME HERE!!!}

\subsection{Other Modules the Current Module Uses}
\begin{itemize}
    \item fill this 
\end{itemize}

\subsection{State Variables}
\begin{itemize}
    \item \textbf{Title}: fill this 
\end{itemize}

\subsection{Exported Constants and Access Programs}
\subsubsection{Exported Access Programs}
\begin{tabular}{|l|l|l|l|}
    \hline
    \textbf{Name} & \textbf{In} & \textbf{Out} & \textbf{Exceptions} \\
    \hline 
    \texttt{fill 1} & fill 2 & fill 3 & \texttt{fill 4} \\
    \hline
    \texttt{fill 1} & fill 2 & fill 3 & \texttt{fill 4} \\
    \hline
\end{tabular}

\subsubsection{Exported Constants}
\begin{itemize}
\item \textbf{Title}: file this 
\end{itemize}

\subsection{Environment Variables}
\begin{itemize}
    \item fill this
\end{itemize}

\subsection{Assumptions}
\begin{itemize}
    \item fill this
\end{itemize}

\subsection{Access Routine Semantics}
\subsubsection{trackProgression(patientID, data)}

\begin{itemize}
    \item fill this
\end{itemize}

\subsubsection{}

\begin{itemize}
    \item fill this
\end{itemize}

\subsection{Local Functions}
\begin{itemize}
    \item fill this
\end{itemize}
% /********************* M12 *********************/
\section{Module NAME HERE!!!}

\subsection{Other Modules the Current Module Uses}
\begin{itemize}
    \item fill this 
\end{itemize}

\subsection{State Variables}
\begin{itemize}
    \item \textbf{Title}: fill this 
\end{itemize}

\subsection{Exported Constants and Access Programs}
\subsubsection{Exported Access Programs}
\begin{tabular}{|l|l|l|l|}
    \hline
    \textbf{Name} & \textbf{In} & \textbf{Out} & \textbf{Exceptions} \\
    \hline 
    \texttt{fill 1} & fill 2 & fill 3 & \texttt{fill 4} \\
    \hline
    \texttt{fill 1} & fill 2 & fill 3 & \texttt{fill 4} \\
    \hline
\end{tabular}

\subsubsection{Exported Constants}
\begin{itemize}
\item \textbf{Title}: file this 
\end{itemize}

\subsection{Environment Variables}
\begin{itemize}
    \item fill this
\end{itemize}

\subsection{Assumptions}
\begin{itemize}
    \item fill this
\end{itemize}

\subsection{Access Routine Semantics}
\subsubsection{trackProgression(patientID, data)}

\begin{itemize}
    \item fill this
\end{itemize}

\subsubsection{}

\begin{itemize}
    \item fill this
\end{itemize}


\subsection{Local Functions}
\begin{itemize}
    \item fill this
\end{itemize}
% /********************* M13 *********************/
\section{Module NAME HERE!!!}

\subsection{Other Modules the Current Module Uses}
\begin{itemize}
    \item fill this 
\end{itemize}

\subsection{State Variables}
\begin{itemize}
    \item \textbf{Title}: fill this 
\end{itemize}

\subsection{Exported Constants and Access Programs}
\subsubsection{Exported Access Programs}
\begin{tabular}{|l|l|l|l|}
    \hline
    \textbf{Name} & \textbf{In} & \textbf{Out} & \textbf{Exceptions} \\
    \hline 
    \texttt{fill 1} & fill 2 & fill 3 & \texttt{fill 4} \\
    \hline
    \texttt{fill 1} & fill 2 & fill 3 & \texttt{fill 4} \\
    \hline
\end{tabular}

\subsubsection{Exported Constants}
\begin{itemize}
\item \textbf{Title}: file this 
\end{itemize}

\subsection{Environment Variables}
\begin{itemize}
    \item fill this
\end{itemize}

\subsection{Assumptions}
\begin{itemize}
    \item fill this
\end{itemize}

\subsection{Access Routine Semantics}
\subsubsection{trackProgression(patientID, data)}

\begin{itemize}
    \item fill this
\end{itemize}

\subsubsection{}

\begin{itemize}
    \item fill this
\end{itemize}


\subsection{Local Functions}
\begin{itemize}
    \item fill this
\end{itemize}
% /********************* M14 *********************/
\section{Module NAME HERE!!!}

\subsection{Other Modules the Current Module Uses}
\begin{itemize}
    \item fill this 
\end{itemize}

\subsection{State Variables}
\begin{itemize}
    \item \textbf{Title}: fill this 
\end{itemize}

\subsection{Exported Constants and Access Programs}
\subsubsection{Exported Access Programs}
\begin{tabular}{|l|l|l|l|}
    \hline
    \textbf{Name} & \textbf{In} & \textbf{Out} & \textbf{Exceptions} \\
    \hline 
    \texttt{fill 1} & fill 2 & fill 3 & \texttt{fill 4} \\
    \hline
    \texttt{fill 1} & fill 2 & fill 3 & \texttt{fill 4} \\
    \hline
\end{tabular}

\subsubsection{Exported Constants}
\begin{itemize}
\item \textbf{Title}: file this 
\end{itemize}

\subsection{Environment Variables}
\begin{itemize}
    \item fill this
\end{itemize}

\subsection{Assumptions}
\begin{itemize}
    \item fill this
\end{itemize}

\subsection{Access Routine Semantics}
\subsubsection{trackProgression(patientID, data)}

\begin{itemize}
    \item fill this
\end{itemize}

\subsubsection{}

\begin{itemize}
    \item fill this
\end{itemize}


\subsection{Local Functions}
\begin{itemize}
    \item fill this
\end{itemize}
% /********************* M15 *********************/
\section{Module NAME HERE!!!}

\subsection{Other Modules the Current Module Uses}
\begin{itemize}
    \item fill this 
\end{itemize}

\subsection{State Variables}
\begin{itemize}
    \item \textbf{Title}: fill this 
\end{itemize}

\subsection{Exported Constants and Access Programs}
\subsubsection{Exported Access Programs}
\begin{tabular}{|l|l|l|l|}
    \hline
    \textbf{Name} & \textbf{In} & \textbf{Out} & \textbf{Exceptions} \\
    \hline 
    \texttt{fill 1} & fill 2 & fill 3 & \texttt{fill 4} \\
    \hline
    \texttt{fill 1} & fill 2 & fill 3 & \texttt{fill 4} \\
    \hline
\end{tabular}

\subsubsection{Exported Constants}
\begin{itemize}
\item \textbf{Title}: file this 
\end{itemize}

\subsection{Environment Variables}
\begin{itemize}
    \item fill this
\end{itemize}

\subsection{Assumptions}
\begin{itemize}
    \item fill this
\end{itemize}

\subsection{Access Routine Semantics}
\subsubsection{trackProgression(patientID, data)}

\begin{itemize}
    \item fill this
\end{itemize}

\subsubsection{}

\begin{itemize}
    \item fill this
\end{itemize}


\subsection{Local Functions}
\begin{itemize}
    \item fill this
\end{itemize}

\bibliographystyle {plainnat}
\bibliography {../../../refs/References}

\newpage

\section{Appendix} \label{Appendix}

\wss{Extra information if required}

\newpage{}

\section*{Appendix --- Reflection}

\wss{Not required for CAS 741 projects}

The information in this section will be used to evaluate the team members on the
graduate attribute of Problem Analysis and Design.



\begin{enumerate}
  \item What went well while writing this deliverable? 
  \item What pain points did you experience during this deliverable, and how
    did you resolve them?
  \item Which of your design decisions stemmed from speaking to your client(s)
  or a proxy (e.g. your peers, stakeholders, potential users)? For those that
  were not, why, and where did they come from?
  \item While creating the design doc, what parts of your other documents (e.g.
  requirements, hazard analysis, etc), it any, needed to be changed, and why?
  \item What are the limitations of your solution?  Put another way, given
  unlimited resources, what could you do to make the project better? (LO\_ProbSolutions)
  \item Give a brief overview of other design solutions you considered.  What
  are the benefits and tradeoffs of those other designs compared with the chosen
  design?  From all the potential options, why did you select the documented design?
  (LO\_Explores)
\end{enumerate}


\end{document}